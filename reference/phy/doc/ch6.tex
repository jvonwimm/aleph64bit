\chapter{\label{sec-M}Mnemonic symbols}
\par
Mnemonic symbols are Fortran variables, arrays, parameters, functions,
or
statement functions.
Mnemonic symbols which
give access to information for specific reconstructed or Monte Carlo
objects are described in Chapter \ref{sec-TVA}.
When possible, the names of the mnemonic symbols
follow the same convention as the HAC parameters.
 
The units used in ALPHA are  cm, sec, GeV, GeV/c, GeV/c$^2$, kG.
 
\section{\label{sec-MCP}Mathematical and physical constants}
\par
\begin{indentlist}{ 2.50cm}{ 2.75cm}
\indentitem{QQE}e = 2.718282
\indentitem{QQPI}$\pi$ = 3.141593
\indentitem{QQ2PI}$2\pi$
\indentitem{QQPIH}$\pi/2$
\indentitem{QQRADP}$180/\pi$
\indentitem{QQC}speed of light = 2.997925E10 cm/sec
\indentitem{QQIRP}speed of light in units cm / KGauss
(inverse track bending radius $\rightarrow$ track momentum)
\indentitem{QQH}Planck constant / 2$\pi$ = 6.582173E$-$25 GeV sec
\indentitem{QQHC}QQH * QQC
\end{indentlist}
Note: The standard ALEPH constants (ALCONS) are also available.
\section{\label{sec-MR}Run information}
\par
\begin{indentlist}{ 2.50cm}{ 2.75cm}
\indentitem{KRINNE }number of events in run (with HV on)
\indentitem{KRINLF }LEP fill number
\indentitem{KRINDQ }data quality (see bank description for RLUM variable RQ)
\indentitem{XIOKLU }Logical flag,  .TRUE. if Luminosity obtained from LCAL
(for LEP I runs before 16600 , and all LEP II runs )
\indentitem{QRINLU }Luminosity from LCAL in nb**-1 , from database , if XIOKLU = .TRUE.
\indentitem{XIOKSI }Logical flag,  .TRUE. if Luminosity obtained from SiCAL
(for LEP I runs after 16600 ; always .FALSE. for LEP II runs )
\indentitem{QRSLLU }Luminosity from SiCAL in nb**-1 , from database , if XIOKSI = .TRUE.
\indentitem{KRINNZ }number of Z $\to$ hadrons  (from database if available) .
                    This is only an estimation obtained using CLAS 16 events with XLUMOK = .TRUE.
\indentitem{KRINNB }number of Bhabhas (from database if available)

\indentitem{QMFLD  }magnetic field (best estimate), in kGauss. Taken from
data card FIEL (if given),
run header bank RALE, or run header bank AFID for MC events.
If ABS(QMFLD) $>$ 20 QMFLD is set to 15.

\indentitem{QVXNOM, QVYNOM, QVZNOM }run-by run mean position of interaction
point;
taken from the database bank LFIL (see KBPSTA below).\\
{\bf Important warning: These quantities are to be considered as obsolete, except for 1989-1990 data.}

 For all other years of data taking, and for all Monte Carlo datasets, they are strictly equivalent to
 QVTXBP(I) described in the next paragraph, \ref{sec-CHUNKINF} on p. ~\pageref{sec-CHUNKINF}.

\indentitem{QVXNSG, QVYNSG, QVZNSG }(Statistical error) $^2$ on QVXNOM
etc.;

 {\bf Obsolete, kept only for backward-compatibility}.

\indentitem{KBPSTA}Beam Position Status: method used to determine
QVXNOM, QVYNOM, etc.

 {\bf Obsolete, kept only for backward-compatibility}.
\begin{itemize}
\item KBPSTA = 0:  No mean beam position for the run
\item  KBPSTA = 1:  Mean beam position computed with VDET, per RUN
(most of  runs taken from 1991 to 1994)
\item  KBPSTA = 2:  Mean beam position computed with VDET, per FILL
(this happens when the run has too few events)
\item  KBPSTA = 3:  Mean beam position computed WITHOUT VDET, per FILL
(this is the status of all runs in 1989/1990 and
of the runs in 1991 which do not have a working VDET)
\end{itemize}

\indentitem{QDBOFS }average systematic offset of D0.

 {\bf Obsolete, kept only for backward-compatibility}.

\end{indentlist}

\par
\section{\label{sec-CHUNKELEP}LEP c.m.s. energy  QELEP}
\par
The ALPHA variable {\bf QELEP} gives the best available value of the LEP c.m.s. energy, in GeV.

This is intermediate between a Run information and an event-by-event information.

                    Since ALPHA 123.08 (October 1998) this value is obtained as follows:

\begin{itemize}

\item                    For Monte Carlo events, QELEP is the generated LEP energy.

\item                    For about 99\% of LEP2 events
                     QELEP is  obtained from the 15 minutes time-chunk determinations
                    of the LEP c.m.s energy in ALEPH, as given by the LEP Energy Group (values stored in the database bank RNL2).
                    See the note ALEPH 97-029 (PHYSIC 97-24) for more details.

\item                    For LEP1 events, or for a tiny fraction of LEP2 events without time-chunk, QELEP is
                    obtained from run-averaged energies (values stored in the database bank RNR2).

           
\item               If none of the above values is available (in general because the run is too short), QELEP is taken as the
                    average LEP energy over the corresponding LEP fill (values stored in the database bank RNF2).

\end{itemize}

                    For a given event, to know how the LEP energy was obtained, please use the subroutine QWHICH$\_$EN
                    described in \ref{sec-ELEP2} on p.~\pageref{sec-ELEP2}.

                    {\bf Important: Please DO NOT call any more the subroutines  QELEP2 or QELEP1}.
  


\par
\section{\label{sec-CHUNKINF}Chunk-by-chunk luminous region information}
\par
 The quantities described below are used internally by the QFNDIP package
  (sec. \ref{sec-DCQFND} on page ~\pageref{sec-DCQFND}). They give the best possible precision
 and  should therefore
 be preferred, when available, to  QVXNOM,QVYNOM,
QVZNOM described above.
\par
\begin{indentlist}{ 2.50cm}{ 2.75cm}
 
\indentitem{XGETBP}
 = .TRUE. if the event-chunk beam information from QFGET$\_$BP
           is available.  The quantities listed below are defined only if:\\
            XGETBP = .TRUE.
 
\indentitem{QVTXBP(I)}  I=1,2,3: x,y,z position of luminous region centroid (cm).
           Notice: QVTXBP(3) is always equal to zero.
 
\indentitem{QVTEBP(I)}  I=1,2,3: Estimated uncertainty on QVTXBP(I) (cm).
           Notice: QVTEBP(3) is always equal to 1.
 
\indentitem{QVTSBP(I)}  I=1,2,3: x,y,z rms size of luminous region (cm), averaged
           over the year (or run period). The routine QBEAMX
           (see section \ref{sec-OAQBEAM}) provides access to the metachunk-by-metachunk
           values of sigma$\_$x.
 
\indentitem{Real data:}
The beam position x and y and their uncertainties returned in QVTXBP
and QVTEBP are taken from the ALPB bank in the run record, or from the
beam.position file if the dataset was produced by JULIA version $<$  275.03 .  The sizes of the luminous
region returned in QVTSBP are taken from the ALRP database bank and are
average values for the year (or run period).
 
\indentitem{Monte Carlo:}
The beam position x and y returned in QVTXBP are generated so as to
simulate the desired beam size distribution.  The primary vertex position
of the event is extracted from truth information and an offset is added
according to the size of the luminous region.  The smearing corresponding
to the average chunk-by-chunk beam position uncertainty (QVTEBP) is
included.
 
 The ``desired'' beam size distribution is by default taken from
measurements of the actual LEP conditions for the geometry year of the
Monte Carlo file.  This default may be changed by means of data cards ( BPER,BPWT,BSIZ,SIBE) as
explained in section \ref{sec-DCHUNK}.
 
The horizontal size of the luminous region
$\sigma_x$  is thrown at random from the measured distribution, which is
parametrized as the sum of two Gaussians (not a delta function).  The
uncertainties in x and y returned in QVTEBP, as well as the three sizes
returned in QVTSBP, are taken from the ALRP database bank and are average
values for the year (or run period).
\end{indentlist}


{\bf Important remarks:}

To know how the beam position information was obtained for a given event, use the
subroutine QWHICH$\_$BP (see Sec \ref{sec-EBSPOT} on p.~\pageref{sec-EBSPOT}).

Another way of obtaining the beam position is to use the LEP BOMs (for LEP2 data only). See
the description of the data card BOBS in Sec \ref{sec-EBSPOT} on p.~\pageref{sec-EBSPOT})
and the subroutine QFILBP$\_$STATUS to know how the beam spot was found for a given event in
Sec \ref{sec-BOMSTAT} on p.~\pageref{sec-BOMSTAT}.
 
\section{\label{sec-MH}Event information}
\par
\subsection{\label{sec-MHE}Event header: from bank EVEH}
\par
\begin{indentlist}{ 2.50cm}{ 2.75cm}
\indentitem{KRUN         }Run Number
\indentitem{KEVT         }Event Number 
\indentitem{KEVEDA       }DAte
\indentitem{KEVETI       }TIme
\indentitem{KEVEMI(I)    }trigger Mask I, I = 1 to 4
\indentitem{KEVETY       }event TYpe
\indentitem{KEVEES       }Error Status
\end{indentlist}
\subsection{\label{sec-MED}Event directory information}
\par
\begin{indentlist}{ 2.50cm}{ 2.75cm}
\indentitem{KCLASW       }EDIR class word.
                          See App. G for the definition of event classes.

\end{indentlist}
\subsection{\label{sec-MHK}Event generator status: from bank KEVH}
\par
\begin{indentlist}{ 2.50cm}{ 2.75cm}
\indentitem{KKEVID       }process ID
\indentitem{QKEVWT       }WeighT
\end{indentlist}
\subsection{\label{sec-MHR}Detector HV status: from banks REVH, LOLE}
\par
\begin{indentlist}{ 2.50cm}{ 2.75cm}
\indentitem{XHVTRG   }=.TRUE. if XLUMOK checks are satisfied. (To
save time,
XHVTRG should be used rather than calling the function XLUMOK.)
\indentitem{KREVDS   }Detector status word from REVH bank
\indentitem{XVLCAL   }= .TRUE. if the LCAL is OK (i.e., the LOLE bank
is present and
there is no error condition)
\indentitem{XVSATR   }= .TRUE. if SATR HV is OK
\indentitem{XVITC    }= .TRUE. if ITC HV is OK
\indentitem{XVTPC    }= .TRUE. if TPC HV is OK  (bit 15)
\indentitem{XVTPCD   }= .TRUE. if all TPC HV is OK (dE/dx bit $-$
bit 4)
\indentitem{XVECAL   }= .TRUE. if ECAL HV is OK (i.e., all ECAL HV
bits are on)
\indentitem{XVHCAL   }= .TRUE. if HCAL HV is OK (i.e., all HCAL HV
bits are on)
\end{indentlist}
\noindent Note: For VDET HV status, see Sec.~\ref{sec-XVDEOK}.
\par
\subsection{\label{sec-TRIG}Trigger Information: from XTEB or XTRB,
XTCN}
\par
\begin{indentlist}{ 2.50cm}{ 2.75cm}
\indentitem{KXTET1 }Level 1 trigger bit pattern
\indentitem{KXTET2 }Level 2 trigger bit pattern
\indentitem{KXTEL2 }Level 2 bit pattern after applying the enabled
trigger mask
\indentitem{KXTCGC }Number of GBXs since the last event readout
\indentitem{KXTCLL }Number of level 1 yes conditions since the last
event readout
\indentitem{KXTCBN }$e^-$ bunch number
\indentitem{KXTCCL }level 1 control word
\indentitem{KXTCHV }HV status word (equivalent to KREVDS above)
\indentitem{KXTCEN }mask of enabled triggers
\end{indentlist}
\par
\subsection{\label{sec-BOM}Beam position from BOM system: from bank BOMB}
\par
\begin{indentlist}{ 2.50cm}{ 2.75cm}
\indentitem{QVXBOM       }$x$ beam position from BOM
\indentitem{QVYBOM       }$y$ beam position from BOM
\indentitem{KERBOM       }Error code for BOM
\begin{itemize}
\item $<$ 0 fatal error, QVXBOM and QVYBOM are filled with QVXNOM and QVYNOM
\item = 0 BOM data good
\item = 1 BOM data in $x$ disagrees with VDET average
\item = 2 BOM data in $y$ disagrees with VDET average
\item = 3 BOM data in both $x$ and $y$ disagrees with VDET average
\end{itemize}
\end{indentlist}
\section{\label{sec-ECWI}ECAL Wire Energies}
\begin{indentlist}{ 2.50cm}{ 2.75cm}
\par
\indentitem{QEECWI(IMOD) }ECAL wire energy summed over the whole module IMOD in GeV.
Modules 1 $-$ 12 refer to endcap A, 13 $-$ 24 to the barrel, and 25
$-$ 36 to endcap
B.
\end{indentlist}

Please refer to subroutine QEWSUM (see \ref{sec-EWSU2} on p.~\pageref{sec-EWSU2}) to get
the separate wire energies on even/odd planes of the ECAL modules.

\section{\label{sec-MC}ALPHA Internal Constants, Variables}
\par
\subsection{\label{sec-MCN}Event counts}
\par
\begin{indentlist}{ 2.50cm}{ 2.75cm}
\indentitem{KNEVT}Total number of events read in
\indentitem{KNEFIL}Number of events read from the current input file
\indentitem{KNREIN}Number of records read from the current input file
(including run records)
\indentitem{KNEOUT}Number of events written to the output file
\end{indentlist}
\subsection{\label{sec-MCS}Program status}
\par
\begin{indentlist}{ 2.50cm}{ 2.75cm}
\indentitem{KSTATU}$-$1: program initialization;
0: event processing;
1: program termination
\indentitem{KDEBUG}debug level (see \ref{sec-DCDEBU})
\end{indentlist}
\subsection{\label{sec-MCE}Event status}
\par
\begin{indentlist}{ 2.50cm}{ 2.75cm}
\indentitem{XMINI}{\bf = .TRUE.} if event read from Mini-DST
{\bf = .FALSE.} if event read from POT or DST
\indentitem{XNANO}{\bf = .TRUE.} if event read from Nano-DST
{\bf = .FALSE.} if event read from POT or DST or MINI
\indentitem{XMCEV}{\bf = .TRUE.} if MC truth available for the event
\end{indentlist}
\subsection{\label{sec-MCU}Input / output units}
\par
\begin{indentlist}{ 2.50cm}{ 2.75cm}
\indentitem{KUINPU}event input = 20, 21
\indentitem{KUOUTP}event output = 50
\indentitem{KUEDIN}EDIR (event directory) input = 30
\indentitem{KUEDOU}EDIR (event directory) output = 60
\indentitem{KUCONS}data base = 4
\indentitem{KUPRNT}log file = 6
\indentitem{KUPTER}terminal = 76 or 0 (see Ch.~\ref{sec-GS})
\indentitem{KUCARD}card input =  7
\indentitem{KUCAR2}second card input = 8
\indentitem{KUHIST}histogram output unit = 15
\indentitem{KURTOX}EXCH format histogram output on UNIX machines = 16
\end{indentlist}
Note: Units 90, 91, 92, and 93 are always free for
private output files.
\subsection{\label{sec-MCT}Timing}
\par
\begin{indentlist}{ 2.50cm}{ 2.75cm}
\indentitem{QTIMEL}time remaining before time limit
\indentitem{QTIME}seconds given on the TIME card (see \ref{sec-DCTIME})
\end{indentlist}
On all CERN computers, the time units are
IBM 370/168 seconds.
\subsection{\label{sec-MCC}Character variables}
\par
\begin{indentlist}{ 2.50cm}{ 2.75cm}
\indentitem{CQVERS}ALPHA version number e.g. 121.01 (6 char)
\indentitem{CQDATE}date at start of job (8 char)
\indentitem{CQTIME}time at start of job (8 char)
\indentitem{CQFOUT}data set name of event output file
= ` ' if no output file given
\end{indentlist}
