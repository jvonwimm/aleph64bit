\chapter{\label{sec-miniapp}Using the Mini-DST with ALPHA}
\par
In this appendix, a brief introduction to the use of the Mini-DST
with ALPHA is given. For more details, consult the current
ALEPH Mini-DST User's Guide (version 8.0 or later).

Since ALPHA version 124 (May 1999), the MINI code is fully integrated within ALPHA, both
for reading and for writing MINIs. All MINI routines are now in the ALPHA library.

The source code for all MINI routines can be found in the directory:

\$ALROOT/alphavsn/mini     (vsn: current ALPHA version number, e.g. 124). 

An executable to make MINIs with all standard banks    is available in:

\$ALEPH/phy/miniprod

{\bf Important warning: Do not use old MINI libraries any more. Please use now
 only the ALPHA library which contains all the necessary MINI features.} 

\par
\section{\label{sec-rtm}Doing analysis with the Mini}
\par
The Mini-DST file(s) to be read (data or event directory) should be
declared with a standard FILI card.
ALPHA will convert the Mini-DST banks to POT banks, and the ALPHA
variables will be filled.
The logical XMINI will be set to .TRUE..
\par
Since the Mini-DST contains many of the basic
quantities from the POT/DST, most of the ALPHA quantities are available
from the Mini-DST, and most of the ALPHA code will work.
Below is a list of quantities which are NOT available on the
Mini-DST -- there may be others.
\begin{itemize}
\item XTCN: all
\item FRFT: KFRFNO.
\item FRTL: KFRTNE, KFRTNR.
\item FRID: KFRIBC, KFRIBC, KFRIPE, KFRIPM, KFRIPI, KFRIPK, KFRIPP, KFRINK.
\item EIDT: QEIDEC, QEIDRI(I,N=1,4,5).
\item PECO: QPECEC, KPECKD, KPEPPC.
\item PEPT: all.
\item PHCO: QPHCER, KPHCKD, KPHCCC, KPHCRB, KPHCPC.
\item YV0V: KYV0IC.
\item EGPC: QEGPR1, QEGPR2, QEGPF4, QEGPDM, KEGPST.  Note that EGPC is obsolete since MINI 9.0 .
\item EFOL: KEFOLC.
\item Vertex: QVEM.
\end{itemize}
The energy flow stored on the Mini-DST corresponds to the ENFLW
algorithm;  PCPA was available also until MINI 8.6 .
\par
All of ALPHA's event topology routines work with the Mini-DST.  Also,
the routines
described in Ch.~\ref{sec-OARD} may be used.
In particular, QMUIDO, QDEDXM, QPAIRF, QVDHIT, and XVDEOK will work.
\par
A few packages cannot run on the MINI, e.g. QEWSUM for datasets written with MINI 10.2 and before, or VDHMATCH for all MINIs.


\par
\section{\label{sec-dbpam}Differences between POT/DST and Mini-DST}
\par
In addition to the absence of some information, other differences
may be
observed when using the Mini.  The Mini-DST has limited precision,
so some quantities
will have slightly different values from those found when reading
the POT or DST.
Non-trivial differences are:
\begin{itemize}
\item FRFT/0 is generally not available on the Mini-DST.
\item For ITC tracks, the z$\sb{0}$ error is limited to 25cm, which
increases QBC2 substantially.
\item If the total number of vertices is greater than 30, only the main
vertices and the best V$\sp{0}$ vertices (ordered
by chi-squared) up to a total of 30 will be available when reading
the Mini-DST.
\item Further, the V$\sp{0}$
quantities available are the result of swimming the tracks,
and if greater precision is required, a refit must be performed.
Infrequently, problems arise when the charge of fitted track
(FRFT/2) is different from that of the original track used (FRFT/0),
and hence the two V$^0$ tracks appear to have the same charge.
\item The chi-squared and NDF for the helix fit for charged tracks may not
be the same as those found for the DST, but the chi-squared per NDF
should be correct within the foreseen precision.
\item The values of R2 and R3 are truncated in the range [-10.23,+10.23],
but where there are indications of problems, the values are set to +1000.
\item For old MINIs ( 8.6 and before ) with GAMPEK photons, only the presence of dead stories is recorded,
not information for each storey. Hence DST1,2,3 of the variable KEGPQU are
all 0 or all 1.
\end{itemize}

\par
\section{\label{sec-mlep}Differences between LEP1 and LEP2 MINIs}
\par

As requested by the Physics Analysis groups, the MINIs for LEP2 datasets (both real data and Monte Carlo)
 contain several banks in addition to the ones written on LEP1 MINIs: 
\begin{itemize}
\item used by EFLOW:\\
  PHCO,HTUB,PEWI,PWEI,EFOL,EIDT
\item muon identification:\\
  MHIT,MCAD,HMAD,MUID,MUEX,MTHR,HPDI,HROA
\item related to ${\mathrm{BCAL^{++}}}$:\\
  BCTR,BCSC,BCSL,BCGN,BCHG
\item HCAL calibration:\\
  HT0C,HCLB,HCCV
\item HCAL slow control:\\
  HSSR,HCNF
\item ITC coordinates:\\
  PIDI ($\rightarrow$ ITCO)
\item LCAL crack PMs :\\
  LCRA
\item SiCAL cluster identification:\\
  SIID


\item several specific KINGAL banks:\\
  KWGT,KWTK,KSEC      


\end{itemize}
 
 The MINI-making program recognizes automatically if it reads LEP1 or LEP2 datasets and adds the above banks 
 if the dataset is a LEP2 one.

\par
\section{\label{sec-wtm}Writing a Mini-DST}
\par
To write a private Mini-DST file from a DST (or POT), two options are possible. You may:


{\bf - either use the \$ALEPH/phy/miniprod executable and the MINP data card:}

If you do so you will obtain MINIs with the official ALEPH banks, especially those for LEP2 datasets.
If you want the correct pad dE/dx for data, you must read input POTs, not DSTs. 

Beware ! Using miniprod, you can't select your output events ({\bf all} input events are written on the 
output file) and you can't modify the cuts for QSELEP,QIPBTAG, etc.. However, you may add standard ALEPH banks which are
not foreseen in the official MINIs, using the MINA data card (see below).

{\bf - or write your own customised MINI:} 

In that case you can select your events and  put (almost) all what you want on your MINI. The MINI-making 
program is an ALPHA job that you can submit with {\bf alpharun}; you have to proceed as follows: 


\begin{enumerate}

\item Read a POT (for Monte Carlo) or a DST (preferable and much faster for real data if you don't need the special pad dE/dx bank 
PTPX), or even a MINI, declared with a standard FILI card.
\item Declare an output file which will contain your Mini-DST with
a FILO card.

\item ALPHA must be informed that you wish to create a Mini-DST with the data card:

 MINI 

\item Invoke integer compression with COMP `INTE' to save space (about a factor of
two) for the output file. 
 

\item 
If you want to write on your MINI some additional banks which are not ALEPH standard, you must define their format.

Banks which are not ALEPH standard are banks which are not described in the documentation file:

 al\$general:[doc.bankdoc]bankal.fmt on ALWS, 

\$ALEPH/dbase/bankal.fmt on UNIX.

To define the bank formats: in QUINIT, call the BOS routine BKFMT, one call per bank, see the BOS documentation.

If you don't do so, these banks will be {\bf irrecoverably unreadable} when you will try to read your private MINI
 afterwards.

\item
 you have to {\bf give the list of all your non-standard banks, if any,} to the MINI-making program through a data card:

MINA 'ban1ban2ban3...' where ``bani' are the names (4 letters, as usual) of your private banks.

   MINA may be used also to add to your MINI standard ALEPH banks which are not put in the standard MINIs. 

\item In your QUEVNT subroutine, after your analysis and selections, call QWRITE whenever you wish to output an event
according to your selection criteria.

\item If you wish to create the ENFLW  information on the MINI file,
 you must put an EFLJ data card.

\item You may define your own QSELEP cuts with the appropriate set of LSxx data cards (see \ref{sec-QSELCU} on p.
~\pageref{sec-QSELCU}).

\end{enumerate}
When you have written a small sample of events, it is worth {\bf checking}
that these are readable
before creating many events.
 
