\chapter{\label{sec-U}User routines}
\par
In this chapter, ALPHA routines which are intended to be modified
by the
user are described.
\par Normally, only three routines have to be provided by
the user:
initialization (QUINIT), event analysis
(QUEVNT), and program termination (QUTERM).
Models for these three subroutines are available (see Appendix
\ref{sec-ZB}).
Other subroutines which
may be modified by the user are also described in this chapter.
\par
User routines can be provided either as a plain Fortran file or as
a CVS source file; the {\bf alpharun} command file described in
Appendix~\ref{sec-ZB} supports both options. 

\par
For all user routines, default versions exist in the ALPHA library
which
are loaded automatically if no user code is given.
 
\section{\label{sec-UG}General Comments}
\par
\subsection{\label{sec-UC}Name conventions}
\par
All Fortran symbols defined in the ALPHA package start with
Q, K, C, or X:
\begin{indentlist}{ 1.00cm}{ 1.25cm}
\indentitem{Q }subroutines; real functions, variables, or arrays
\indentitem{K }integer functions, variables, or arrays
\indentitem{X }logical functions, variables, or arrays
\indentitem{C }character functions, variables,
or arrays (always in combination with Q as 2nd character).
\end{indentlist}
To avoid conflicts with the hundreds of variables defined in the ALPHA
package,
it would be safest if your own Fortran names for subroutines, variables,
etc. did {\it NOT} start with Q, K, X, or CQ.
 
\subsection{\label{sec-UA}Including ALPHA features in Fortran code}
\par
In addition to subroutines, the ALPHA package contains include files
which have to be included at the beginning of
user subroutines or functions.
There are several such includes, three of them being the following:
\begin{indentlist}{ 2.50cm}{ 2.75cm}
\indentitem{QCDE}
COMMONs, DIMENSIONs, EQUIVALENCEs, PARAMETERs, DATAs,
type declarations (all ALPHA symbols starting with C or X are
individually declared as CHARACTER or LOGICAL, respectively).
\indentitem{QMACRO}
 
statement function definitions (from the user's point of
view, statement functions look exactly like ``normal'' Fortran
functions, but their execution is faster).

\indentitem{QDECL}

Contains all integer/real declarations of all ALPHA internal variables,
to be put in the code by people intending to use IMPLICIT NONE.

\end{indentlist}
The BOS array RW(...) and IW(...), as well as
the BMACRO statement functions (RTABL, etc.), are included in QCDE
and
QMACRO.
 
These sets of statements can be included in user subroutines
by machine$-$dependent
Fortran statements or by CVS statements, as shown below.
\par
\noindent {\bf Fortran includes for DEC  machines:}
\begin{verbatim}
      INCLUDE 'PHYINC:QCDE.INC'
      INCLUDE 'PHYINC:QMACRO.INC'
\end{verbatim}
{\bf Fortran includes for UNIX Machines:}
\begin{verbatim}
      INCLUDE '$ALEPH/phy/qcde.inc'
      INCLUDE '$ALEPH/phy/qmacro.inc'
\end{verbatim}
{\bf CVS statements for CVS source files, on all machines:}
\begin{verbatim}
#include "qdecl.h"
#include "qcde.h"
#include "qmacro.h"
\end{verbatim}
\begin{description}\item[\bf{Important!
}]The following sequence of statements must be observed:\end{description}
\begin{enumerate}
\item SUBROUTINE or FUNCTION statement
\item QDECL, if you need it (e.g.if you want to use IMPLICIT NONE)
\item QCDE 
\item your own variable declarations, COMMONs, DIMENSIONs, etc.
\item your own DATA statements
\item QMACRO
\item your own statement function definitions (if any)
\item your executable Fortran statements
\end{enumerate}
If you don't do so, you may end up with specially unintelligible Fortran compiler 
diagnostics...

\par
\subsection{\label{sec-UHAC}``HAC'' parameters}
\par
The HAC (Handy ACcess)
parameters denote the offset of attributes within each officially maintained ALEPH BOS bank.
For banks accessible by mnemonic symbols in ALPHA (see Ch.
\ref{sec-M}), this offset
is taken into account automatically, and the corresponding HAC
parameters are available in QCDE (note that names of HAC parameters
and mnemonic symbols are closely related).
 
A separate include file QHAC is provided for the HAC
parameters of all banks appearing on POT/DST/MINI
event files which are NOT
available in QCDE. QHAC contains only HAC parameters of banks
needed by ALPHA and which are official ALEPH banks maintained 
in the standard documentation system. 
It 
can be included in the same way as
QCDE and may be used in conjunction with it. 
 
\par
\subsection{\label{sec-UIMP}Implicit None}
\par
The include file  QDECL must be used by         
 people wishing to use IMPLICIT NONE in their program.
\par
 
\par
\subsection{\label{sec-UBNK}Booking of private BOS banks in ALPHA}
\par
To be faster, ALPHA stores internally a lot of BOS bank indices in the ALPHA
commons. Therefore one should absolutely avoid any BOS garbage collection
during the execution of ALPHA since the program would crash immediately.
\par
 If you need to book your own BOS banks:
 use the BOS routines NBANK, MBANK, WBANK. One should
absolutely avoid the use of the ALEPHLIB routine AUBOS since it performs a
BOS garbage collection if the space is too small in the BOS array. Of course,
the use of the BOS routines BGARB or WGARB in ALPHA is absolutely forbidden.
 
 
\par
 
\section{\label{sec-UI}User Initialization}
\par
\fbox{SUBROUTINE QUINIT}
\par This routine should be used to book histograms and to perform
other user initializations.
All standard initialization work is performed automatically in the ALPHA
subroutine QMINIT before QUINIT is called. The standard ALPHA initialization includes:
\begin{itemize}
\item Initialization of BOS (default : 1500,000 words working space)
\item Initialization of HBOOK (default : 100,000 words working space)
\item Reading special files containing constants and calibrations
\item Reading the user's data cards
\item Opening the ALEPH data base
\item Initialization of ALPHA.
\end{itemize}
These space allocations are large enough for most applications; they
can be increased by modifying the routines described in
sections \ref{sec-QUIH} and \ref{sec-QUIB}.
 
\section{\label{sec-UE}Event analysis routine}
\par
\fbox{SUBROUTINE QUEVNT(QT,KT,QV,KV)}
\par {\it QUEVNT is called once for each event.}

The current event is
read in, unpacked, and ready to be analyzed when QUEVNT is called.
\begin{description}\item[\bf{Subroutine arguments
}]{\it QT,KT,QV,KV}
are used for special applications;
see \ref{sec-MBTV}. The subroutine arguments must
be given even if they are not used.
\end{description}
\begin{description}\item[\bf{IMPORTANT:
}]Do NOT perform a BOS garbage collection (CALL BGARB(IW)) in QUEVNT or in
any routine called by QUEVNT: this would cause ALPHA to crash irrecoverably.
If you must use private BOS banks, you have to book them using the BOS routines NBANK,MBANK or WBANK.
DO NOT use the AUBOS routine from the ALEPHLIB, which performs an automatic BOS garbage collection
when no more space is available.
\end{description}
\section{\label{sec-UT}User termination routine}
\par
\fbox{SUBROUTINE QUTERM}
\par This subroutine can be used for anything which needs to
be done at the end of a job (e.g., histogram manipulations).
Histogram output is done automatically in the ALPHA routine QMTERM.
 
QUTERM must never be called directly. For program termination, use
the
statement
(see \ref{sec-QMT}):
\begin{verbatim}
              CALL QMTERM ('any message')
\end{verbatim}
QMTERM, in turn, calls QUTERM.
QMTERM is called automatically after all input files have
been processed.
\section{\label{sec-OUS}Other User Subroutines}
\par
The routines in this section normally do not have to be modified.
As mentioned above, default versions of all user routines are loaded
if
no new versions are provided.
\par
\subsection{\label{sec-QUN}New Run}
\par
\fbox{SUBROUTINE QUNEWR (IROLD,IRNEW)}
\par This routine is
called from QMNEWR once a new run is encountered on the event input
file, i.e.,
\begin{itemize}
\item either a run record is read on the input file
\item or the run number in an event record has changed
\item or both conditions are fulfilled.
\end{itemize}
QUNEWR
may be used to initialize run$-$dependent data or to print run statistics.
\begin{indentlist}{ 2.50cm}{ 2.75cm}
\indentitem{Input arguments}
\indentitem{IROLD}
old run number:
= 0 if called for the first time.
\indentitem{IRNEW}
new run number:
= 0 if called from QMTERM during the program termination.
\indentitem{Default}no action: RETURN.
 
\end{indentlist}
\par
\subsection{\label{sec-QUSREC}Unkown Record Type}
\par
\fbox{SUBROUTINE QUSREC}
\par This routine is
called whenever a record is read that is neither a run
nor an event record (e.g. a slow control record); the routine can be used
to analyze these special records.
\begin{indentlist}{ 3.00cm}{ 3.25cm}
\indentitem{Default:}
no action: RETURN.
\end{indentlist}
\subsection{\label{sec-QUIH}Initialize the histogram package}
\par
\fbox{SUBROUTINE QUIHIS}
\par NOT intended for histogram
booking (use QUINIT).
\begin{itemize}
\item Called automatically from QMINIT.
\end{itemize}
\begin{indentlist}{ 3.00cm}{ 3.25cm}
\indentitem{Default:}
 
Initialize HBOOK : CALL HLIMIT (100000).
\end{indentlist}
{\bf Warnings:} If the above size of the HBOOK array is too small, which can happen with
a very large number of scatter plots and/or Ntuples, one can increase it
by modifying the parameter LQPAW in the QUIHIS routine.
Users who want to write on output file  {\em very}
large NTUPLES ( more than 16 Mbytes ) will have to give appropriate NREC and/or  RECL parameters
in their HIST data card ( see \ref{sec-HISTW} on p. ~\pageref{sec-HISTW} ).
 
\subsection{\label{sec-QUTH}Terminate the histogram package}
\par
\fbox{SUBROUTINE QUTHIS}
\par
\begin{itemize}
\item Called automatically from QMTERM.
\end{itemize}
\begin{indentlist}{ 3.00cm}{ 3.25cm}
\indentitem{Default:}Terminate HBOOK: CALL HISTDO
If the HIST data card is given, write output on histogram file.
\end{indentlist}
\subsection{\label{sec-QUIB}Initialize BOS}
\par
\fbox{SUBROUTINE QUIBOS}
\par The length of the
BOS working space COMMON /BCS/ is explicitly declared in this
subroutine.
\begin{itemize}
\item Called automatically from QMINIT.
\end{itemize}
\begin{indentlist}{ 3.00cm}{ 3.25cm}
\indentitem{Default:}initialize BOS with 1500,000 words working space.
\end{indentlist}

If you need more space, you have to edit the source code of this routine (see below where to find it),
 to increase the value of the PARAMETER LQBOS to the desired value, and to put
  the modified QUIBOS routine in your CVS files or into your FORTRAN user file.

The CVS source code of QUIBOS can be found in the following files:

on the UNIX machines in  \$ALROOT/alphavsn/pack/quibos.F

                        (vsn is the ALPHA version number, e.g.  124)

on the ALWS cluster  in  phy:quibos.F

