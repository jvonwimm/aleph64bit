 \item{\bf [c]}Constant = Fortran parameter\\
 \item{\bf [f]}Fortran function\\
 \item{\bf [s]}Fortran subroutine\\
 \item{\bf [sf]}Fortran statement function\\
 \item{\bf [v]}Fortran variable/array stored in a COMMON block
 
 \item{\bf ADBSCONS } ALEPH database, may be defined on FDBA data card, \ref{sec-DCFDBA} on p.~\pageref{sec-DCFDBA}\\
 \item{\bf add    }vectors: \ref{sec-QVA} on
 p.~\pageref{sec-QVA} and \ref{sec-QJA} on p.~\pageref{sec-QJA}\\
 \item{\bf ALCORnnn }
  Obsolete: correction file to ALPHA version nnn.
 All corrections are included into the Alpha library, see
 Appendix \ref{sec-ZB} on p.~\pageref{sec-ZB}\\
 \item{\bf ALENFLW  }Obsolete: library to run ENFLW, QMUIDO, GAMPEX:
 now  in the Alpha library, see
 Appendix \ref{sec-ZB} on p.~\pageref{sec-ZB}\\
 \item{\bf ALEPH file types }\ref{sec-DCFT} on p.~\pageref{sec-DCFT}\\
 \item{\bf ALGUIDE} App \ref{sec-ZB} on p.~\pageref{sec-ZB}  Where to find the \LaTeX source
  and PostScript file of the present documentation\\
 \item{\bf ALLR    }parameter on FILO data card: \ref{sec-DCFILO} on
 p.~\pageref{sec-DCFILO}
 , \item{\bf ALPHA   }initialization in QMALPH called from QMINIT:
 \ref{sec-UI} on p.~\pageref{sec-UI}\\
 \item{\bf $\mathrm{ALPHA^{++}}$} $\mathrm{C^{++}}$ extension of ALPHA: Appendix \ref{sec-cextalp} on p.~\pageref{sec-cextalp}\\
 \item{\bf ALPHARUN    }command file to run ALPHA on VMS:Ch.~\ref{sec-GS} on p.~\pageref{sec-GS}
 and App.~\ref{sec-ZB} on p.~\pageref{sec-ZB}\\
 \item{\bf alpharun    }shell file to run ALPHA on UNIX:Ch.~\ref{sec-GS} on p.~\pageref{sec-GS}
 and App.~\ref{sec-ZB} on p.~\pageref{sec-alphar}
 
 \item{\bf angle   }\\
 \subitem azimuth, polar angle: \ref{sec-MK} on p.~\pageref{sec-MK}\\
 \subitem decay angle: \ref{sec-MK} on p.~\pageref{sec-MK}\\
 \item{\bf antiparticle    }\\
 \subitem access to antiparticles: \ref{sec-ADA} on p.~\pageref{sec-ADA}\\
 \subitem definition on data cards: \ref{sec-DCPMOD} on p.~\pageref{sec-DCPMOD}\\
 \item{\bf AUBOS }[s]ALEPHLIB routine to book BOS banks, use FORBIDDEN in ALPHA:
 \ref{sec-UBNK} on p.~\pageref{sec-UBNK}
 
 \item{\bf batch job   }see ALPHARUN: \ref{sec-GS} on p.~\pageref{sec-GS}\\
 \item{\bf beam position, from chunk-by-chunk information (GET\_BP)} see XGETBP, etc.: \ref{sec-MR} on
 p.~\pageref{sec-MR},\\
 \item{\bf beam position from BOMs} \ref{sec-BOM} on p.~\pageref{sec-BOM}\\
 \item{\bf bending radius  }of a reconstructed charged track:
 \ref{sec-TVAFRFT} on p.~\pageref{sec-TVAFRFT}\\
 \item{\bf beta    }see QBETA: \ref{sec-MK} on p.~\pageref{sec-MK}\\
 \item{\bf BGARB }[s]BOS routine to cleanup BOS array, forbidden in ALPHA :
 \ref{sec-UBNK} on p.~\pageref{sec-UBNK}\\
 \item{\bf BHIS    }data card for QIPBTAG: \ref{sec-QIPBCD} on p.~\pageref{sec-QIPBCD}
 
 \item{\bf BMACRO  }standard BOS statement functions:  \ref{sec-UA} on p.~\pageref{sec-UA}\\
 \item{\bf BNEG    }data card for QIPBTAG: \ref{sec-QIPBCD} on p.~\pageref{sec-QIPBCD}
 
 \item{\bf book   }histograms:~\ref{sec-HBOOK} on p.~\pageref{sec-HBOOK}\\
 \item{\bf boost   }Lorentz: \ref{sec-QT} on p.~\pageref{sec-QT}\\
 \item{\bf BOBS    }data card for LEP BOMs: \ref{sec-DCBOBS} on p.~\pageref{sec-DCBOBS}\\
 \item{\bf BOM     }beam position from BOMs: \ref{sec-BOM} on p.~\pageref{sec-BOM}\\
 \item{\bf BOS     }initialization in QUIBOS: \ref{sec-QUIB} on p.~\pageref{sec-QUIB}\\
 \item{\bf BPER    }data card for MCarlo beam position: \ref{sec-DCHUNK} on p.~\pageref{sec-DCHUNK}\\
 \item{\bf BPWT    }data card for MCarlo beam position: \ref{sec-DCHUNK} on p.~\pageref{sec-DCHUNK}\\
 \item{\bf BSIZ    }data card for MCarlo beam position: \ref{sec-DCHUNK} on p.~\pageref{sec-DCHUNK}\\
 \item{\bf BTRK    }data card for QIPBTAG: \ref{sec-QIPBCD} on p.~\pageref{sec-QIPBCD}
 
 
 \item{\bf c   }[c] speed of light: \ref{sec-MCP} on p.~\pageref{sec-MCP}\\
 \item{\bf CALB    }data card for QIPBTAG: \ref{sec-QIPBCD} on p.~\pageref{sec-QIPBCD} \\
 \item{\bf CALPHA  }to use ALPHA from C  programs : Appendix \ref{sec-cextalp} on p.~\pageref{sec-cextalp}\\
 \item{\bf calorimeter objects }\ref{sec-AL} on p.~\pageref{sec-AL} and
 \ref{sec-AR} on p.~\pageref{sec-AR}\\
 \item{\bf CARDS   }file type: \ref{sec-DCFT} on p.~\pageref{sec-DCFT}\\
 \item{\bf charge  }\\
 \subitem of an individual particle: \ref{sec-TVABA} on p.~\pageref{sec-TVABA}\\
 \subitem on particle table:~\ref{sec-PTAC} on p.~\pageref{sec-PTAC}\\
 \item{\bf charged tracks  }\ref{sec-AL} on p.~\pageref{sec-AL} and
 \ref{sec-AR} on p.~\pageref{sec-AR}\\
 \item{\bf CHTSIM }[f] To decide to use QDEDX or QDEDXM in MCarlo analysis
 \ref{sec-OARDEDC} on p.~\pageref{sec-OARDEDC}\\
 \item{\bf CLAS    }Data card for use with event directories to select events:
 \ref{sec-DCEVD} on p.~\pageref{sec-DCEVD}\\
 \item{\bf class   }\\
 \subitem reading class word for EDIRs: \ref{sec-MED} on p.~\pageref{sec-MED}\\
 \subitem setting class word for EDIRs: \ref{sec-QWCLAS} on p.~\pageref{sec-QWCLAS}\\
 \subitem ``track'' class: \ref{sec-ADI} on p.~\pageref{sec-ADI} and \ref{sec-TVAFP} on p.~\pageref{sec-TVAFP}
 
 \item{\bf COPY    }data card: \ref{sec-DCCOPY} on p.~\pageref{sec-DCCOPY}\\
 \item{\bf copy    }\\
 \subitem track vectors into other track vectors:
 \ref{sec-QVC} on p.~\pageref{sec-QVC} and \ref{sec-QVST} on p.~\pageref{sec-QVST}\\
 \subitem track vectors into Fortran arrays:~\ref{sec-QVG} on p.~\pageref{sec-QVG}\\
 \subitem Fortran arrays into track vectors: \ref{sec-QVM} on p.~\pageref{sec-QVM}\\
 \item{\bf CQDATE  }[v] date at start of job: \ref{sec-MCC} on p.~\pageref{sec-MCC}\\
 \item{\bf CQFOUT  }[v] name of output file: \ref{sec-MCC} on p.~\pageref{sec-MCC}\\
 \item{\bf CQPART  }[f] particle name for a given integer code:
 \ref{sec-PTAC} on p.~\pageref{sec-PTAC}\\
 \item{\bf CQTPN   }[f] track's particle name: \ref{sec-TVAFP} on p.~\pageref{sec-TVAFP}\\
 \item{\bf CQTIME  }[v] time at start of job: \ref{sec-MCC} on p.~\pageref{sec-MCC}\\
 \item{\bf CQVERS  }[v] ALPHA version: \ref{sec-MCC} on p.~\pageref{sec-MCC}\\
 \item{\bf create  }new track: \ref{sec-QVN} on p.~\pageref{sec-QVN}\\
 \item{\bf cross product   }QVCROS: \ref{sec-QVX} on p.~\pageref{sec-QVX}
 
 \item{\bf DAF     }file type -- direct access files: \ref{sec-DCFT} on p.~\pageref{sec-DCFT}\\
 \item{\bf data}\\
 \subitem  base -- opened in QMINIT: \ref{sec-UI} on p.~\pageref{sec-UI}\\
 \subitem cards -- description \ref{sec-DC} on p.~\pageref{sec-DC}\\
 \subitem set name --conventions \ref{sec-DCFT} on p.~\pageref{sec-DCFT};
 examples \ref{sec-DCFILI} on p.~\pageref{sec-DCFILI}\\
 \item{\bf daughter particles }\ref{sec-AMM} on p.~\pageref{sec-AMM} and
 \ref{sec-AV} on p.~\pageref{sec-AV}\\
 \item{\bf DEBU    }data card: \ref{sec-DCDEBU} on p.~\pageref{sec-DCDEBU}\\
 \item{\bf debug }\\
 \subitem special VAX debugger features: \ref{sec-GS} on p.~\pageref{sec-GS}\\
 \subitem level -- see KDEBUG: \ref{sec-MCS} on p.~\pageref{sec-MCS}\\
 \item{\bf decay angle }\ref{sec-MK} on p.~\pageref{sec-MK}\\
 \item{\bf dE/dx       }\ref{sec-OARDEDX} on p.~\pageref{sec-OARDEDX},
 \ref{sec-TVATEXS} on p.~\pageref{sec-TVATEXS}\\
 \item{\bf DISP    }parameter on HIST or FILO  data card \\
 \item{\bf dot product }\ref{sec-MK} on p.~\pageref{sec-MK}\\
 \item{\bf DRGA    }data card: \ref{sec-DCSPCA} on p.~\pageref{sec-DCSPCA}\\
 \item{\bf drop tracks }\ref{sec-QVD} on p.~\pageref{sec-QVD}\\
 \item{\bf DST unpacking   }\ref{sec-DCUNPK} on p.~\pageref{sec-DCUNPK}\\
 \item{\bf D0 }see QDB\\
 \item{\bf DURHAM  }jet finding -- scaled invariant mass sq. algorithm:
 \ref{sec-QGJMMC} on p.~\pageref{sec-QGJMMC}\\
 \item{\bf DWIN    }data card for QFNDIP cuts: \ref{sec-DCQFND} on p.~\pageref{sec-DCQFND}
 
 \item{\bf e   }constant: \ref{sec-MCP} on p.~\pageref{sec-MCP}\\
 \item{\bf ECAL}\\
 \subitem objects \ref{sec-AL} on p.~\pageref{sec-AL} and
 \ref{sec-AR} on p.~\pageref{sec-AR}\\
 \subitem wire energy -- see QEECWI: \ref{sec-ECWI} on p.~\pageref{sec-ECWI}\\
 \item{\bf EDIR    }event directory: \ref{sec-DCEVD} on p.~\pageref{sec-DCEVD}\\
 \item{\bf EFLW    }energy flow data card: \ref{sec-EFLWM} on p.~\pageref{sec-EFLWM}\\
 \item{\bf EFOL    }see energy flow\\
 \item{\bf EFOU    }data card: to write the EFOL bank on output tape
 \ref{sec-DCSPCA} on p.~\pageref{sec-DCSPCA}\\
 \item{\bf EGPC    }see GAMPEC, obsolete\\
 \item{\bf ENDQ    }BOS data card: \ref{sec-DC} on p.~\pageref{sec-DC}\\
 \item{\bf energy  }\\
 \subitem center-of-mass LEP energy, see QELEP:~\ref{sec-CHUNKELEP} on p.~\pageref{sec-CHUNKELEP}\\
 \subitem for ALPHA tracks,
 see QE: \ref{sec-TVABA} on p.~\pageref{sec-TVABA}\\
 \subitem missing energy: \ref{sec-QJME} on p.~\pageref{sec-QJME}\\
 \item{\bf energy flow }Ch. \ref{sec-EF} on p.~\pageref{sec-EF}\\
 \item{\bf ENFLW }energy flow algorithm: \ref{sec-EFLWJ} on p.~\pageref{sec-EFLWJ}\\
 \item{\bf EPIO    }file type: machine-independent input / output:
 \ref{sec-DCFT} on p.~\pageref{sec-DCFT}\\
 \item{\bf EVEH    }bank: \ref{sec-MHE} on p.~\pageref{sec-MHE}\\
 \item{\bf event }\\
 \subitem directories: \ref{sec-DCEVD} on p.~\pageref{sec-DCEVD}\\
 \subitem input -- see FILI data card: \ref{sec-DCFILI} on p.~\pageref{sec-DCFILI}\\
 \subitem output-- see FILO data card \ref{sec-DCFILO} on p.~\pageref{sec-DCFILO}
 and routine QWRITE: \ref{sec-QWR} on p.~\pageref{sec-QWR}\\
 \subitem processing -- see QUEVNT: \ref{sec-UE} on p.~\pageref{sec-UE}
 
 \item{\bf FDBA }Data card to set ADBSCONS database :  \ref{sec-DCFDBA} on p.~\pageref{sec-DCFDBA}\\
 \item{\bf FIEL }Data card to set magnetic field: \ref{sec-DCFIEL} on p.~\pageref{sec-DCFIEL}\\
 \item{\bf file types  }= ALEPH file types: \ref{sec-DCFT} on p.~\pageref{sec-DCFT}\\
 \item{\bf FILI    }data card -- input data set(s): \ref{sec-DCFILI} on p.~\pageref{sec-DCFILI}\\
 \item{\bf FILO    }data card -- output data set: \ref{sec-DCFILO} on p.~\pageref{sec-DCFILO}\\
 \item{\bf flags   }user: \ref{sec-TVAFP} on p.~\pageref{sec-TVAFP},
 \ref{sec-USFL} on p.~\pageref{sec-USFL}\\
 \item{\bf Fox-Wolfram }moments: \ref{sec-QJFW} on p.~\pageref{sec-QJFW}\\
 \item{\bf frame   }access to Lorentz frames: \ref{sec-ADI} on p.~\pageref{sec-ADI}\\
 \item{\bf FRF0  }data card -- ignore vertex det. in track
 fit:~\ref{sec-FRF0} on p.~\pageref{sec-FRF0}\\
 \item{\bf FR10 or FR12  }data card -- use unsmeared hits for track fits in Mcarlo POTs
 :~\ref{sec-FR12} on p.~\pageref{sec-FR12}
 
 \item{\bf gamma   }see QGAMMA: \ref{sec-MK} on p.~\pageref{sec-MK}\\
 \item{\bf gamma conversions }see QPAIRF:~\ref{sec-OARPAIR} on
 p.~\pageref{sec-OARPAIR}\\
 \item{\bf GAMPEC }photons: \ref{sec-altrack}
 on p.~\pageref{sec-altrack} and
 \ref{sec-TVAEGPC} on p.~\pageref{sec-TVAEGPC}\\
 \item{\bf GAMPEX }photons: \ref{sec-altrack}
 on p.~\pageref{sec-altrack} and
 \ref{sec-TVAPGPC} on p.~\pageref{sec-TVAPGPC}\\
 \item{\bf GENEVA  }jet finding -- so-called GENEVA algorithm:           
 \ref{sec-QGJMMC} on p.~\pageref{sec-QGJMMC}\\
 \item{\bf GET\_BP(I) }[s] Find the event--chunk beam position
 \ref{sec-MR} on p.~\pageref{sec-MR}\\
  \item{\bf GETYIJ  }[s]jet finding --$>$> yij for a fixed number of jets
 \ref{sec-QGJMMC} on p.~\pageref{sec-QGJMMC}\\

 
 \item{\bf h   }Planck constant \ref{sec-MCP} on p.~\pageref{sec-MCP}\\
 \item{\bf HAC parameters  }bank offset: \ref{sec-UHAC} on p.~\pageref{sec-UHAC}\\
 \item{\bf hbar    }constant: \ref{sec-MCP} on p.~\pageref{sec-MCP}\\
 \item{\bf HBOOK   }\ref{sec-HBOOK} on p.~\pageref{sec-HBOOK}\\
 \subitem initialization -- QUIHIS: \ref{sec-QUIH} on p.~\pageref{sec-QUIH}\\
 \subitem termination --QUTHIS: \ref{sec-QUTH} on p.~\pageref{sec-QUTH}\\
 \item{\bf HCAL objects    }\ref{sec-AL} on p.~\pageref{sec-AL} and
 \ref{sec-AR} on p.~\pageref{sec-AR}\\
 \item{\bf hemispheres }see QJHEMI: \ref{sec-QJHE} on p.~\pageref{sec-QJHE}\\
 \item{\bf High Voltage } \ref{sec-MHR} on p.~\pageref{sec-MHR}\\
 \item{\bf HIS     }histogram file type \ref{sec-HISTW} on p.~\pageref{sec-HISTW}\\
 \item{\bf HIST    }histogram file data card: \ref{sec-HISTW} on p.~\pageref{sec-HISTW}\\
 \item{\bf histograms }Ch. \ref{sec-HIST} on p.~\pageref{sec-HIST}\\
 \item{\bf histogram output  }see \ref{sec-HISTW} on p.~\pageref{sec-HISTW} and
 \ref{sec-HISTP} on p.~\pageref{sec-HISTP}\\
 \item{\bf HJET    }data card for QIPBTAG:  \ref{sec-QIPBCD} on p.~\pageref{sec-QIPBCD}\\
 \item{\bf HTIT    }data card: general histogram title \ref{sec-DCHT} on p.~\pageref{sec-DCHT}
 
 \item{\bf Implicit None }\ref{sec-UIMP} on p.~\pageref{sec-UIMP}\\
 \item{\bf INCLUDE     }Fortran statement: \ref{sec-UA} on p.~\pageref{sec-UA}\\
 \item{\bf initialization  }see QMINIT and QUINIT: \ref{sec-UI} on p.~\pageref{sec-UI}\\
 \item{\bf invariant mass  }\ref{sec-MK} on p.~\pageref{sec-MK}\\
 \item{\bf INVMAS  }jet finding -- invariant mass  algorithm:
 \ref{sec-QGJMMC} on p.~\pageref{sec-QGJMMC}\\
 \item{\bf IRUN    }data card: ignore runs \ref{sec-DCRS} on p.~\pageref{sec-DCRS}
 
 
 \item{\bf JADE  }jet finding -- scaled invariant mass sq. algorithm:
 \ref{sec-QJMMCL} on p.~\pageref{sec-QJMMCL}\\
 \item{\bf JMES    }data card for QIPBTAG:  \ref{sec-QIPBCD} on p.~\pageref{sec-QIPBCD}\\
 \item{\bf JULMATCH }[s] matching between reconstructed tracks and MC truth:
 \ref{sec-OAJULMAT} on p.~\pageref{sec-OAJULMAT}\\
 \item{\bf jets }\ref{sec-QJMM} on p.~\pageref{sec-QJMM}
 
 \item{\bf KBESTM  }[f] best match to a charged track: \ref{sec-AX} on p.~\pageref{sec-AX}\\
 \item{\bf KBFLAG  }[sf] track flag bits\\
 \item{\bf KBMASK  }[sf] track mask bits
 
 \item{\bf KCANTI  }[sf] particle - antiparticle: \ref{sec-PTAC} on p.~\pageref{sec-PTAC}\\
 \item{\bf KCDIR   }[sf] direct access to particles: \ref{sec-ADE} on p.~\pageref{sec-ADE}\\
 \item{\bf KCDIRA  }[sf] direct access to (anti)particles: \ref{sec-ADA} on p.~\pageref{sec-ADA}\\
 \item{\bf KCHGD   }[sf] list of associated charged tracks: \ref{sec-AR} on p.~\pageref{sec-AR}\\
 \item{\bf KCLASS  }[sf] class KRECO,KMONTE,Lorentz fr.: \ref{sec-TVAFP} on p.~\pageref{sec-TVAFP}\\
 \item{\bf KCLASW  }[v] event directory class. word: \ref{sec-MED} on p.~\pageref{sec-MED}\\
 \item{\bf KCH     }[sf] track's charge: \ref{sec-TVABA} on p.~\pageref{sec-TVABA}\\
 \item{\bf KCHT    }[f] original copy of a charged track: \ref{sec-KCHT} on p.~\pageref{sec-KCHT}\\
 \item{\bf KDAU    }[sf] access to daughter particles: \ref{sec-AMM} on p.~\pageref{sec-AMM}\\
 \item{\bf KDEBUG  }[v] debug level: \ref{sec-MCS} on p.~\pageref{sec-MCS}\\
 \item{\bf KECAL   }[sf] list of associated ECAL objects: \ref{sec-AR} on p.~\pageref{sec-AR}\\
 \item{\bf KEFOxx  }[sf] access to ENFLW energy flow informations
                    \ref{sec-EFLWS} on p.~\pageref{sec-EFLWS}\\
 \item{\bf KEIDxx  }[sf] bank EIDT: \ref{sec-TVAEIDT} on p.~\pageref{sec-TVAEIDT}\\
 \item{\bf KENDV   }[sf] track's end vertex: \ref{sec-AV} on p.~\pageref{sec-AV}\\
 \item{\bf KEVExx  }[v] event header bank EVEH: \ref{sec-MHE} on p.~\pageref{sec-MHE}\\
 \item{\bf KEVH    }bank: \ref{sec-MHK} on p.~\pageref{sec-MHK}\\
 \item{\bf KEVT    }[v] current event number: \ref{sec-MHE} on p.~\pageref{sec-MHE}\\
 \item{\bf KEXP    }[v] experiment number: \ref{sec-MHE} on p.~\pageref{sec-MHE}
 
 \item{\bf KFAST   }[v] first cal object associated to a charged
 track:
 \ref{sec-AL} on p.~\pageref{sec-AL}\\
 \item{\bf KFCHT   }[v] first charged track: \ref{sec-AL} on p.~\pageref{sec-AL}\\
 \item{\bf KFCOT   }[v] first cal object: \ref{sec-AL} on p.~\pageref{sec-AL}\\
 \item{\bf KFDCT   }[v] first decay track: \ref{sec-AL} on p.~\pageref{sec-AL}\\
 \item{\bf KFIST   }[v] first isolated cal object: \ref{sec-AL} on p.~\pageref{sec-AL}\\
 \item{\bf KFJET   }[v] first reconstructed jet: \ref{sec-AL} on p.~\pageref{sec-AL}\\
 \item{\bf KFKIV   }[v] first Kink vertex: \ref{sec-alvert} on p.~\pageref{sec-alvert}\\
 \item{\bf KFLJET  }[v] first jet built by subroutine QSELEP : \ref{sec-QSELJT} on p.~\pageref{sec-QSELJT}  \\
 \item{\bf KFLVT   }[v] first reconstructed Long V0 track: \ref{sec-altrack} on p.~\pageref{sec-altrack}\\
 \item{\bf KFLV0   }[v] first reconstructed Long V0 vertex: \ref{sec-alvert} on p.~\pageref{sec-alvert}
 
 \item{\bf KFMCT   }[v] first MC particle: \ref{sec-AL} on p.~\pageref{sec-AL}\\
 \item{\bf KFMCV   }[v] first MC vertex  : \ref{sec-alvert} on p.~\pageref{sec-alvert}\\
 \item{\bf KFNIV   }[v] first Nuclear Interaction vertex: \ref{sec-alvert} on p.~\pageref{sec-alvert}
 
 \item{\bf KFOLLO  }[sf] following track: \ref{sec-ADE} on p.~\pageref{sec-ADE}\\
 \item{\bf KRDFL   }[sf] read user flag: \ref{sec-TVAFP} on p.~\pageref{sec-TVAFP}\\
 \item{\bf KFRET   }[v] first reconstructed track: \ref{sec-AL} on p.~\pageref{sec-AL}\\
 \item{\bf KFREV   }[v] first reconstructed vertex  : \ref{sec-alvert} on p.~\pageref{sec-alvert}\\
 \item{\bf KFRFxx  }[sf] bank FRFT = track fit: \ref{sec-TVAFRFT} on p.~\pageref{sec-TVAFRFT}\\
 \item{\bf KFRIxx  }[sf] bank FRID: \ref{sec-TVAFRID} on p.~\pageref{sec-TVAFRID}\\
 \item{\bf KFRTxx  }[sf] bank FRTL: \ref{sec-TVAFRTL} on p.~\pageref{sec-TVAFRTL}\\
 \item{\bf KFV0T   }[v] first particle pointing to V0: \ref{sec-AL} on p.~\pageref{sec-AL}
 
 \item{\bf KHCAL   }[sf] list of associated HCAL objects: \ref{sec-AR} on p.~\pageref{sec-AR}\\
 \item{\bf KHMAxx  }[sf] bank HMAD = HCAL--muon association:
 \ref{sec-TVAHMAD} on p.~\pageref{sec-TVAHMAD}\\
 \item{\bf Kinematic fitting }\ref{sec-QFIT} on p.~\pageref{sec-QFIT}\\
 \item{\bf KKEVxx  }[v] bank KEVH \ref{sec-MHK} on p.~\pageref{sec-MHK}\\
 \item{\bf KLAST   }[v] last cal object associated to a charged track:
 \ref{sec-AL} on p.~\pageref{sec-AL}\\
 \item{\bf KLCHT   }[v] last charged track: \ref{sec-AL} on p.~\pageref{sec-AL}\\
 \item{\bf KLCOT   }[v] last cal object: \ref{sec-AL} on p.~\pageref{sec-AL}\\
 \item{\bf KLDCT   }[v] last decay track: \ref{sec-AL} on p.~\pageref{sec-AL}\\
 \item{\bf KLEPxx  }[sf] Output variables from subroutine QSELEP : \ref{sec-QSELTL} on p.~\pageref{sec-QSELTL}\\
 \item{\bf KLIST   }[v] last isolated cal object: \ref{sec-AL} on p.~\pageref{sec-AL}\\
 \item{\bf KLJET   }[v] last reconstructed jet: \ref{sec-AL} on p.~\pageref{sec-AL}\\
 \item{\bf KLJTNO  }[sf] number of objects inside jet built by QSELEP : \ref{sec-QSELJT} on p.~\pageref{sec-QSELJT}
 
 \item{\bf KLKIV   }[v] last Kink vertex: \ref{sec-alvert} on p.~\pageref{sec-alvert}\\
 \item{\bf KLLJET  }[v] last jet built by subroutine QSELEP : \ref{sec-QSELJT} on p.~\pageref{sec-QSELJT}\\
 \item{\bf KLLVT   }[v] last reconstructed Long V0 track: \ref{sec-altrack} on p.~\pageref{sec-altrack}\\
 \item{\bf KLLV0   }[v] last reconstructed Long V0 vertex: \ref{sec-alvert} on p.~\pageref{sec-alvert}\\
 \item{\bf KLMCT   }[v] last MC particle: \ref{sec-AL} on p.~\pageref{sec-AL}\\
 \item{\bf KLMCV   }[v] last MC vertex  : \ref{sec-alvert} on p.~\pageref{sec-alvert}\\
 \item{\bf KLNIV   }[v] last Nuclear Interaction vertex: \ref{sec-alvert} on p.~\pageref{sec-alvert}
 
 \item{\bf KLRET   }[v] last reconstructed track: \ref{sec-AL} on p.~\pageref{sec-AL}\\
 \item{\bf KLREV   }[v] last reconstructed vertex  : \ref{sec-alvert} on p.~\pageref{sec-alvert}\\
 \item{\bf KLUNDS  }[sf] Lund status code: \ref{sec-TVAFP} on p.~\pageref{sec-TVAFP}\\
 \item{\bf KLV0T   }[v] last particle pointing to V0: \ref{sec-AL} on p.~\pageref{sec-AL}\\
 \item{\bf KMCAxx  }[sf] bank MCAD = muon chambers: \ref{sec-TVAMCAD} on p.~\pageref{sec-TVAMCAD}\\
 \item{\bf KMOTH   }[sf] access to mother particle: \ref{sec-AMD} on p.~\pageref{sec-AMD}\\
 \item{\bf KMTCH   }[sf] match MC -- reconstructed charged tracks: \ref{sec-AX} on p.~\pageref{sec-AX}\\
 \item{\bf KMUIIF  }[sf] Muon identification flag for charged tracks:
                    \ref{sec-TVAMUID} on p.~\pageref{sec-TVAMUID}\\
 \item{\bf KNAST   }[v] number of cal objects assoc. to a charged
 track:
 \ref{sec-AL} on p.~\pageref{sec-AL}\\
 \item{\bf KNCHGD  }[sf] number of associated charged tracks: \ref{sec-AR} on p.~\pageref{sec-AR}\\
 \item{\bf KNCHT   }[v] number of charged tracks: \ref{sec-AL} on p.~\pageref{sec-AL}\\
 \item{\bf KNCOT   }[v] number of cal objects: \ref{sec-AL} on p.~\pageref{sec-AL}\\
 \item{\bf KNDAU   }[sf] number of daughters: \ref{sec-AMM} on p.~\pageref{sec-AMM}\\
 \item{\bf KNDCT   }[v] number of decay tracks: \ref{sec-AL} on p.~\pageref{sec-AL}\\
 \item{\bf KNECAL  }[sf] number of associated ECAL objects: \ref{sec-AR} on p.~\pageref{sec-AR}\\
 \item{\bf KNEFIL  }[v] number of events on current input file:
 \ref{sec-MCN} on p.~\pageref{sec-MCN}\\
 \item{\bf KNEOUT  }[v] number of events on output file: \ref{sec-MCN} on p.~\pageref{sec-MCN}\\
 \item{\bf KNEVT   }[v] number of events read in up to now:
 \ref{sec-MCN} on p.~\pageref{sec-MCN}\\
 \item{\bf KNHCAL  }[sf] number of associated HCAL objects: \ref{sec-AR} on p.~\pageref{sec-AR}\\
 \item{\bf KNIST   }[v] number of isolated cal objects: \ref{sec-AL} on p.~\pageref{sec-AL}\\
 \item{\bf KNJET   }[v] number of reconstructed jets: \ref{sec-AL} on p.~\pageref{sec-AL}\\
 \item{\bf KNLJET  }[v] number of jets built by subroutine QSELEP : \ref{sec-QSELJT} on p.~\pageref{sec-QSELJT}\\
 \item{\bf KNMCT   }[v] number of MC particles: \ref{sec-AL} on p.~\pageref{sec-AL}\\
 \item{\bf KNMCV   }[v] number of MC vertices : \ref{sec-alvert} on p.~\pageref{sec-alvert}\\
 \item{\bf KNMOTH  }[sf] number of mother particles: \ref{sec-AMD} on p.~\pageref{sec-AMD}\\
 \item{\bf KNMTCH  }[sf] number of matching particles: \ref{sec-AX} on p.~\pageref{sec-AX}\\
 \item{\bf KNOVT   }[v] number of overlap objects: \ref{sec-AL} on p.~\pageref{sec-AL}\\
 \item{\bf KNREIN  }[v] number of records read from current input
 file:
 \ref{sec-MCN} on p.~\pageref{sec-MCN}\\
 \item{\bf KNRET   }[v] number of reconstructed tracks: \ref{sec-AL} on p.~\pageref{sec-AL}\\
 \item{\bf KNREV   }[v] number of reconstructed vertices  : \ref{sec-alvert} on p.~\pageref{sec-alvert}\\
 \item{\bf KNTEX   }[sf] number of TPC sectors for dE/dx:
 \ref{sec-TVATEXS} on p.~\pageref{sec-TVATEXS}\\
 \item{\bf KNTRU   }[f] number of matchings between ENFLW objects and true MC:
  \ref{sec-EFLWMA} on p.~\pageref{sec-EFLWMA}
 
 \item{\bf KNV0T   }[v] number of particle pointing to V0s: \ref{sec-AL} on p.~\pageref{sec-AL}
 
 \item{\bf KORIV   }[sf] vertex at origin of track: \ref{sec-AV} on p.~\pageref{sec-AV}
 
 \item{\bf KPART   }[f] integer code from particle name:
 \ref{sec-PTAC} on p.~\pageref{sec-PTAC} and \ref{sec-ADT} on p.~\pageref{sec-ADT}\\
 \item{\bf KPDIR   }[f] direct access to particles: \ref{sec-ADE} on p.~\pageref{sec-ADE}\\
 \item{\bf KPDIRA  }[f] direct access to (anti)particles: \ref{sec-ADA} on p.~\pageref{sec-ADA}\\
 \item{\bf KPECxx  }[sf] bank PECO: \ref{sec-TVAPECO} on p.~\pageref{sec-TVAPECO}\\
 \item{\bf KPEPxx  }[sf] bank PEPT: \ref{sec-TVAPEPT} on p.~\pageref{sec-TVAPEPT}\\
 \item{\bf KPGAxx  }[sf] bank PGAC: \ref{sec-TVAPGAC} on p.~\pageref{sec-TVAPGAC}\\
 \item{\bf KPHCxx  }[sf] bank PHCO: \ref{sec-TVAPHCO} on p.~\pageref{sec-TVAPHCO}
 
 \item{\bf KRUN    }[v] run number: \ref{sec-MHE} on p.~\pageref{sec-MHE}
 
 \item{\bf KSAME   }[sf] access to same objects: \ref{sec-AS} on p.~\pageref{sec-AS}\\
 \item{\bf KSMTCH  }[sf] number of shared hits in match: \ref{sec-AX} on p.~\pageref{sec-AX}\\
 \item{\bf KSTABC  }[sf] stability code: \ref{sec-TVASC} on p.~\pageref{sec-TVASC}\\
 \item{\bf KSTATU  }[v] job status (init / event proc. / term):
 \ref{sec-MCS} on p.~\pageref{sec-MCS}
 
 \item{\bf KTEXxx  }[sf] dE/dx bank TEXS: \ref{sec-TVATEXS} on p.~\pageref{sec-TVATEXS}\\
 \item{\bf KTLOR   }[f] Lorentz transformation: \ref{sec-QTL} on p.~\pageref{sec-QTL}\\
 \item{\bf KTLOR1  }[f] Lorentz transformation: \ref{sec-QT1} on p.~\pageref{sec-QT1}\\
 \item{\bf KTN     }[sf] Julia/Galeph track number: \ref{sec-TVAFP} on p.~\pageref{sec-TVAFP}\\
 \item{\bf KTPCOD  }[sf] track's particle code: \ref{sec-TVAFP} on p.~\pageref{sec-TVAFP}
 
 \item{\bf KUCARD  }[v] log. unit for the card file: \ref{sec-MCU} on p.~\pageref{sec-MCU}\\
 \item{\bf KUCRD2  }[v] 2nd log. unit for card files: \ref{sec-MCU} on p.~\pageref{sec-MCU}\\
 \item{\bf KUCONS  }[v] log.unit for the data base: \ref{sec-MCU} on p.~\pageref{sec-MCU}\\
 \item{\bf KUINPU  }[v] log. unit for event input: \ref{sec-MCU} on p.~\pageref{sec-MCU}\\
 \item{\bf KUOUTP  }[v] log. unit for event output: \ref{sec-MCU} on p.~\pageref{sec-MCU}\\
 \item{\bf KUPRNT  }[v] log. unit for the line printer output:
 \ref{sec-MCU} on p.~\pageref{sec-MCU} and \ref{sec-MCU} on p.~\pageref{sec-MCU}\\
 \item{\bf KUPTER  }[v] log. unit for the terminal:
 \ref{sec-MCU} on p.~\pageref{sec-MCU} and \ref{sec-MCU} on p.~\pageref{sec-MCU}
 
 \item{\bf KVBFLG  }[sf] vertex bit flags\\
 \item{\bf KVDAU   }[sf] access to tracks from a vertex: \ref{sec-AV} on p.~\pageref{sec-AV}\\
 \item{\bf KVFITA  }[f] kinematic fitting: \ref{sec-QFIT} on p.~\pageref{sec-QFIT}\\
 \item{\bf KVFITC  }[f] Fitting of N tracks with YTOP with mass constraint ,
 \ref{sec-QVFIT} on p.~\pageref{sec-QVFIT}\\
 \item{\bf KVFITM  }[f] kinematic fitting: \ref{sec-QFIT} on p.~\pageref{sec-QFIT}\\
 \item{\bf KVFITN  }[f] Fitting of N tracks with YTOP ,
 \ref{sec-QVFIT} on p.~\pageref{sec-QVFIT}\\
 \item{\bf KVFITV  }[f] Fitting of N tracks with YTOP with vertex constraint ,
 \ref{sec-QVFIT} on p.~\pageref{sec-QVFIT}\\
 \item{\bf KVFTMC  }[f] Fitting of a subset of n tracks with YTOP with mass constraint ,
 \ref{sec-QVFIT} on p.~\pageref{sec-QVFIT}\\
 \item{\bf KVGOOD  }[f] VDET readout: \ref{sec-KVGOOD} on p.~\pageref{sec-KVGOOD}\\
 \item{\bf KVINCP  }[sf] incoming particle to a vertex: \ref{sec-AV} on p.~\pageref{sec-AV}\\
 \item{\bf KVN     }[sf] Julia/Galeph vertex number: \ref{sec-TVAVA} on p.~\pageref{sec-TVAVA}\\
 \item{\bf KVNDAU  }[sf] number of outgoing tracks: \ref{sec-AV} on p.~\pageref{sec-AV}\\
 \item{\bf KVNEW   }[f] create new track vector: \ref{sec-QVN} on p.~\pageref{sec-QVN}\\
 \item{\bf KVSAVE  }[f] save track: \ref{sec-QVST} on p.~\pageref{sec-QVST}\\
 \item{\bf KVSAVC  }[f] save track in specific class:
 \ref{sec-QVSC} on p.~\pageref{sec-QVSC}\\
 \item{\bf KVTYPE  }[sf] vertex type : \ref{sec-TVAVA} on p.~\pageref{sec-TVAVA}
 
 \item{\bf KYV0xx  }[sf] bank YV0V: \ref{sec-TVAYV0V} on p.~\pageref{sec-TVAYV0V}
 
 \item{\bf LEP energy} see QELEP  \\
 \item{\bf lepton identification } for Heavy Flavours, see QSELEP : \ref{sec-OAQSELE} on p.~\pageref{sec-OAQSELE}\\
 \item{\bf lifetime   }on particle table: see QCLIFE / QPLIFE:
 \ref{sec-PTAC} on p.~\pageref{sec-PTAC}\\
 \item{\bf line printer    }see KUPRNT\\
 \item{\bf LJET } Jets built by subroutine QSELEP : \ref{sec-QSELJT} on p.~\pageref{sec-QSELJT}\\
 \item{\bf lock        }\ref{sec-QL} on p.~\pageref{sec-QL}\\
 \item{\bf logical units  }\ref{sec-MCU} on p.~\pageref{sec-MCU}\\
 \item{\bf loops       }over tracks (= vectors) and vertices:
 \ref{sec-A} on p.~\pageref{sec-A}\\
 \item{\bf Lorentz     }transformations: \ref{sec-QT} on p.~\pageref{sec-QT};
 see also decay angle: \ref{sec-MK} on p.~\pageref{sec-MK}\\
 \item{\bf LSxx    }data cards for QSELEP: \ref{sec-QSELCU} on p.~\pageref{sec-QSELCU}\\
 \item{\bf LUCLUS }jet finding algorithm: \ref{sec-QJLU} on p.~\pageref{sec-QJLU}\\
 \item{\bf Luminosity  }value for current run: see XIOKLU,QRINLU,XIOKSI,QRSLLU
 
 \item{\bf main program    }see QMAIN\\
 \item{\bf mass    }\\
 \subitem of an individual particle: \ref{sec-TVABA} on p.~\pageref{sec-TVABA}\\
 \subitem invariant mass of a system of particles: \ref{sec-MK} on p.~\pageref{sec-MK}\\
 \subitem missing mass: \ref{sec-QJME} on p.~\pageref{sec-QJME}\\
 \subitem nominal mass in the particle table: \ref{sec-PTAC} on p.~\pageref{sec-PTAC}\\
 \item{\bf match   }reconstructed charged tracks and MC particles: \ref{sec-AX} on p.~\pageref{sec-AX}\\
 \item{\bf match   }Energy Flow objects and MC particles: \ref{sec-EFLWMA} on p.~\pageref{sec-EFLWMA} \\
 \item{\bf MCMATCH }[s] matching between reconstructed charged tracks and MC truth:
 \ref{sec-OAMCMAT} on p.~\pageref{sec-OAMCMAT}\\
 \item{\bf MEXT    }data card: to force the muon extrapolation in HCAL
 \ref{sec-DCSPCA} on p.~\pageref{sec-DCSPCA}\\
 \item{\bf MINA    }data card: add non-standard banks to a private MINI, App. ~\ref{sec-wtm} on p.~\pageref{sec-wtm}\\
 \item{\bf Mini-DST   }App.~\ref{sec-miniapp} on p.~\pageref{sec-miniapp}\\
 \item{\bf MINI}\\
 \subitem card: ~\ref{sec-DCFILO} on  p.~\pageref{sec-DCFILO} and App.~\ref{sec-miniapp} on
 p.~\pageref{sec-miniapp}\\
 \subitem flag for Mini-DST input: \ref{sec-MCE} on p.~\pageref{sec-MCE}\\
 \item{\bf MINP}data card for official MINI writing
 :~\ref{sec-DCFILO}on p.~\pageref{sec-DCFILO} and App.~\ref{sec-miniapp} on
 p.~\pageref{sec-miniapp}\\
 \item{\bf missing     }mass, energy, momentum: \ref{sec-QJME} on p.~\pageref{sec-QJME}\\
 \item{\bf momentum    }of a particle see QP \ref{sec-TVABA} on p.~\pageref{sec-TVABA} /
 missing momentum \ref{sec-QJME} on p.~\pageref{sec-QJME}\\
 \item{\bf Monte Carlo }\\
 \subitem flag for an event: \ref{sec-MCE} on p.~\pageref{sec-MCE}\\
 \subitem loops over MC particles: \ref{sec-AL} on p.~\pageref{sec-AL} and
 \ref{sec-AD} on p.~\pageref{sec-AD}\\
 \subitem particle code:~\ref{sec-PTD} on p.~\pageref{sec-PTD}\\
 \subitem particle table:~\ref{sec-PTD} on p.~\pageref{sec-PTD}\\
 \item{\bf mother particle }\ref{sec-AMD} on p.~\pageref{sec-AMD}\\
 \item{\bf MUID    }access to MUID (QMUIDO)
 information: \ref{sec-TVAMUID} on p.~\pageref{sec-TVAMUID}
 
 \item{\bf Nano-DST   }App.~\ref{sec-NANinf} on p.~\pageref{sec-NANinf}\\
 \item{\bf NANCOM     }Include File to use when reading a Nano :
      see previous edition of this manual           \\
 \item{\bf NATIVE  }file type: machine--dependent input/output
 \ref{sec-DCFT} on p.~\pageref{sec-DCFT}\\
 \item{\bf NATIVE  }parameter on FILI / FILO data cards (q.v.)\\
 \item{\bf NEVT    }data card: select NEVT events: \ref{sec-DCRS} on p.~\pageref{sec-DCRS}\\
 \item{\bf new track   }KVNEW: \ref{sec-QVN} on p.~\pageref{sec-QVN}\\
 \item{\bf nominal mass    }on particle table: \ref{sec-PTAC} on p.~\pageref{sec-PTAC}\\
 \item{\bf NOBG    }data card for QIPBTAG: \ref{sec-QIPBCD} on p.~\pageref{sec-QIPBCD}\\
 \item{\bf NOOV    }parameter on FILO / HIST data cards (q.v.)\\
 \item{\bf NOPH    }no hostogram printing: \ref{sec-HISTP} on p.~\pageref{sec-HISTP}\\
 \item{\bf NOPX    }data card to suppress rebuilding of PTPX bank for pad dE/dx: \ref{sec-DCSPCA} on p.~\pageref{sec-DCSPCA}\\
 \item{\bf NORU    }parameter on FILO data card: \ref{sec-DCFILO} on p.~\pageref{sec-DCFILO}\\
 \item{\bf NOxx    }ALPHA process cards: \ref{sec-DCPC} on p.~\pageref{sec-DCPC}\\
 \item{\bf NQIP    }data card for QIPBTAG: \ref{sec-QIPBCD} on p.~\pageref{sec-QIPBCD}\\
 \item{\bf NREC    }parameter on HIST data card : \ref{sec-HISTW} on p.~\pageref{sec-HISTW}\\
 \item{\bf NSEQ    }data card: to read files with runs in any order :
  \ref{sec-DCRS} on p.~\pageref{sec-DCRS}\\
 \item{\bf NSID    }data card for ENFLW to suppress SICAL cleaning: \ref{sec-DCSPCA} on p.~\pageref{sec-DCSPCA}\\
 \item{\bf Ntuples     }Ch.~\ref{sec-HIST} on p.~\pageref{sec-HIST}\\
 \item{\bf NUMJ    }data card for QIPBTAG: \ref{sec-QIPBCD} on p.~\pageref{sec-QIPBCD}\\
 \item{\bf NWRT    }data card: Write  NWRT events: \ref{sec-DCRS} on p.~\pageref{sec-DCRS}
 
 
 \item{\bf OQIP    }data card for QIPBTAG: \ref{sec-QIPBCD} on p.~\pageref{sec-QIPBCD}\\
 \item{\bf output  }\\
 \subitem events -- see FILO card:~\ref{sec-DCFILO} on p.~\pageref{sec-DCFILO}
 and routine QWRITE:
 \ref{sec-QWR} on p.~\pageref{sec-QWR}\\
 \subitem histograms -- see HIST data card: \ref{sec-HISTW} on p.~\pageref{sec-HISTW}
 
 
 \item{\bf parameters  }HAC parameters: \ref{sec-UHAC} on p.~\pageref{sec-UHAC}\\
 \item{\bf particle}\\
 \subitem analysis of particle systems: \ref{sec-ADS} on p.~\pageref{sec-ADS}\\
 \subitem --antiparticle relation: \ref{sec-ADA} on p.~\pageref{sec-ADA}\\
 \subitem attributes: \ref{sec-PTAC} on p.~\pageref{sec-PTAC}\\
 \subitem code: \ref{sec-ADT} on p.~\pageref{sec-ADT} and
 \ref{sec-PTN} on p.~\pageref{sec-PTN}\\
 \subitem direct access to specific particles: \ref{sec-AD} on p.~\pageref{sec-AD}\\
 \subitem invariant mass of particle systems: \ref{sec-MK} on p.~\pageref{sec-MK}\\
 \subitem table\\
 \subsubitem data cards: \ref{sec-PTDC} on p.~\pageref{sec-PTDC}\\
 \subsubitem MC table: \ref{sec-PTD} on p.~\pageref{sec-PTD} and
 \ref{sec-PTDC} on p.~\pageref{sec-PTDC}\\
 \subsubitem standard table \ref{sec-PTDC} on p.~\pageref{sec-PTDC}\\
 \item{\bf PAW     }interactive analysis of histograms and Ntuples:
 \ref{sec-HISTW} on p.~\pageref{sec-HISTW}\\
 \item{\bf PCOR    }data card to call QPCORR automatically: \ref{sec-DCSPCA} on p.~\pageref{sec-DCSPCA}
                    and \ref{sec-QPCOR} on p.~\pageref{sec-QPCOR}\\
 \item{\bf PCPA }neutral objects from PCPA: \ref{sec-altrack} on p.~\pageref{sec-altrack} and
 \ref{sec-EFLWP} on p.~\pageref{sec-EFLWP}\\
 \item{\bf PGAC    } bank for photons in ECAL, \ref{sec-TVAPGAC} on p.~\pageref{sec-TVAPGAC}\\
 \item{\bf PGPC    }see GAMPEX ( banks for photons, obsolete )\\
 \item{\bf photon conversions } see QPAIRF: \ref{sec-OARPAIR} on p.~\pageref{sec-OARPAIR}\\
 \item{\bf photons} from GAMPEC: \ref{sec-altrack} on p.~\pageref{sec-altrack} and
 \ref{sec-TVAEGPC} on p.~\pageref{sec-TVAEGPC}\\
 \item{\bf pi      }constant: \ref{sec-MCC} on p.~\pageref{sec-MCC}\\
 \item{\bf PI0DEB  }[s] Debug printout of $\pi^0$ found by QPI0DO
 \ref{sec-OARQPI0} on p.~\pageref{sec-OARQPI0}\\
 \item{\bf Planck  }constant: \ref{sec-MCC} on p.~\pageref{sec-MCC}\\
 \item{\bf PMOD    }data card: modify particle table \ref{sec-PTPMOD} on p.~\pageref{sec-PTPMOD}\\
 \item{\bf PNEW    }data card: new entry into particle table \ref{sec-PTPNEW}
 on p.~\pageref{sec-PTPNEW}\\
 \item{\bf PTRA    }data card: modify MC particle code translation:
 \ref{sec-PTPTRA} on p.~\pageref{sec-PTPTRA}\\
 \item{\bf POT     }unpacking: \ref{sec-DCUNPK} on p.~\pageref{sec-DCUNPK}\\
 \item{\bf process }ALPHA process cards: \ref{sec-DCPC} on p.~\pageref{sec-DCPC}\\
 \item{\bf PTCLUS }jet finding algorithm: \ref{sec-QJPT} on p.~\pageref{sec-QJPT}\\
 \item{\bf PVM    }Running ALPHA in parallel on SAGA: \ref{sec-alphar} on p.~\pageref{sec-alphar}
 
 
 \item{\bf QBEAMX }[s] size of luminous region: \ref{sec-OAQBEAM} on p.~\pageref{sec-OAQBEAM}\\
 \item{\bf QBETA   }[sf] beta of a particle: \ref{sec-MK} on p.~\pageref{sec-MK}\\
 \item{\bf QBMTAG  }[s] Invariant-mass b-tagging: \ref{sec-BMTAG} on p.~\pageref{sec-BMTAG}\\
 \item{\bf QBOOKN  }[s] book Ntuples: \ref{sec-QBN} on p.~\pageref{sec-QBN}\\
 \item{\bf QBOOKR  }[s] book Ntuples with run and event number:
 \ref{sec-QBR} on p.~\pageref{sec-QBR}\\
 \item{\bf QBOOK1  }[s] book 1--dimensional histograms: \ref{sec-QB1} on p.~\pageref{sec-QB1}\\
 \item{\bf QBOOK2  }[s] book 2--dimensional histograms: \ref{sec-QB2} on p.~\pageref{sec-QB2}\\
 \item{\bf QBPROF  }[s] book Profile histograms: \ref{sec-QBP} on p.~\pageref{sec-QBP}\\
 \item{\bf QCDE    }macro: all parameters, commons etc.: \ref{sec-UA} on p.~\pageref{sec-UA}\\
 \item{\bf QCDESH  }short subset of QCDE\\
 \item{\bf QCFxxx  }macros containing statement functions\\
 \item{\bf QCH     }[sf] track's charge: \ref{sec-TVABA} on p.~\pageref{sec-TVABA}\\
 \item{\bf QCOSA   }[sf] cos (angle between two tracks):
 \ref{sec-MK} on p.~\pageref{sec-MK}\\
 \item{\bf QCT     }[sf] cos (theta): \ref{sec-MK} on p.~\pageref{sec-MK}
 
 \item{\bf QDATA   }[s] (quasi) block data\\
 \item{\bf QDB     }[sf] distance to beam axis: \ref{sec-TVAD} on p.~\pageref{sec-TVAD}\\
 \item{\bf QDBS2   }[sf] error$^2$ on QDB: \ref{sec-TVAD} on p.~\pageref{sec-TVAD}\\
 \item{\bf QDDX    }[s] combined dE/dx estimation using pads and wires: \ref{sec-OARDDX} on p.~\pageref{sec-OARDDX}\\
 \item{\bf QDECAN  }[f] decay angle: \ref{sec-MK} on p.~\pageref{sec-MK}\\
 \item{\bf QDECA2  }[f] decay angle: \ref{sec-MK} on p.~\pageref{sec-MK}\\
 \item{\bf QDEDX   }[s] dE/dx analysis: \ref{sec-OARDEDX} on
 p.~\pageref{sec-OARDEDX}\\
 \item{\bf QDEDXM  }[s] dE/dx analysis (MCarlo datasets): \ref{sec-OARDEDM} on
 p.~\pageref{sec-OARDEDM}\\
 \item{\bf QDHExx  }[v] header bank DHEA: obsolete since May 1993\\
 \item{\bf QDMMCL  }[s]jet finding -- scaled invariant mass sq. DURHAM algorithm:
 \ref{sec-QDMMCL} on p.~\pageref{sec-QDMMCL}\\
 \item{\bf QDMSQ   }[sf] mass difference $^2$: \ref{sec-MK} on p.~\pageref{sec-MK}\\
 \item{\bf QDOT3   }[sf] dot product (3--vector): \ref{sec-MK} on p.~\pageref{sec-MK}\\
 \item{\bf QDOT4   }[sf] dot product (4--vector): \ref{sec-MK} on p.~\pageref{sec-MK}
 
 \item{\bf QE      }[sf] energy: \ref{sec-TVABA} on p.~\pageref{sec-TVABA}\\
 \item{\bf QEECWI  }[v] ECAL wire energy: \ref{sec-ECWI} on p.~\pageref{sec-ECWI}\\
 \item{\bf QEIDxx  }[sf] bank EIDT = electron identification:
 \ref{sec-TVAEIDT} on p.~\pageref{sec-TVAEIDT}.\\
 \item{\bf QELEP   }[v] LEP c.m.s. energy, in GeV: \ref{sec-CHUNKELEP} on p.~\pageref{sec-CHUNKELEP}\\
 \item{\bf QEWSUM   }[s] ECAL wire energy on even/odd wire planes: \ref{sec-EWSU2} on p.~\pageref{sec-EWSU2}\\
 \item{\bf QFILBP$\_$STATUS   }[s] to know how the beam position with BOMS was found:
  \ref{sec-BOMSTAT} on p.~\pageref{sec-BOMSTAT}\\
 \item{\bf QFND    }data card: to call the QFNDIP package
 \ref{sec-DCQFND} on p.~\pageref{sec-DCQFND}\\
 \item{\bf QFNDIP  }[s]event interaction point finding routine
 \ref{sec-DCQFND} on p.~\pageref{sec-DCQFND}\\
 \item{\bf QFRFxx  }[sf] bank FRFT = track fit: \ref{sec-TVAFRFT} on p.~\pageref{sec-TVAFRFT}\\
 \item{\bf QFRIxx  }[sf] bank FRID: \ref{sec-TVAFRID} on p.~\pageref{sec-TVAFRID}\\
 \item{\bf QFRTxx  }[sf] bank FRTL = appendix to FRFT:
 \ref{sec-TVAFRTL} on p.~\pageref{sec-TVAFRTL}\\
 \item{\bf QGAMMA  }[sf] particle's gamma: \ref{sec-MK} on p.~\pageref{sec-MK}\\
 \item{\bf QGJMMC  }[s]jet finding : ALPHA interface to ALEPHLIB FJMMCL routine :
 \ref{sec-QGJMMC} on p.~\pageref{sec-QGJMMC}\\
 \item{\bf QHFN    }[s] fill Ntuple: \ref{sec-QBFN} on p.~\pageref{sec-QBFN}\\
 \item{\bf QHFNR   }[s] fill Ntuple with run and event number:
 \ref{sec-QBFN} on p.~\pageref{sec-QBFN}\\
 \item{\bf QHFR    }[s] fill Ntuple with run and event number:
 \ref{sec-QBFR} on p.~\pageref{sec-QBFR}\\
 \item{\bf QHMAxx  }[sf] bank HMAD = HCAL--muon association:
 \ref{sec-TVAHMAD} on p.~\pageref{sec-TVAHMAD}
 
 \item{\bf QIDV0   }[s]Recalculate V0 4--vector: \ref{sec-QIDV0} on p.~\pageref{sec-QIDV0}
 
 \item{\bf QIPBTAG }[f] B-Tagging routine using impact parameter method
 \ref{sec-OARQIPB} on p.~\pageref{sec-OARQIPB}\\
 \item{\bf QITI    }data card for QIPBTAG: \ref{sec-QIPBCD} on p.~\pageref{sec-QIPBCD}\\
 \item{\bf QJADDP  }[s]add 4--vectors: \ref{sec-QJA} on p.~\pageref{sec-QJA}\\
 \item{\bf QJEIG   }[s]eigenvalues of mom. tensor: \ref{sec-QJEI} on p.~\pageref{sec-QJEI}\\
 \item{\bf QJFOXW  }[s]Fox--Wolfram moments: \ref{sec-QJFW} on p.~\pageref{sec-QJFW}\\
 \item{\bf QJHEMI  }[s]divide the event into two hemispheres:
 \ref{sec-QJHE} on p.~\pageref{sec-QJHE}\\
 \item{\bf QJMISS  }[s]missing energy, mass, and momentum:
 \ref{sec-QJME} on p.~\pageref{sec-QJME}\\
 \item{\bf QJMDCL  }[s]jet finding -- scaled minimum distance algorithm:
 \ref{sec-QJMD} on p.~\pageref{sec-QJMD}\\
 \item{\bf QJMMCL  }[s]jet finding -- scaled invariant mass sq. JADE algorithm:
 \ref{sec-QJSIM} on p.~\pageref{sec-QJSIM}\\
 \item{\bf QJLUCL }[s]jet finding -- LUCLUS: \ref{sec-QJLU} on p.~\pageref{sec-QJLU}\\
 \item{\bf QJOPTM }[s]select MC particles for QJxxxx routines: \ref{sec-QJOMC}
 on p.~\pageref{sec-QJOMC}\\
 \item{\bf QJOPTR }[s]select reconstructed objects for QJxxxx routines: \ref{sec-QJORE}
 on p.~\pageref{sec-QJORE}\\
 \item{\bf QJPTCL }[s]jet finding -- PTCLUS: \ref{sec-QJPT}
 on p.~\pageref{sec-QJPT}\\
 \item{\bf QJSPHE }[s]sphericity: \ref{sec-QJSP} on p.~\pageref{sec-QJSP}\\
 \item{\bf QJTENS  }[s]linearized momentum tensor: \ref{sec-QJEN} on
 p.~\pageref{sec-QJEN}\\
 \item{\bf QJTHRU  }[s]thrust value / axis: \ref{sec-QJTH}
 on p.~\pageref{sec-QJTH}
 
 \item{\bf QKEVxx  }[v] bank KEVH: \ref{sec-MHK} on p.~\pageref{sec-MHK}
 
 \item{\bf QKINKT  }[s] to get the list of tracks ending in Kink Vertices: \ref{sec-TVKINK} on p.~\pageref{sec-TVKINK}
 
 \item{\bf QKINKV  }[s] to get first/last Kink Vertices: \ref{sec-alvert} on p.~\pageref{sec-alvert}
 
 \item{\bf QLEPxx  }[sf] Output variables from subroutine QSELEP : \ref{sec-QSELTL} on p.~\pageref{sec-QSELTL}\\
 \item{\bf QLID } data card to trigger the execution of QSELEP : \ref{sec-OAQSELE} on p.~\pageref{sec-OAQSELE}\\
 \item{\bf QLTRK   }[s] lock individual track: \ref{sec-QLI} on p.~\pageref{sec-QLI}\\
 \item{\bf QLOCK   }[s] lock track family: \ref{sec-QLO} on p.~\pageref{sec-QLO}\\
 \item{\bf QLOCK2  }[s] lock track family: \ref{sec-QL2} on p.~\pageref{sec-QL2}\\
 \item{\bf QLREV   }[s] reverse lock: \ref{sec-QLR} on p.~\pageref{sec-QLR}\\
 \item{\bf QLREV2  }[s] reverse lock: \ref{sec-QL2} on p.~\pageref{sec-QL2}\\
 \item{\bf QLUTRK  }[s] unlock individual track: \ref{sec-QLU} on p.~\pageref{sec-QLU}\\
 \item{\bf QLV0T   }[s] to get first/last Long V0 Tracks: \ref{sec-altrack} on p.~\pageref{sec-altrack}\\
 \item{\bf QLV0V   }[s] to get first/last Long V0 Vertices: \ref{sec-alvert} on p.~\pageref{sec-alvert}
 
 \item{\bf QLZER   }[s] zero lock: \ref{sec-QLZ} on p.~\pageref{sec-QLZ}\\
 \item{\bf QLZER2  }[s] zero lock: \ref{sec-QL2} on p.~\pageref{sec-QL2}
 
 \item{\bf QM      }[sf] particle's mass: \ref{sec-TVABA} on p.~\pageref{sec-TVABA}\\
 \item{\bf QMACRO  }macro: statement functions: \ref{sec-UA} on p.~\pageref{sec-UA}\\
 \item{\bf QMAIN   }ALPHA main program: App.~\ref{sec-PSTRUC} on p.~\pageref{sec-PSTRUC}\\
 \item{\bf QMASV0  }[f]V0 mass: \ref{sec-TVAV0M} on p.~\pageref{sec-TVAV0M}; see also QIDV0.\\
 \item{\bf QMCAxx  }[sf] bank MCAD = muon chambers: \ref{sec-TVAMCAD} on p.~\pageref{sec-TVAMCAD}\\
 \item{\bf QMCHI2 }[f] $\chi^2$ from mass difference: \ref{sec-MK} on
 p.~\pageref{sec-MK}\\
 \item{\bf QMCHIF }[f] $\chi^2$ for track mass-constrained fit : \ref{sec-MK} on
 p.~\pageref{sec-MK}\\
 \item{\bf QMCHIV }[f] $\chi^2/$NDF for vertex fit : \ref{sec-MK} on
 p.~\pageref{sec-MK}\\
 \item{\bf QMDIFF }[f] mass difference: \ref{sec-MK} on
 p.~\pageref{sec-MK}\\
 \item{\bf QMFLD   }[v] ALEPH magnetic field: \ref{sec-MR} on p.~\pageref{sec-MR}\\
 \item{\bf QMINIT  }[s] system initialization: \ref{sec-UI} on p.~\pageref{sec-UI}\\
 \item{\bf QMUIDO }[s] muon identification: \ref{sec-OARMUID} on p.~\pageref{sec-OARMUID} and
 \ref{sec-TVAMUID} on p.~\pageref{sec-TVAMUID}\\
 \item{\bf QMSQ2,QMSQ3,QMSQ4 }[sf] invariant mass$^2$: \ref{sec-MK} on p.~\pageref{sec-MK}\\
 \item{\bf QMTERM  }[s] system termination:
 \ref{sec-UT} on p.~\pageref{sec-UT} and \ref{sec-QMT} on p.~\pageref{sec-QMT}\\
 \item{\bf QM2,QM3,QM4 }[sf] invariant mass: \ref{sec-MK} on p.~\pageref{sec-MK}\\
 \item{\bf QNCDE   }Include File to use when reading a Nano :
      see previous edition of this manual\\
 \item{\bf QNMACR  }Macro of statement functions to be used when reading a Nano :
      see previous edition of this manual
 
 \item{\bf QNTEX   }[sf] number of sectors for dE/dx: \ref{sec-TVATEXS} on p.~\pageref{sec-TVATEXS} \\
 \item{\bf QNUCL   }[s] to get first/last Nuclear Interaction Vertices: \ref{sec-alvert} on p.~\pageref{sec-alvert}\\
 \item{\bf QP      }[sf] momentum: \ref{sec-TVABA} on p.~\pageref{sec-TVABA}\\
 \item{\bf QPAIRF  }[s] photon conversions: \ref{sec-OARPAIR} on
 p.~\pageref{sec-OARPAIR}\\
 \item{\bf QPCHAR  }[f] particle table charge: \ref{sec-PTAC} on p.~\pageref{sec-PTAC}\\
 \item{\bf QPCORR  }[s] ``sagitta correction" to charged particle momenta: \ref{sec-QPCOR} on p.
 ~\pageref{sec-QPCOR}
 see also the PCOR data card \ref{sec-DCSPCA} on p.~\pageref{sec-DCSPCA}   \\
 \item{\bf QPECxx  }[sf] bank PECO: \ref{sec-TVAPECO} on p.~\pageref{sec-TVAPECO}\\
 \item{\bf QPEPxx  }[sf] bank PEPT: \ref{sec-TVAPEPT} on p.~\pageref{sec-TVAPEPT}\\
 \item{\bf QPGAxx  }[sf] bank PGAC: \ref{sec-TVAPGAC} on p.~\pageref{sec-TVAPGAC}\\
 \item{\bf QPHCxx  }[sf] bank PHCO: \ref{sec-TVAPHCO} on p.~\pageref{sec-TVAPHCO}\\
 \item{\bf QPH     }[sf] track's azimuth: \ref{sec-MK} on p.~\pageref{sec-MK}\\
 \item{\bf QPI0BK  }[f] Subroutine booking internal histograms for QPI0DO
 \ref{sec-OARQPI0} on p.~\pageref{sec-OARQPI0}\\
 \item{\bf QPI0DO  }[f] $\pi^0$ finding routine
 \ref{sec-OARQPI0} on p.~\pageref{sec-OARQPI0}\\
 \item{\bf QPLIFE  }[f] particle table life time: \ref{sec-PTAC} on p.~\pageref{sec-PTAC}\\
 \item{\bf QPMASS  }[f] particle table mass: \ref{sec-PTAC} on p.~\pageref{sec-PTAC}\\
 \item{\bf QPPAR   }[sf] momentum parallel to a vector: \ref{sec-MK} on p.~\pageref{sec-MK}\\
 \item{\bf QPPER   }[sf] momentum perpendicular to a vector: \ref{sec-MK} on p.~\pageref{sec-MK}\\
 \item{\bf QPT     }[sf] transverse momentum: \ref{sec-MK} on p.~\pageref{sec-MK}\\
 \item{\bf QPWIDT  }[f] particle table width: \ref{sec-PTAC} on p.~\pageref{sec-PTAC}
 
 \item{\bf QQC     }[c] speed of light: \ref{sec-MCC} on p.~\pageref{sec-MCC}\\
 \item{\bf QQE     }[c] e: \ref{sec-MCC} on p.~\pageref{sec-MCC}\\
 \item{\bf QQH     }[c] hbar: \ref{sec-MCC} on p.~\pageref{sec-MCC}\\
 \item{\bf QQHC    }[c] hbar * c \ref{sec-MCC} on p.~\pageref{sec-MCC}\\
 \item{\bf QQIRP   }[c] factor between inv. bending radius and momentum:
 \ref{sec-MCC} on p.~\pageref{sec-MCC}\\
 \item{\bf QQPI    }[c] $\pi$: \ref{sec-MCC} on p.~\pageref{sec-MCC}\\
 \item{\bf QQPIH   }[c] $\pi$ / 2: \ref{sec-MCC} on p.~\pageref{sec-MCC}\\
 \item{\bf QQRADP  }[c] 360 / $\pi$: \ref{sec-MCC} on p.~\pageref{sec-MCC}\\
 \item{\bf QQ2PI   }[c] 2 $\pi$: \ref{sec-MCC} on p.~\pageref{sec-MCC}
 
 \item{\bf QRDFL   }[sf] read user flag: \ref{sec-TVAFP} on p.~\pageref{sec-TVAFP}\\
 \item{\bf QRINLU  }[sf] LCAL  luminosity for current run: \ref{sec-MR} on p.~\pageref{sec-MR}\\
 \item{\bf QRSLLU  }[sf] SiCAL luminosity for current run: \ref{sec-MR} on p.~\pageref{sec-MR}\\
 \item{\bf QSELEP  }[s] Lepton Identification for Heavy Flavours : \ref{sec-OAQSELE} on p.~\pageref{sec-OAQSELE}\\
 \item{\bf QSIGxx  }[sf] track's error matrix: \ref{sec-TVATM} on p.~\pageref{sec-TVATM}\\
 \item{\bf QSTFLI  }[s] set user flag (integer): \ref{sec-USFL} on p.~\pageref{sec-USFL}\\
 \item{\bf QSTFLR  }[s] set user flag (real): \ref{sec-USFL} on p.~\pageref{sec-USFL}\\
 \item{\bf QSTRU   }[f] matching quantity for ENFLW/MC matching:\ref{sec-EFLWMA} on p.~\pageref{sec-EFLWMA}
 
 \item{\bf QSUSTR  }[s] allocate user's track space \ref{sec-MBRST} on p.~\pageref{sec-MBRST}\\
 \item{\bf QSUSVX  }[s] allocate user's vertex space \ref{sec-MBRSV} on p.~\pageref{sec-MBRSV}
 
 \item{\bf QTCLAS  }[s] Lorentz transformation: \ref{sec-QTC} on p.~\pageref{sec-QTC}\\
 \item{\bf QTEXxx  }[sf] bank TEXS = dE/dx: \ref{sec-TVATEXS} on p.~\pageref{sec-TVATEXS}\\
 \item{\bf QTIME   }[v] as given on the TIME data card: \ref{sec-MCT} on p.~\pageref{sec-MCT}\\
 \item{\bf QTIMEL  }[v] remaining job time: \ref{sec-MCT} on p.~\pageref{sec-MCT}\\
 \item{\bf QTRUTH  }[s] History of a reconstructed MCarlo track : \ref{sec-OAQTRUT} on p.~\pageref{sec-OAQTRUT}
 
 \item{\bf QUEVNT  }[s] event processing user routine:
 \ref{sec-UE} on p.~\pageref{sec-UE}\\
 \item{\bf QUIBOS  }[s] initialize BOS: \ref{sec-QUIB} on p.~\pageref{sec-QUIB}\\
 \item{\bf QUIHIS  }[s] initialize histograms: \ref{sec-QUIH} on p.~\pageref{sec-QUIH}\\
 \item{\bf QUINIT  }[s] user initialization routine:
 \ref{sec-UI} on p.~\pageref{sec-UI}\\
 \item{\bf QUNEWR  }[s] user routine: called for every new run:
 \ref{sec-QUN} on p.~\pageref{sec-QUN}\\
 \item{\bf QUSREC }[s] special records: \ref{sec-QUSREC} on p.~\pageref{sec-QUSREC}\\
 \item{\bf QUTERM  }[s] user termination routine:
 \ref{sec-UT} on p.~\pageref{sec-UT}\\
 \item{\bf QUTHIS  }[s] terminate histograms: \ref{sec-QUTH} on p.~\pageref{sec-QUTH} \\
 \item{\bf QVADD2, QVADD3, QVADD4, QVADDN}
 [s] add track vectors: \ref{sec-QVA} on p.~\pageref{sec-QVA}\\
 \item{\bf QVCHIF  }[sf] chisquare of vertex fit: \ref{sec-TVAVA} on p.~\pageref{sec-TVAVA}\\
 \item{\bf QVCOPY  }[s] copy track vectors: \ref{sec-QVC} on p.~\pageref{sec-QVC}\\
 \item{\bf QVCROS  }[s] cross product: \ref{sec-QVX} on p.~\pageref{sec-QVX}\\
 \item{\bf QVDHIT  }[s] VDET hits: \ref{sec-QVDHIT} on p.~\pageref{sec-QVDHIT}\\
 \item{\bf QVDROP  }[s] drop tracks: \ref{sec-QVD} on p.~\pageref{sec-QVD}\\
 \item{\bf QVEM    }[sf] vertex error matrix: \ref{sec-TVAVA} on p.~\pageref{sec-TVAVA}\\
 \item{\bf QVGETS  }[s] copy error matrix into Fortran array:
 \ref{sec-QVG} on p.~\pageref{sec-QVG}\\
 \item{\bf QVGET3,QVGET4   }[s] copy track vector into Fortran array:
 \ref{sec-QVG} on p.~\pageref{sec-QVG}\\
 \item{\bf QVKINK  }[s] get special attributes of a kink vertex \ref{sec-TVKINK} on p.~\pageref{sec-TVKINK}\\
 \item{\bf QVSCAL  }[s] scale track momentum: \ref{sec-QVM} on p.~\pageref{sec-QVM}\\
 \item{\bf QVSETM  }[s] set mass of a track: \ref{sec-QVM} on p.~\pageref{sec-QVM}\\
 \item{\bf QVSETS  }[s] copy Fortran array into error matrix:
 \ref{sec-QVM} on p.~\pageref{sec-QVM}\\
 \item{\bf QVSET3,QVSET4   }[s] copy Fortran array into track vector:
 \ref{sec-QVM} on p.~\pageref{sec-QVM}\\
 \item{\bf QVSRCH  }[f] Routine for secondary vertices and B-tagging
 \ref{sec-OARVSRC} on p.~\pageref{sec-OARVSRC}\\
 \item{\bf QVSUB   }[s] subtract track vectors: \ref{sec-QVSU} on p.~\pageref{sec-QVSU}\\
 \item{\bf QVTEBP(I) }[v] i=1,2,3 : x,y,z errors on beam position from GET\_BP
 \ref{sec-MR} on p.~\pageref{sec-MR}\\
 \item{\bf QVTSBP(I) }[v] i=1,2,3 : x,y,z size of the beam spot from GET\_BP
 \ref{sec-MR} on p.~\pageref{sec-MR}\\
 \item{\bf QVTXBP(I) }[v] i=1,2,3 : x,y,z beam position from GET\_BP
 \ref{sec-MR} on p.~\pageref{sec-MR}\\
 \item{\bf QVX,QVY,QVZ }[sf] vertex position: \ref{sec-TVAVA} on p.~\pageref{sec-TVAVA}\\
 \item{\bf QV0CHK  }[f] Computes chisq of V0 track w.r.t main vertex ,
 \ref{sec-QV0CK} on p.~\pageref{sec-QV0CK}\\
 \item{\bf QVZERO  }[s] zero track vector: \ref{sec-QVZ} on p.~\pageref{sec-QVZ} \\
 \item{\bf QWCLAS  }[s] set classification word for EDIRs: \ref{sec-QWCLAS} on p.~\pageref{sec-QWCLAS}\\
 \item{\bf QWEVNT  }[s] print whole event: \ref{sec-QWE} on p.~\pageref{sec-QWE}\\
 \item{\bf QWHEAD  }[s] print event header: \ref{sec-QWH} on p.~\pageref{sec-QWH}\\
 \item{\bf QWHFUL  }[s] print full event
 header: \ref{sec-QWHF} on p.~\pageref{sec-QWHF}\\
 \item{\bf QWHICH$\_$BP  }[s] to know how the beam spot was found:
  \ref{sec-EBSPOT} on p.~\pageref{sec-EBSPOT}\\
 \item{\bf QWHICH$\_$EN  }[s] to know how the LEP energy QELEP was found:
  \ref{sec-ELEP2} on p.~\pageref{sec-ELEP2}\\
 \item{\bf QWITK   }[s] print individual track(s): \ref{sec-QWTK} on p.~\pageref{sec-QWTK}\\
 \item{\bf QWIVX   }[s] print individual vertices: \ref{sec-QWV} on p.~\pageref{sec-QWV}\\
 \item{\bf QWMESS  }[s] message routine: \ref{sec-OARPM} on p.~\pageref{sec-OARPM}\\
 \item{\bf QWMESE  }[s] message routine: \ref{sec-OARPE} on p.~\pageref{sec-OARPE}\\
 \item{\bf QWRITE  }[s] event output routine: \ref{sec-QWR} on p.~\pageref{sec-QWR}\\
 \item{\bf QWSEC   }[s] print section of tracks/vertices: \ref{sec-QWS} on p.~\pageref{sec-QWS}\\
 \item{\bf QWTIME  }[s] print time consumption: \ref{sec-OARPT} on p.~\pageref{sec-OARPT}\\
 \item{\bf QWTREE  }[s] print decay chain tree: \ref{sec-QWTR} on p.~\pageref{sec-QWTR}\\
 \item{\bf QX      }[sf] x--momentum: \ref{sec-TVABA} on p.~\pageref{sec-TVABA}\\
 \item{\bf QY      }[sf] y--momentum: \ref{sec-TVABA} on p.~\pageref{sec-TVABA}\\
 \item{\bf QZ      }[sf] z--momentum: \ref{sec-TVABA} on p.~\pageref{sec-TVABA}\\
 \item{\bf QZB     }[sf] z--distance to interaction point: \ref{sec-TVAD} on p.~\pageref{sec-TVAD}\\
 \item{\bf QZBS2   }[sf] error$^2$ on QZB: \ref{sec-TVAD} on p.~\pageref{sec-TVAD}
 
 \item{\bf QYV0xx  }[sf] bank YV0V: \ref{sec-TVAYV0V} on p.~\pageref{sec-TVAYV0V}
 
 \item{\bf READ    }data card; read cards from several card
 files: \ref{sec-DCREAD} on p.~\pageref{sec-DCREAD}\\
 \item{\bf RECL    }parameter on HIST data card : \ref{sec-HISTW} on p.~\pageref{sec-HISTW}\\
 \item{\bf REPG    }data card: to redo the GAMPEK photon search on POT/DST
 \ref{sec-DCSPCA} on p.~\pageref{sec-DCSPCA}\\
 \item{\bf REV0    }data card: to redo the V0 finding on POT/DST
 \ref{sec-DCSPCA} on p.~\pageref{sec-DCSPCA}\\
 \item{\bf run }\\
 \subitem change: \ref{sec-QUN} on p.~\pageref{sec-QUN}\\
 \subitem information: \ref{sec-MR} on p.~\pageref{sec-MR}\\
 \subitem selection: \ref{sec-DCRS} on p.~\pageref{sec-DCRS}
 
 \item{\bf same}\\
 \subitem objects in diff. Lorentz frames, with diff. hypotheses:
 \ref{sec-AS} on p.~\pageref{sec-AS}\\
 \subitem  two particles based on the same object -- see
 XSAME:
 \ref{sec-TVATPSO} on p.~\pageref{sec-TVATPSO}\\
 \item{\bf save tracks }KVSAVE; KVSAVC: \ref{sec-QVSC} on p.~\pageref{sec-QVSC} and
 \ref{sec-QVST} on p.~\pageref{sec-QVST}\\
 \item{\bf scale momentum  }QVSCAL: \ref{sec-QVM} on p.~\pageref{sec-QVM}\\
 \item{\bf SCANBOOK  }interactive tool to create FILI cards: \ref{sec-DCFILI} on p.~\pageref{sec-DCFILI}\\
 \item{\bf selection   }see run/event selection: \ref{sec-DCRS} on p.~\pageref{sec-DCRS}\\
 \item{\bf SELR    }parameter on FILO data card \ref{sec-DCFILO} on p.~\pageref{sec-DCFILO}\\
 \item{\bf set mass    }QVSETM: \ref{sec-QVM} on p.~\pageref{sec-QVM}\\
 \item{\bf SEVT    }data card: select events \ref{sec-DCRS} on p.~\pageref{sec-DCRS}\\
 \item{\bf SFALPHA   }Obsolete: tool to run ALPHA on SHIFT, see alpharun
 \ref{sec-alphar} on p.~\pageref{sec-alphar}\\
 \item{\bf SIBE    }data card for MCarlo beam spots mearing \ref{sec-DCHUNK} on p.~\pageref{sec-DCHUNK}\\
 \item{\bf slow control }read s.c. records: \ref{sec-QUSREC} on p.~\pageref{sec-QUSREC}\\
 \item{\bf speed of light  }constant: \ref{sec-MCC} on p.~\pageref{sec-MCC}\\
 \item{\bf sphericity }\ref{sec-QJSP} on p.~\pageref{sec-QJSP}, \ref{sec-QJEI}
 on p.~\pageref{sec-QJEI}\\
 \item{\bf SRUN    }data card: select runs \ref{sec-DCRS} on p.~\pageref{sec-DCRS}\\
 \item{\bf stagelist  }interactive tool to know which datasets are staged on CERN UNIX computers:
                           \ref{sec-DCFILI} on p.~\pageref{sec-DCFILI}\\
 \item{\bf start ALPHA }interactively or in batch: Ch.~\ref{sec-GS} on p.~\pageref{sec-GS}\\
 \item{\bf STOP    }Fortran statement: forbidden:
 \ref{sec-UT} on p.~\pageref{sec-UT}\\
 \item{\bf submit a job    }Ch. \ref{sec-GS} on p.~\pageref{sec-GS}\\
 \item{\bf subtract    }track vectors: \ref{sec-QVSU} on p.~\pageref{sec-QVSU}\\
 \item{\bf SYNT    }data card: indicates a syntax check run
 \ref{sec-DCSYNT} on p.~\pageref{sec-DCSYNT}
 
 
 \item{\bf tapes }\ref{sec-DCFILI} on p.~\pageref{sec-DCFILI}\\
 \item{\bf terminal output }see KUPTER\\
 \item{\bf thrust }\ref{sec-QJTH} on p.~\pageref{sec-QJTH}\\
 \item{\bf TIME    }data card: time to terminate the job properly
 \ref{sec-DCTIME} on p.~\pageref{sec-DCTIME}\\
 \item{\bf time    }remaining job time: see QTIMEL \ref{sec-MCT} on p.~\pageref{sec-MCT}\\
 \item{\bf timing  }time consumption: \ref{sec-OARTI} on p.~\pageref{sec-OARTI}\\
 \item{\bf title   }general title for HBOOK histograms: \ref{sec-HISTT} on p.~\pageref{sec-HISTT}\\
 \item{\bf topology }routines: Ch.~\ref{sec-QJ} on p.~\pageref{sec-QJ}\\
 \item{\bf TRA2    }data card for QIPBTAG: \ref{sec-QIPBCD} on p.~\pageref{sec-QIPBCD}\\
 \item{\bf track }\\
 \subitem class:~\ref{sec-ADI} on p.~\pageref{sec-ADI}\\
 \subitem track number:~Ch. \ref{sec-A} on p.~\pageref{sec-A}\\
 \item{\bf trigger information }\ref{sec-TRIG} on p.~\pageref{sec-TRIG}\\
 \item{\bf TXTREE }[s] Create \LaTeX source for a MC true particle decay tree:
 \ref{sec-TXTR} on p.~\pageref{sec-TXTR}
 
 
 \item{\bf UNIX }App.~\ref{sec-ZB} on p.~\pageref{sec-ZB}\\
 \item{\bf unpack  }POT/DST/MINI unpacking: \ref{sec-DCUNPK} on p.~\pageref{sec-DCUNPK}\\
 \item{\bf UNPK    }data card: control POT/DST/MINI unpacking \ref{sec-DCUNPK} on p.~\pageref{sec-DCUNPK}\\
 \item{\bf UPDA    }parameter on HIST data card
 \ref{sec-HISTW} on p.~\pageref{sec-HISTW}\\
 \item{\bf unit    }log. input / output units: \ref{sec-MCU} on p.~\pageref{sec-MCU}\\
 \item{\bf units   }ALEPH phys. unit system:
 \ref{sec-M} on p.~\pageref{sec-M}\\
 \item{\bf user routines }Ch. \ref{sec-U} on p.~\pageref{sec-U}\\
 \item{\bf user track / vertex sections }\ref{sec-QVN} on p.~\pageref{sec-QVN}
 
 
 \item{\bf VDET  }\\
 \subitem utility routines: \ref{sec-OARVDET} on p.~\pageref{sec-OARVDET}\\
 \subitem tracks not using
 VDET: \ref{sec-FRF0} on p.~\pageref{sec-FRF0}\\
 \item{\bf VDHMATCH }[s] VDET hit matching between  reconstructed charged tracks and MC truth:
 \ref{sec-OAVDHMAT} on p.~\pageref{sec-OAVDHMAT}\\
 \item{\bf vector  }synonym for ``track'' or ``particle''\\
 \subitem class:~\ref{sec-ADI} on p.~\pageref{sec-ADI} and
 \ref{sec-TVAFP} on p.~\pageref{sec-TVAFP}\\
 \subitem number -- see track or vertex number\\
 \subitem operations:~\ref{sec-QV} on p.~\pageref{sec-QV}\\
 \item{\bf vertex number   }Ch. \ref{sec-A} on p.~\pageref{sec-A}\\
 \item{\bf V0 mass }\ref{sec-TVAV0M} on p.~\pageref{sec-TVAV0M} and
 \ref{sec-QIDV0} on p.~\pageref{sec-QIDV0}
 
 
 \item{\bf width   }see particle table: \ref{sec-PTAC} on p.~\pageref{sec-PTAC}\\
 \item{\bf write}\\
 \subitem events -- see FILO data card: \ref{sec-DCFILO} on p.~\pageref{sec-DCFILO}
 and QWRITE:
 \ref{sec-QWR} on p.~\pageref{sec-QWR}\\
 \subitem on line printer, terminal: Ch. \ref{sec-GS} on p.~\pageref{sec-GS}
 
 
 \item{\bf XCEQAN, XCEQOR, XCEQU  }[sf] check particle name:
 \ref{sec-TVATPN} on p.~\pageref{sec-TVATPN}\\
 \item{\bf XEFO    }[sf] does EFOL exist ? \ref{sec-TVAEFOL} on p.~\pageref{sec-TVAEFOL}\\
 \item{\bf XEID    }[sf] does EIDT exist ? \ref{sec-TVAEIDT} on p.~\pageref{sec-TVAEIDT}\\
 \item{\bf XFRF    }[sf] do FRFT and FRTL exist ? \ref{sec-TVAFRFT} on p.~\pageref{sec-TVAFRFT}\\
 \item{\bf XGETBP  }[v] Is the event--chunk beam position from GET\_BP available?
 \ref{sec-MR} on p.~\pageref{sec-MR}\\
 \item{\bf XHMA    }[sf] does HMAD exist ? \ref{sec-TVAHMAD} on p.~\pageref{sec-TVAHMAD}\\
 \item{\bf XHVTRG  }[v]  detector HV and trigger status: \ref{sec-MHR} on p.~\pageref{sec-MHR}\\
 \item{\bf XIOKLU  }[sf] LCAL  luminosity available for current run ? \ref{sec-MR} on p.~\pageref{sec-MR}\\
 \item{\bf XIOKSI  }[sf] SiCAL luminosity available for current run ? \ref{sec-MR} on p.~\pageref{sec-MR}\\
 \item{\bf XLEPTG  }[sf] track is a Lepton tagged by QSELEP ? : \ref{sec-QSELTL} on p.~\pageref{sec-QSELTL}\\
 \item{\bf XLEPTH  }[sf] MCarlo track is tagged by QTRUTH ? : \ref{sec-QSELTH} on p.~\pageref{sec-QSELTH}\\
 \item{\bf XLOCK   }[sf] track locked?~\ref{sec-QL} on p.~\pageref{sec-QL}\\
 \item{\bf XLUMOK  }see XHVTRG: \ref{sec-MHR} on p.~\pageref{sec-MHR}\\
 \item{\bf XMC     }[sf] MC particle? \ref{sec-TVAFP} on p.~\pageref{sec-TVAFP}\\
 \item{\bf XMCA    }[sf] does MCAD exist? \ref{sec-TVAMCAD} on p.~\pageref{sec-TVAMCAD}\\
 \item{\bf XMCEV   }[v] MC event? \ref{sec-MCE} on p.~\pageref{sec-MCE}\\
 \item{\bf XMINI   }[v] event input MINI? \ref{sec-MCE} on p.~\pageref{sec-MCE}\\
 \item{\bf XNANO   }[v] event input NANO? \ref{sec-MCE} on p.~\pageref{sec-MCE}\\
 \item{\bf XPEQAN,XPEQOR,XPEQU   }[f] test particle name:
 \ref{sec-TVATPN} on p.~\pageref{sec-TVATPN}\\
 \item{\bf XPGAC   }[sf] does PGAC exist ? \ref{sec-TVAPGAC} on p.~\pageref{sec-TVAPGAC}\\
 \item{\bf XSAME   }[sf] tracks based on the same object? \ref{sec-TVATPSO} on p.~\pageref{sec-TVATPSO}\\
 \item{\bf XTEX    }[sf] does TEXS exist? \ref{sec-TVATEXS} on p.~\pageref{sec-TVATEXS}\\
 \item{\bf XVITC,XVTPC, etc. }[v] detector HV status: \ref{sec-MHR} on p.~\pageref{sec-MHR}\\
 \item{\bf XVDEOK }[f] VDET HV: \ref{sec-XVDEOK} on p.~\pageref{sec-XVDEOK}
 
 \item{\bf YCUT }see QJMMCL or QGJMMC\\
 \item{\bf YTOP   }[s] Vertex finding package: \ref{sec-QVFIT} on p.~\pageref{sec-QVFIT}\\
 \item{\bf zero    }track vectors: \ref{sec-QVZ} on p.~\pageref{sec-QVZ}\\
 \item{\bf Z0 }see QZB
