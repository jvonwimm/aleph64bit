\chapter{\label{sec-SED}Definition of Event Directory Classes}
\par
 
\bf{EDIR classes for LEP1 and LEP 1.5 Data and Monte Carlo:}
\par
\begin{verbatim}
 
 Class # 1:  More than 2 Ecal modules with energy(wires) >=2.5 GeV in each
 
 Class # 2:  Hcal energy(pads) + Ecal energy(wires) > 15 GeV
 
 Class # 3:  End cap A and End cap B both with energy(wires) > 2. GeV
             or Barrel with energy > 6 GeV
 
 Class # 4:  Hcal energy(pads) > 3 GeV + HCW(4 planes) * ITC trigger
 
 Class # 5:  1-->7 tracks with >=4 TPC hits, D0 <5 cm and Z0 <20 cm
 
 Class # 6:  >=8 tracks with the same cuts
 
 Class # 7:  LUM A and LUM B, both with E >15 GeV
 
 Class # 8:  LUM A or LUM B, with E >15 GeV
 
 Class # 9:  Muon (HMAD flag) with energy >3 GeV
 
 Class #10:  Electron with momentum >=2 GeV
             electron candidates based on (-3.5>R2 , -3.5<R3<4.0)
 
 Class #11:  ECAL high voltage ON
 
 Class #12:  TPC high voltage ON (Logical or between bits 4 and 15
             of the high voltage word)
 
 Class #13:  ITC  high voltage ON
 
 Class #14:  LCAL high voltage ON
 
 Class #15:  Dilepton candidates
 ---------
                   Selections based on TPC tracks only
 
     The requirements are less strict than those used in the common
     lepton analysis.
 
1. Only tracks with >= 4 TPC hits, |z_0| < 10 cm, |p| > 0.1 GeV/c are used.
   Their numbers are counted separately for |d_0| < 5 cm (including |d_0| <
   2 cm), and for |d_0| < 2 cm.
 
2. a) If there are exactly 2 such tracks with |d_0| < 5 cm, they are declared
      as good tracks, and the selection continues.
   b) If there are between 2 and 8 such tracks with |d_0| < 2 cm, they are
      declared as good tracks, and the selection continues. Any track with
      2 cm < |d_0| < 5 cm is declared as bad.
 
3. The thrust axis is calculated using a routine from JETSET. Then, each of
   the two hemispheres as defined by the thrust axis is required to contain
   at least one good track.
 
4. It is required that at least one track with |d_0| < 2 cm has a momentum
   exceeding 3 GeV/c. IT IS HERE WHERE 2 TRACK EVENTS, BOTH TRACKS HAVING
   2cm < |d_0| < 5 cm, ARE LOST. It seems to me that this was not the ori-
   ginal intention.
 
5. If there are more than 4 good tracks, each of them is required to have an
   opening angle eta with respect to the axis of the corresponding jet
   which fulfills cos(eta) > 0.85.
 
      Events surviving all these requirements are flagged as class 15.
 
 Class #16:  QQbar candidates (based on TPC tracks information)
 ---------
 
                  Selections based on TPC tracks only
 
     Class 16 contains events with :
 
     - At least 5 TPC tracks satisfying the following cuts :
       |D0| < 2 cm
       |Z0| < 10 cm
       TPC coordinates >= 4
       |cos(theta)| < 0.95
     - in addition the total energy of all TPC tracks satisfying
       the cuts above, should have more than 10% of C.M. energy.
 
 Class #17:  QQbar candidates (based on calorimetry information)
 ---------
 
               Selections based on Calorimetry information
 
       Class #17 candidats are:
       - Events satisfying the Total Electromagnetic trigger:
         3 GeV in odd and even wire planes of Barrel or 0.75 GeV
         in odd and even wire of each electromagnetic End cap.
       - They should be in time with beam crossing :
         T0 from Ecal wires used (abs(T0)<120 ns in End caps,
         abs(T0)<100 ns in Barrel)
       - They should satisfy the total Energy cuts :
         Ecal wire energy + Hcal analog energy (matching with
         digital) being defined as total energy should be > 20 GeV
       - Cosmics are rejected using pattern of fired modules :
         2 modules distant by at least 1 should have wire energy > 3 GeV
       - Bhabhas are rejected using fraction of energy deposited in
         the 2 most energetic clusters.
 
 Class #18:  Events in time with beam crossing:
             selection based on T0 information from ECAL wires:
             abs(T0)<120 ns in End caps and abs(T0)<100 ns in Barrel
 
 Class #19:  Muon candidates of all energies:
             based on HMAD or MCAD flags or Mucalo(proba.>.80)
 
 Class #20:  Bhabha candidates (selection based on Ecal only):
             -at least 2 non adjecent modules with E(wires)> 35 GeV each
             -2 Ecal clusters with E(pads)> 35 GeV and |cos(theta)|<0.95
 
 Class #21:   Single photon candidates
 ---------
 
       This class flags events with at least one photon with an
      energy > 1.0 GeV and no charged tracks. To be flagged the
      event must have TPC HT bit ON, no good (i.e. 4 or more
      space points) TPC tracks and the SNG_N_EM trigger bit must
      be set in the Level 1 trigger word. In addition, the event
      must have at least one cluster with pad energy greater than
      1.0 GeV, stack 2 pad energy greater than 0.1 GeV, more than
      0.1 GeV in either of the other two stacks and the theta of
      the cluster must lie in the range 13 to 167 degrees.
      The final requirement is that if such a cluster exists then
      there must be 1.0 GeV or more on the wires for the module
      in which the cluster lies.
 
 Class #22:  Coincidence in Sical A AND B (E >= 20 GeV)
 
 Class #23:  Single arm  in Sical A OR  B (E >= 20 GeV)
 
 Class #24:   Di-Lepton Extended Class 15 candidates
 ---------
 
                   Selections based on TPC and GAMPEK photons
 
     The requirements are less strict than those used in the common
     lepton analysis.
 
1. Only tracks with >= 4 TPC hits, |z_0| < 10 cm, |p| > 0.1 GeV/c are used.
   Their numbers are counted separately for |d_0| < 5 cm (including |d_0| <
   2 cm), and for |d_0| < 2 cm.
 
2. a) If there are exactly 2 such tracks with |d_0| < 5 cm, they are declared
      as good tracks, and the selection continues.
   b) If there are between 2 and 8 such tracks with |d_0| < 2 cm, they are
      declared as good tracks, and the selection continues. Any track with
      2 cm < |d_0| < 5 cm is declared as bad.
 
3. It is required that at least one track with |d_0| < 2 cm or at least
   one GAMPEK photon has a momentum exceeding 2 GeV/c.
 
4. If there are more than 4 good tracks, each of them is required to have an
   opening angle eta with respect to the axis of the corresponding jet
   which fulfills cos(eta) > 0.85.
 
      Events surviving all these requirements are flagged as class 24.
 
 Class #25:  Slow control records
 
 Class #26:  Events to be used for alignement and calibration purposes
             Muon events selected by mean of trigger bits
             Bhabha events selected from two ecal modules with wire
             energies above 30 GeV
 
 Class #27:  Two different possible definitions but never on the same dataset :
     POT     VDET Laser events for VDET alignment / calibration
     MINI    electron selection from QSELEP (Rt>-3, -3>Rl>3)   MINI >= 10 only
 
 Class #28:  Heavy Flavour muon flag , defined on MINIs only
     MINI    muon selection from QSELEP (QMUIDO flag 13 or 14) MINI >= 10 only
 
 Class #29:  Random trigger events
 
 Events which do not satisfy any of the preceeding selections are put in
 class #30
 
\end{verbatim}

\newpage
 
\bf{EDIR classes for LEP 2 Data and Monte Carlo:}
\par
\begin{verbatim}
 
 Class # 1:  >= 1 Ecal cluster with energy(wires) >= 1.5 GeV
             + 1 module with Ewire > 0.5 Gev, |T0| < 200ns.
 
 Class # 2:  Hcal energy(pads) + Ecal energy(wires) > 15 GeV
 
 Class # 3:  Cosmics passing through VDET
 
 Class # 4:  Hcal energy(pads) > 3 GeV + HCW(4 planes) * ITC trigger
 
 Class # 5:  1-->7 tracks with >=4 TPC hits, D0 <5 cm and Z0 <20 cm
 
 Class # 6:  >=8 tracks with the same cuts as in Class 5
 
 Class # 7:  LUM A and LUM B, both with E >30 GeV
 
 Class # 8:  LUM A or LUM B, with E >30 GeV
 
 Class # 9:  2-photon: >= 3 tracks, Ecal/ELEP < 0.5, Echarged/ELEP < 0.4
 
 Class #10:  Spare
 
 Class #11:  Low multiplicity W W A: >= 1 track of each sign
 
 Class #12:  Low multiplicity W W B: all tracks of same sign
 
 Class #13:  Spare
 
 Class #14:  Spare
 
 Class #15:  Dilepton candidates (e+e-, mu+ mu-, tau+ tau-).
 
                   Selections based on TPC tracks only
 
 Class #16:  QQbar candidates (based on TPC tracks information)
 
 Class #17:  QQbar candidates (based on calorimetry information)
 
 Class #18:  Spare
 
 Class #19:  Muon candidates of all energies:
             based on HMAD or MCAD flags or QMUIDO
 
 Class #20:  Bhabha candidates (selection based on Ecal only):
             -at least 2 non adjacent modules with E(wires)> 35 GeV each
             -2 Ecal clusters with E(pads)> 35 GeV and |cos(theta)|<0.95
 
 Class #21:  Single photon candidates
 
 Class #22:  Sical A (E >= 20 Gev) AND Sical B (E >= 20*ELEP/91.2 GeV)
 
 Class #23:  Spare
 
 Class #24:   Di-Lepton Extended Class 15 candidates
 
 Class #25:  Slow control records
 
 Class #26:  Events to be used for alignement and calibration purposes
             Muon events selected by mean of trigger bits
             Bhabha events selected from two ecal modules with wire
             energies above 30 GeV
 
 Class #27:  Two different possible definitions but never on the same dataset :
     POT     VDET Laser events for VDET alignment / calibration
     MINI    electron selection from QSELEP (Rt>-3, -3>Rl>3)   MINI >= 10 only
 
 Class #28:  Heavy Flavour muon flag , defined on MINIs only
     MINI    muon selection from QSELEP (QMUIDO flag 13 or 14) MINI >= 10 only
 
 Class #29:  Random trigger events
 
 Events which do not satisfy any of the preceeding selections are put in
 class #30
 
\end{verbatim}
