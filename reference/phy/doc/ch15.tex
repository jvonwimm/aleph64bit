\chapter{\label{sec-PT} Particle Table}
\par
\section{\label{sec-PTD}Description}
\par
The particle table contains the following particle attributes:
nominal mass, charge, life time, width, and particle $-$ antipart.
relation.
\par
In every ALPHA job, an internal particle table is built which combines
data from the following sources:
\begin{itemize}
\item Data cards described below.
\item The ``standard'' ALEPH particle table stored on the data base.
This table contains all standard model particles (three generations)
which can be produced at LEP energies, plus some exotic particles.
\item The ``MC'' particle table stored in the run record of MC event
files.
This table contains the standard table, and if necessary, extra
particles specific to the MC generator.
\end{itemize}
If particle attribute values from different sources do not agree,
they
are taken from data cards with highest and from the MC table with
lowest priority. The standard printout produced at job termination
indicates where the values come from.
\par
New particles can be defined with the PNEW card (see below), or
by using their names in ALPHA
subroutine calls. If particles are created in subroutine calls,
a warning message is printed.
\par
\section{\label{sec-PTN}Particle name, particle code}
\par
Particles can be specified either by their name (example:
`GAMMA ') or by their integer particle code.
\begin{indentlist}{ 2.50cm}{ 2.75cm}
\indentitem{General rule:}
Only the particle name is relevant.
The integer
code may change from
one job to another; if you wish to use the integer code,
it must be initialized in each job
by calling the function (see \ref{sec-ADT}):{\it
integer = KPART ('part$-$name')}.
\end{indentlist}
\section{\label{sec-PTH}How to spell particle names}
\par
On data cards, every particle name (1 ... 12 characters) has to be
terminated by exactly one blank space.
\par
{\bf Example}
\begin{verbatim}
        PMOD 'PI+ PI- ' 0.14     ! is correct !
        PMOD 'PI+ PI-' 0.14      ! SERIOUS MISTAKE !
\end{verbatim}
\par
In the Fortran program, this extra blank space can be omitted or typed.
\par
Lower case characters are translated into upper case characters.
It would be wise, nevertheless, to use UPPER case characters only.
 
\section{\label{sec-PTDC}Data cards for particle table}
\par
\subsection{\label{sec-PTPMOD}PMOD: Modify particle attributes}
\par
\begin{description}\item[\bf{Format}]
{\it PMOD `part$-$name antipart$-$name ' mass charge life$-$time width}
\end{description}
\begin{indentlist}{ 3.75cm}{ 4.00cm}
\indentitem{Parameters:}
\indentitem{'part$-$name antipart$-$name'}see \ref{sec-DCPMOD}.
The attributes of a particle and its antiparticle are
modified at the same time. If a particle is its own anti$-$
particle, the same name has to be given twice.
\indentitem{mass charge life$-$time width:}
Real numbers (with decimal point).
The charge of the antiparticle is set to $-$charge. If less than
four numbers are given, the remaining particle attributes
are not changed.
\end{indentlist}
\par
{\bf Example:}
\begin{verbatim}
        PMOD 'GAMMA GAMMA ' 0.001
\end{verbatim}
\par sets the photon mass to 1 MeV; the other particle
attributes (charge, lifetime, width) are not changed.
\par
{\bf Mistake:}
\begin{verbatim}
        PMOD 'PI+ PI0 ' .14
\end{verbatim}
\par because pi+ and pi0 are NOT antiparticles of each other.
Once a particle$-$antiparticle relation is established (for
example on the standard table), it can never be changed.
\par If the particle names given on this card are not yet established
in the
table then
\begin{itemize}
\item new table entries are created;
\item a warning is issued;
\item the program execution continues.
\end{itemize}
\subsection{\label{sec-PTPNEW}PNEW: New particles}
\par
\begin{description}\item[\bf{Format
}]
{\it PNEW `part$-$name antipart$-$name ' mass charge life$-$time width}
\end{description}
PNEW has the same function as the PMOD card
(\ref{sec-PTPMOD}) and has the same parameters, except
\begin{itemize}
\item PNEW causes a warning if the particles are already known;
\item PMOD causes a warning if the particles are unknown;
\end{itemize}
program execution continues in either case.
\par
\subsection{\label{sec-PTPTRA}PTRA: Modify particle names in the
MC particle table}
\par
The PTRA card assigns an arbitrary particle name to a
specific MC integer code.
It has to be used, for example, if different MC
data sets with contradictory particle tables are read in one job.
\par
The standard procedure to denote the nature of MC generated particles:
\begin{itemize}
\item Start with the integer code given for each generated particle.
\item Get the corresponding particle name from the MC particle table.
\item This name is relevant inside the ALPHA program.
\end{itemize}
\begin{description}\item[\bf{Format
}]{\it PTRA `part$-$name antipart$-$name' iMCcode iMCanticode}
\end{description}
\begin{indentlist}{ 3.75cm}{ 4.00cm}
\indentitem{Parameters:}
\indentitem{'part$-$name antipart$-$name'}see \ref{sec-PTPMOD}.
the names for the particle and its antiparticle
which have to be used inside the ALPHA program.
\indentitem{iMCcode:}
integer particle code used in the MC generator
(WITHOUT decimal point and NOT included inside the apostrophes.)
\indentitem{iMCanticode:}
integer particle code used by the MC
generator for the corresponding antiparticle.
\end{indentlist}
The routine QCPTRA is equivalent to the PTRA card and can be called
from QUNEWR whenever a new MC particle table is read in.
\section{\label{sec-PTAC}Access to particle properties}
\par
Inside an ALPHA job, particle properties can be obtained by
specifying the particle
either by name (symbols starting with the characters ``QP'' or ``KP'')
or by
the integer code (``QC'' or ``KC''). The particle code has to be set
by calling the function
{\bf ICODE = KPART ('part$-$name')} at least once per job.
For more details, see \ref{sec-ADT}.
\begin{indentlist}{ 5.75cm}{ 6.00cm}
\indentitem{KPART ('part$-$name')}Integer particle code for `part$-$name'
\indentitem{CQPART (intg$-$code)}Particle name (12 characters; trailing
characters
filled with blank spaces)
\indentitem{KPANTI ('part$-$name', IANTI)}
 
If IANTI = 0: integer code for `part$-$name'
If IANTI unequal to  0: integer code for the
antiparticle of `part$-$name'
\indentitem{KCANTI (intg$-$code, IANTI)}...
\indentitem{QPMASS ('part$-$name')}nominal mass
 
\indentitem{QCMASS (intg$-$code)}...
\indentitem{QPCHAR ('part$-$name')}charge
\indentitem{QCCHAR (intg$-$code)}...
\indentitem{QPLIFE ('part$-$name')}life time
\indentitem{QCLIFE (intg$-$code)}...
\indentitem{QPWIDT ('part$-$name')}width
\indentitem{QCWIDT (intg$-$code)}...
\end{indentlist}
\par
To check the particle names of ALPHA ``tracks'', see sections
\ref{sec-TVATPN} and \ref{sec-TVAFP}.
\par
{\bf Warning :} All above functions which have a particle name as input argument are case$-$insensitive .
 
