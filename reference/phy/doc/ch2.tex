\chapter{\label{sec-GS}Getting Started}
\par
Two files must be provided to run an ALPHA job:
\begin{enumerate}
\item A file which contains the Fortran or CVS code for the
user
subroutines (see Ch.
\ref{sec-U}).
\item A card file which defines the input / output
data files, and allows the user to select several optional parameters
(see Ch. \ref{sec-DC}).
\end{enumerate}
The libraries needed to link the program are described in Appendix
\ref{sec-ZB}. To run ALPHA, the following Fortran units
must be assigned:\footnote{Other units used by ALPHA are listed in
Section~\ref{sec-MCU}; Fortran units 90, 91, 92, and 93 are always free for
private output files.}
\begin{indentlist}{ 1.25cm}{ 1.50cm}
\indentitem{unit}
\indentitem{6}Print output file
\indentitem{7}Card file
\indentitem{76}(optional): in an interactive session,
this unit may be assigned to the
terminal. Short messages will be sent to the terminal
and long listings sent to the output file.
\end{indentlist}
A command file, {\bf alpharun}, described in Appendix~\ref{sec-ZB}
 is available on all ALEPH computers 
to make
these file assignments, and to perform the following tasks:
\begin{enumerate}
\item compile and link the Fortran (or CVS)  code;
\item link with all additional needed libraries (BOS, ALEPHLIB, JULIA, MINI);
\item run the program interactively or submit a BATCH job.
\end{enumerate}
On the DEC/VMS  machines, {\bf alpharun} also facilitates the use of a
set of DEC debugger command files which simplify
ALPHA program debugging.
