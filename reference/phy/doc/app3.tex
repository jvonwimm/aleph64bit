\chapter{\label{sec-ZB}Where to find ALPHA at CERN}
 
All the ALPHA code and documentation (the present user's guide) may be found on www in the ALEPH home page,
( http://alephwww.cern.ch/ ):
click on Offline Software, then on ALPHA Source Code Index (for the source code of ALPHA subroutines) or on
ALPHA User's Manual (the present user's guide).
 
{\bf Important warning: It is strongly recommended to use always the current ALPHA production version, as given
automatically by the alpharun facilities (see below). Using old versions may be dangerous: there may be incompatibilities, bugs
or obsolete features which may be difficult to discover at first glance.}  


 
 
\section{\label{sec-ONVAX}ALPHA on ALWS}
 
Since version 122, ALPHA is not maintained any more on type VAX machines. All what follows refers to AXP machines only.
Since the same version, there is no correction file any more: all modified routines are directly put into the binary library.
 
Most files needed to run ALPHA on ALWS are in the PHY directory:
 
ALEPH\$GENERAL:[PHY]
 
\begin{itemize}
 
\item ALPHA source code:

Since August 1996, ALPHA is maintained under CVS. The reference source code of ALPHA subroutines can be found
as CVS files with name XXXXX.F  in subdirectories of:

 ALIBROOT:[000000.PHY.ALPHAvsn]

The relevant subdirectories for subroutines are: PACK, UTIL, QFN, ENFL, QVS, QLEP, QPI0, NANO.

For the sample user routines QUINIT, QUEVNT, QUTERM, the subdirectory is PACK.
 
For include files (QCDE,QMACRO,QDECL,QHAC) the subdirectory is INC.
 

\item Fortran files, if you still want to use them:
\begin{indentlist}{ 2.50cm}{ 2.75cm}

\indentitem{PHY:QUUSER.FOR}Fortran code of the routines
 
QUINIT,QUEVNT,QUTERM
 
(model routines which have to be filled)
\end{indentlist}

\item 
Include files for Fortran:
\begin{indentlist}{ 2.50cm}{ 2.75cm}
\indentitem{PHYINC:QCDE.INC}
\indentitem{PHYINC:QDECL.INC}
\indentitem{PHYINC:QHAC.INC}
\indentitem{PHYINC:QMACRO.INC}
\end{indentlist}
 
\item ALPHA binary libraries:
\begin{indentlist}{ 2.50cm}{ 2.75cm}
\indentitem{PHY:ALPHAvsn.OLB}ALPHA binary library.
\indentitem{PHY:ALPHAvsn\_D.OLB}with /DEBUG option.
\end{indentlist}
 
\item ALPHA card file (sample)
\begin{indentlist}{ 2.50cm}{ 2.75cm}
\indentitem{PHY:ALPHA.CARDS}\hspace*{1in}
\end{indentlist}
 
\item ALPHARUN command file (see Ch. \ref{sec-GS})
\begin{indentlist}{ 2.50cm}{ 2.75cm}
\indentitem{PHY:ALPHARUN.COM}
options stored in ALFPARAM.OPTB (by default)
\end{indentlist}
 
ALPHARUN is invoked directly by typing  alpharun. 
It allows to build  an ALPHA job, either interactive or BATCH, and links
automatically the user's program with all necessary libraries (ALPHA, BOS, ALEPHLIB, JULIA, CERNLIB).

ALPHARUN also facilitates the use of a set of debugger command files which
simplify ALPHA program debugging.
\begin{indentlist}{ 4.75cm}{ 5.00cm}
\indentitem{Examples}
\indentitem{EXAMINE IW(512:515)}Standard VAX debug command;
to be used for all
Fortran variables and arrays.
\indentitem{EVALUATE LMHLEN}LMHLEN is a parameter and NOT a variable;
use EVA instead of EXA.
\indentitem{QP(ITK)}Debugger commands are defined for almost all mnemonic
symbols which have one or more arguments (see Ch.
\ref{sec-M}). In
this context,  ``QP'' is a debugger command which has to be
followed by the same argument(s) (given as numbers or
variable names) as the mnemonic symbol QP in Fortran.
\end{indentlist}
 
\item ALPHA documentation and News
\begin{indentlist}{ 2.50cm}{ 2.75cm}
\indentitem{ALEPH\$GENERAL:$[$PHY.DOC$]$ALGUIDE.PS}PostScript file to print the present document
\indentitem{ALEPH\$GENERAL:$[$PHY.DOC$]$ALGUIDE.TEX}\LaTeX source for this document
\indentitem{PHY:ALPHAvsn.NEWS}Description of changes in ALPHA version vsn
\indentitem{PHY:ALPHA.VERSIONS}Short description of the various versions of ALPHA and related ALEPHLIB / MINI versions
\end{indentlist}
\end{itemize}
 
 
\section{\label{sec-ONUNIX}ALPHA on SHIFT,CSF,ALPHA OSF}
 
 
\par
 Reminder : 

 1 - since ALPHA version 122, all corrections to the current default version are put directly in the ALPHA library;
   there is therefore  no source or compiled correction file.

 2 - since ALPHA version 124 (May 1999) the MINI code is fully integrated within ALPHA, both for reading and for writing MINIs.

\par
To run ALPHA on UNIX machines,
it is {\bf strongly} recommended to use\\ 
{\bf alpharun}, which is described below (Sec \ref{sec-alphar}).
  It allows either to run ALPHA interactively or to send it in Batch on a
given machine, as did previouly SFALPHA which is now obsolete.
 
The following files refer to the current ALPHA version.
 
\begin{itemize}
\item ALPHA source code:
 
Since August 1996, ALPHA is maintained under CVS. The reference source code of ALPHA subroutines can be found
as CVS files with name xxxxx.F (Caution! xxxxx must be lower case!) in subdirectories of:
 
\$ALROOT/alphavsn (e.g. for ALPHA 124 vsn=124 and the directory is \$ALROOT/alpha124)
 
The relevant subdirectories are:
 inc,news,pack,util,qfn,enfl,qvs,qlep,qpi0,mini,nano.
 

\item ALPHA include files: 

The reference source code of ALPHA include files is available as CVS files with name yyyy.h (Caution! yyyy must be lower case!)
 in \$ALROOT/alphavsn/inc  (for ALPHA 124, vsn=124)
 
 
To find e.g. the path to the source code of QMACRO for ALPHA 124:
 
   cd \$ALROOT/alpha124

   find . -name ``qmacro.h" 
 
\item Binary library:
 
\begin{indentlist}{ 2.50cm}{ 2.75cm}
\indentitem{\$ALEPH/phy/libalpha.a}

Since May 1999 (ALPHA124) this library contains all MINI routines.

There is no ALPHA debug library any more on UNIX machines (too big!). 

\end{indentlist} 

\item ALPHA card file (sample)
\begin{indentlist}{ 2.50cm}{ 2.75cm}
\indentitem{\$ALEPH/phy/alpha.cards}\hspace*{1in}
\end{indentlist}
 
\item alpharun (see Ch. \ref{sec-GS})
\begin{indentlist}{ 2.50cm}{ 2.75cm}
\indentitem{alpharun}shell file on all UNIX machines
\par
 See more details on alpharun just below. 
\end{indentlist}

\item ALPHA documentation and News
\begin{indentlist}{ 2.50cm}{ 2.75cm}
\indentitem{\$REFERENCE/phy/doc/alguide.ps}PostScript file to print the present document
\indentitem{\$REFERENCE/phy/doc/alguide.tex}\LaTeX source for this document, and all related files
\indentitem{\$ALEPH/phy/alpha.news}Description of changes in the current version of ALPHA
\indentitem{\$ALEPH/phy/README}Short description of the various versions of ALPHA and related ALEPHLIB / MINI versions
\end{indentlist}
\end{itemize}
 
\section{\label{sec-alphar}alpharun: run ALPHA on UNIX platforms:}
\par
{\bf alpharun} is a facility intended for people who want to run ALPHA on UNIX machines.
 It compiles your program with the adequate compiler options,
generates automatically the appropriate links to all needed libraries
 (BOS, ALEPHLIB, JULIA, MINI etc.), and builds the executable through a makefile which you can edit afterwards.
Then {\bf alpharun} executes your ALPHA job either interactively or in batch. 

\par
To execute {\bf alpharun}, you have just to prepare your Fortran analysis program and your Cards file, then to
type {\bf alpharun} and
answer the questions.

If you choose to run in Batch, your job will be sent to the machine you have chosen; the OUTPUT logfile is returned to the original
computer in your working directory. If you create an output EPIO file through a FILO data card,
 it will be also returned to the 
original computer if you have put the DISP parameter in the FILO card (see description of the FILO card, 
sec \ref{sec-DCFILO} on p. ~\pageref{sec-DCFILO}). 
Similarly, if you create an output .HIS histogram file through a HIST data card,
 it will be also returned to the
original computer if you have put the DISP parameter in the HIST card (see description of the HIST card,
sec \ref{sec-HISTW} on p. ~\pageref{sec-HISTW}).

{\bf Warning!} If you write yourself a private output file, you must take care yourself of getting it back from the  
computer where your job has run in Batch.
\par
It is also possible to run {\bf alpharun} with options, which are described in the {\bf man alpharun} and are summarised below:
\par
\begin{verbatim}
           alpharun [-F filename]
           alpharun [-f fortran_file] [-c c_file] [-C card_file] [-g] [-j]
            [-v version] [-o obj_file] [-s file_opt] [-p] [-r]
            [-I include_path] [-m machine] [-t time_limit] <binary_name>
 
    Where:
         -help             get this (h)elp
         -F <string>       expected (F)ilename which alpharun options
         -f <string>       expected (f)ortran file  (my_fortran.f)
         -c <string>       expected (C) file  (my_cfile.c)
         -o <string>       expected (o)bject file  (my_objfile.o)
         -C <string>       expected  C ard file  (my_cards.cards)
         -g                for debug options to compile
         -j                (j)ulia is used
         -I <string>       expected (I)nclude path
         -b                to run in (b)atch
         -m <string>       expected machine to submit your job to
         -t <number>       (t)ime limit of your batch job in seconds
         -p                (P)VM is used
         -v <number>       (v)ersion of alpha
         -s <string>       expected name of the file to (s)ave the options
         -r                (r)uns the binary after creating it
           <binary_name>   expected name for the binary
\end{verbatim}
 
