\chapter{\label{sec-INTRO}Introduction}
\pagenumbering{arabic}
\par
The ALEPH Physics Analysis package ALPHA is intended to simplify Fortran
programs for physics analysis.
Although all ALEPH data types can be processed with ALPHA,
the program is designed primarily
for analysis of JULIA output files (POTs), and of  DSTs and MINIs.
All event input/output is done by ALPHA $-$ the user has to provide
only the name(s) of the input/output
data set(s).
ALPHA also provides easy access to physical variables
(e.g., momentum, energy), so the user can write physics analysis programs
without detailed knowledge of
the ALEPH data structure (tabular BOS banks).
An extensive set of utility routines (e.g., kinematics, event shape,
secondary vertex finding, b-tagging, MINI-writing, etc.)
is available as part of the ALPHA package.

\par
The program structure (Appendix
\ref{sec-PSTRUC}) is extremely simple. Three Fortran routines are
normally
supplied by the user: job initialization, event processing, and job
termination (see Ch. \ref{sec-U}).
Reconstructed objects (tracks, vertices,
cal. objects) can be accessed with
simple DO loops. For Monte Carlo generated events, the MC
``truth'' information is accessible
in the same way as reconstructed tracks and vertices (see Ch. \ref{sec-A}).
\par
This document describes all features of the ALPHA program.
For first$-$time users, the important parts to
read are Ch.~\ref{sec-GS} (getting started),
Ch.~\ref{sec-U} (user routines), Ch.~\ref{sec-DC} (event input),
Ch.~\ref{sec-A} (loops over tracks), and Ch.~\ref{sec-TVA}
(track attributes).
\par
ALPHA has grown up with time , several general utility packages coming from UPHY
being incorporated as they become more popular and tested :
 
\begin{indentlist}{ 3.50cm}{ 3.75cm}
\indentitem{ALPHA 114} in September 1992 : QMUIDO, interfaces to YTOP
\indentitem{ALPHA 115} in May       1993 : ENFLW, SLUMOK, PGPC photons
\indentitem{ALPHA 116} in December  1993 : GET$\_$BP, QFNDIP, QIPBTAG
\indentitem{ALPHA 117} in May       1994 : QPI0DO, QVSRCH, NANO reading 
\indentitem{ALPHA 118} in October   1994 : QTRUTH
\indentitem{ALPHA 120} in March     1995 : QSELEP, MINI 10 reading, PGAC photons
\indentitem{ALPHA 121} in October   1995 : NANO writing, new beam spot
\indentitem{ALPHA 122} in May       1996 : new matching, CVS code manager 
\indentitem{ALPHA 123} in October   1998 : QPCORR, LEP BOMs, new QELEP          
\indentitem{ALPHA 124} in May       1999 : MINI writing, new QIPBTAG, QDDX     
 
\end{indentlist}
 
 
 
