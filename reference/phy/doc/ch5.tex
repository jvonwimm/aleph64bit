\chapter{\label{sec-HIST}Creating Histograms and Ntuples}
\par
The standard histogram package in ALPHA is HBOOK4.
If you don't want to use
HBOOK, the only system routines which
are called automatically and
which refer to HBOOK are the histogram initialization / termination
routines QUIHIS and QUTHIS (\ref{sec-QUIH} and
\ref{sec-QUTH}).
Some utility routines which simplify calls to HBOOK
routines or provide additional protection
against deleting existing histograms are described below.
Histogram output is directed by entries in the card file,
and is described in section
\ref{sec-HISTO}.
 
\section{\label{sec-HBOOK}Booking and Filling Histograms/Ntuples}
\par
All of these routines call standard HBOOK4 routines.
\par
\subsection{\label{sec-QB1}Book a 1$-$dimensional histogram}
\par
\fbox{CALL QBOOK1 (ID, CHTITL, NX, XMI, XMA, VMX)}
\par
\par The arguments
are the same as for CALL HBOOK1 (...):
\begin{indentlist}{ 2.50cm}{ 2.75cm}
\indentitem{Input arguments:}
\indentitem{ID}histogram ID number -- nonzero integer
\indentitem{CHTITL}histogram title -- character variable up to 80 characters
\indentitem{NX}number of bins
\indentitem{XMI}lower edge of lowest bin
\indentitem{XMA}upper edge of highest bin
\indentitem{VMX}normally set equal to 0.-- see HBOOK manual for details.
\end{indentlist}
 
\par
HBOOK1 always deletes an existing histogram and creates a new one.
To make it possible to update existing histograms
(see \ref{sec-DCHW}),
QBOOK1 creates a new histogram only if it does not yet exist.
An existing histogram remains unchanged. Therefore, whenever you want
to
update histogram files, use QBOOK1 instead of HBOOK1.
For new histograms, QBOOK1 and HBOOK1 are identical.
 
\subsection{\label{sec-QB2}Book a 2$-$dimensional histogram}
\par
\fbox{CALL QBOOK2 (ID, CHTITL, NX, XMI, XMA, NY, YMI, YMA, VMX)}
\par
\par
QBOOK2 includes the same features as QBOOK1.
The arguments are the same as for CALL HBOOK2 (...):
\begin{indentlist}{ 2.50cm}{ 2.75cm}
\indentitem{Input arguments:}
\indentitem{ID}histogram ID number -- nonzero integer
\indentitem{CHTITL}histogram title -- character variable up to 80 characters
\indentitem{NX}number of bins in X
\indentitem{XMI}lower edge of lowest X bin
\indentitem{XMA}upper edge of highest X bin
\indentitem{NY}number of bins in Y
\indentitem{YMI}lower edge of lowest Y bin
\indentitem{YMA}upper edge of highest Y bin
\indentitem{VMX}normally set equal to 0.-- see HBOOK manual for details.
\end{indentlist}
 
 
\subsection{\label{sec-QBP}Book a Profile histogram}
\par
\fbox{CALL QBPROF (ID, CHTITL, NX, XMI, XMA, YMI, YMA, CHOPT)}
\par
\par
QBPROF includes the same features as QBOOK2.
The arguments are the same as for CALL HBPROF (...): -- see HBOOK manual .
 
\subsection{\label{sec-QBN}Book an Ntuple}
\par
\fbox{CALL QBOOKN (ID, CHTITL, NVAR, TAGS)}
\par
\par The arguments are NOT the same as for CALL HBOOKN (...):
\begin{indentlist}{ 2.50cm}{ 2.75cm}
\indentitem{Input arguments:}
\indentitem{ID}Ntuple ID number -- nonzero integer
\indentitem{CHTITL}Ntuple title -- character variable up to 80 characters
\indentitem{NVAR}number of variables
\indentitem{TAGS}name of character array of dimension NVAR containing
names for variables to be stored.
\end{indentlist}
\par
\begin{verbatim}
CALL QBOOKN (ID,CHTITL,NVAR,TAGS)
\end{verbatim}
corresponds to :
\begin{verbatim}
CALL HBOOKN (ID,CHTITL,NVAR,'ALPHA',1024,TAGS).
\end{verbatim}
'ALPHA' is the ZEBRA directory name referring to the file
given on the HIST card (\ref{sec-HISTW}).
See \ref{sec-QB1} (QBOOK1 vs. HBOOK1) :
Existing Ntuples will not be
overwritten (see \ref{sec-QB1}).
 
\subsection{\label{sec-QBR}Book an Ntuple with run, event number}
\par
\fbox{CALL QBOOKR (ID, CHTITL, NVAR, TAGS)}
\par
\par The arguments are the same as for CALL QBOOKN (...).
QBOOKR books a Ntuple with NVAR+2 variables. The two additional variables
contain the run and event number.
TAGS consists of NVAR array elements. Two tags KRUN and KEVT are
appended automatically.
 
\subsection{\label{sec-QBFR}Fill Ntuple plus run, event number}
\par
\par
\fbox{CALL QHFR (ID, A)}
\par Fills the Ntuple ID with the array A and with run and event number.
The arguments are the same as for HFN (ID, A).
KRUN and KEVT are filled as variables NVAR+1 and NVAR+2 (see QBOOKR).
 
\subsection{\label{sec-QBFN}Fill Ntuple with many variables}
\par
\fbox{CALL QHFN (ID, A1, A2, A3, ..., An)}
\par
\par Fills the Ntuple ID with the variables A1 ... An (n $<$ 80).
CALL QHFN (ID, A1, A2) corresponds to
\begin{verbatim}
      DIMENSION A(80)
      A(1) = A1
      A(2) = A2
      CALL HFN (ID, A)
\end{verbatim}
 
\subsection{\label{sec-QBFRN}Fill Ntuple with many variables plus
run, event number}
\par
\fbox{CALL QHFNR (ID, A1, A2, A3, ..., An)}
\par
\par Fills the Ntuple ID with the variables A1 ... An (n $<$ 80; see
QHFN) and with
run / event number as variables n+1 and n+2 (see QHFR).
\par
\newpage
\subsection{\label{sec-HIEXAM}Sample ALPHA program to book and fill
histogram, Ntuple}
\par
The following example books and fills a histogram and Ntuple. See
Chapters~\ref{sec-A} and~\ref{sec-TVA}
for explanations of the ALPHA variables used.
\begin{verbatim}
      SUBROUTINE QUINIT
      CHARACTER*4 TAGS(2)
      DATA TAGS/'ECHG','NTRK'/
C--- Book histogram to store momentum distribution for all charged
C--- tracks.
      CALL QBOOK1(1,'Momentum',100,0.,50.,0.)
C--- Book Ntuple to store charged energy and number of charged tracks
C--- per event.
      CALL QBOOKN(1000,'Event parameters',2,TAGS)
      END
      SUBROUTINE QUEVNT (QT,KT,QV,KV)
C-----------------------------------------------------------------------
      INCLUDE 'PHYINC:QCDE.INC'              !VAX
      DIMENSION QT(KCQVEC,1), KT(KCQVEC,1), QV(KCQVRT,1), KV(KCQVRT,1)
      INCLUDE 'PHYINC:QMACRO.INC'            !VAX
C-----------------------------------------------------------------------
      IF(KNCHT.EQ.0)RETURN
      ECHRG=0.
C
C--- sum energy; histogram track momentum
C
      DO 20 IT=KFCHT,KLCHT
        ECHRG=ECHRG+QE(IT)
        CALL HF1(1,QP(IT),1.)
   20 CONTINUE
      CALL QHFN(1000,ECHRG,FLOAT(KNCHT))
      END
\end{verbatim}
 
\subsection{\label{sec-HILIMI}Limitations to the ALPHA histogram facilities}
\par
The default length (100000 words) of the /PAWC/ buffer of HBOOK may be too short for
special uses. It may be increased by modifying  the parameter LQPAW in the QUIHIS routine.
\par
There are no ALPHA facilities to book of fill column-wise Ntuples.
\par
If the user wants to use Ntuples with more than 80 variables,
 he has to use the standard HBOOK calls. The output
file must be open using HROPEN and written using HROUT/HREND.  See the
HBOOK manual for more details.
\par
If the user wants
write very large Ntuple files which exceed the default maximum length
of 64 Mbytes , he has to use the NREC and RECL parameters in the HIST data card ,
as described in \ref{sec-HISTW} .
 
 
 
 
\section{\label{sec-HISTO}Histogram output $-$ the ALPHA cards file}
 
\subsection{\label{sec-HISTW}HIST: Write histogram file}
\par
Unless the NOPH card is included in the card file (see below),
1$-$ and 2$-$dimensional histograms are printed out to the log
file in the program termination phase (i.e., after return from QUTERM;
see \ref{sec-UT} and \ref{sec-QUTH}).
 
The HIST data card is necessary for writing histograms and Ntuples
to a histogram file which can be used in a
subsequent interactive session (PAW). Users who want to write files larger
than the maximum default size of 64 Mbytes must give appropriate NREC and RECL parameters
in their HIST Card , see below  .
 
\begin{description}\item[\bf{Format
}]{\it HIST `data$-$set$-$name $\mid$ parameters'}\end{description}
\begin{indentlist}{ 4.00cm}{ 4.25cm}
\indentitem{data set name}see \ref{sec-DCFT}.
\indentitem{Default file format}HIS
\indentitem{parameters (optional) :}
 
\indentitem{UPDA}
Update existing histograms. Can be used deliberately if a
previous job terminated due to time limit etc. but ...
\begin{description}\item[\bf{CAUTION}]with this option, the old histogram
file will be
overwritten (even on DEC).\end{description}
\indentitem{NOOV}
Overwrite protection (see \ref{sec-DCFILO}).
Cannot be used with UPDA.
\begin{indentlist}{ 2.50cm}{ 2.75cm}
\indentitem{On DEC/VMS :}Unnecessary.
\indentitem{On UNIX :}Strongly recommended.
\end{indentlist}
\indentitem{DISP}
                  Returns the resulting Histogram File to the original computer from which the ALPHA job was submitted.
                  Must be used if you are running ALPHA in batch on a UNIX platform using {\bf alpharun},see 
                  App.~\ref{sec-ZB} on p.~\pageref{sec-ZB}.
\indentitem{NREC nmax}
                  Sets the maximum number of records in the file to 'nmax'.
                  If parameter NREC is missing, or if 'nmax' is missing
                  or not in the range 100-100000, the default of 16000
                  records is used.
                  The NREC parameter cannot be used with the UPDA parameter.
 
\indentitem{RECL nwords}
                  Specifies the record length in the file to be 'nwords'.
                  If parameter RECL is missing, or if 'nwords' is missing
                  or not in the range 1-999999, the default record length
                  is used.
                  The default record length is 1024 if the UPDA parameter
                  is missing. If the UPDA parameter is present, the default
                  record length is 0 to let ZEBRA try to determine the
                  record length from the file itself (if ZEBRA fails to
                  determine the record length, ALPHA retries with RECL=1024;
                  if the file can still not be opened the user must specify
                  explicitely the record length that was used to create
                  the file).
 
\end{indentlist}
 
Only one histogram file can be specified using the HIST card.
If you need several
output files, use the standard HBOOK4 input / output routines and
book
Ntuples with different ZEBRA directory names. The directory name used
by ALPHA is `//ALPHA'.
 
\subsection{\label{sec-HISTP}NOPH: Histogram Printing}
\par
\par Including the NOPH card in the card file will suppress the printing
of HBOOK histograms to the terminal or log file; histograms will still
be
written to a direct access file if the HIST card is used.
\begin{indentlist}{ 2.50cm}{ 2.75cm}
\indentitem{Format :}
{\it NOPH}\end{indentlist}
 
 
\subsection{\label{sec-HISTT}HTIT: General histogram title}
\par
 
The HTIT card assigns a general title to all histograms;
it corresponds to the HBOOK routine HTITLE.
\begin{indentlist}{ 2.50cm}{ 2.75cm}
\indentitem{Format :}
{\it HTIT `This is the general title'}\end{indentlist}
 
