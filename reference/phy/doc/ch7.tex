 \chapter{\label{sec-A}ALPHA ``Tracks'' and ``Vertices''}
\par
Before QUEVNT is called for each event, ALPHA fills its own data structure
with information from the event.  Each ``tracklike'' object (eg.,
tracks, calorimeter objects, energy flow objects, etc.) is
assigned a unique number. (A ``tracklike'' object is any object which
can be
described with a 4 vector.)
This ALPHA ``track'' number is equal to the JULIA
``track'' number + a constant. Unique ALPHA numbers are also assigned
to
vertices (reconstructed vertices and Monte Carlo vertices).
The constant is introduced in order to obtain a unique numbering scheme
for all
species of ``tracks'' or vertices (in JULIA and GALEPH,
different species start with
the number 1).  In the description below, ITK always refers to an
ALPHA ``track''
number and IVX to an ALPHA vertex number.
\par The properties of the tracks and vertices are found using functions
which
refer to the ALPHA ``track'' and vertex numbers. For example, the
energy of
ALPHA ``track'' ITK is QE(ITK). The properties available for each
tracklike
object and each vertex are described in sections \ref{sec-TVATA}
and \ref{sec-TVAVA}, respectively.
\par
In the following sections, several methods for determining ALPHA track
and vertex numbers are described. All of these methods can be nested.
Functions which give simple access to relationships between different
types of objects are also described.
 
\section{\label{sec-AL}Access by Fortran DO loops}
\par
In ALPHA, fortran DO loops can be used to loop over most types of
objects.
For each type of object, three variables are defined:
KFxxx, KLxxx, KNxxx. xxx represents the type of object. The last
letter of the variables is either T (tracklike) or V (vertex).
DO loops must be made from KFxxx to KLxxx; KNxxx is the number of
objects
of type xxx.
\par
For example, the following three lines will make a histogram of
the momentum spectrum of charged particles.
\begin{verbatim}
              DO 10 ITK = KFCHT, KLCHT
                CALL HF1 (47,QP(ITK),1.)
         10   CONTINUE
\end{verbatim}
\begin{indentlist}{ 3.25cm}{ 3.50cm}
\indentitem{KFCHT,KLCHT}
number of first (last) charged track.
\indentitem{ITK}loop index = ALPHA track number
\indentitem{QP(ITK)}momentum of track ITK (see \ref{sec-TVABA})
\end{indentlist}
The number of charged tracks is given by the variable
KNCHT.
KNCHT stands for KLCHT $-$ KFCHT + 1.
Therefore, if KNCHT = 0, KLCHT = KFCHT $-$ 1.
\par The objects which can be accessed with these DO loops are
listed in the following two sections.
 
\subsection{\label{sec-altrack}ALPHA ``TRACKS'' :}
\par
ALPHA ``tracks'' are stored in the internal ALPHA bank QVEC.

\begin{indentlist}{ 2.00cm}{ 2.25cm}
\indentitem{Charged Tracks: KFCHT, KLCHT, KNCHT}

If the FRF0 card is present in the ALPHA cards file, the NR=0
version of the FRFT bank (track parameters determined without vertex
detector coordinates) will be used.  Otherwise, the NR=2 version
of FRFT (TPC + ITC + VDET tracks) will be used.  Only FRFT NR=2
tracks are available on the MiniDST.
 
\indentitem{Calorimeter Objects: KFCOT, KLCOT, KNCOT}

For physics analysis with optimised energy resolution,
 one should better use the Energy Flow 
objects (see below).

Calorimeter objects can
be any of the following:
\begin{itemize}
\item ECAL objects with no associated HCAL object.\footnote{For
ECAL wire energies, see~\ref{sec-ECWI}.}
\item HCAL objects with no associated ECAL object.
\item Composite cal.
objects consisting of at least one ECAL and HCAL object
associated to each other. See \ref{sec-ALC}
for getting access to the contributing ECAL and HCAL objects separately.
See the end of Section \ref{sec-AR} for a more detailed description
of composite cal. objects in ALPHA.
\end{itemize}
Calorimeter objects can be further divided into:
\begin{indentlist}{ 2.00cm}{ 2.25cm}
\indentitem{ISolated cal. objects:  KFlST, KLIST, KNIST}
Cal objects with NO associated charged track.
\indentitem{ASsociated cal. objects: KFAST, KLAST, KNAST}
Cal objects with one or more associated charged track.
\end{indentlist}
 
\indentitem{``REconstructed'' objects: KFRET, KLRET, KNRET}

``REconstructed'' objects are:
\begin{itemize}
\item Charged tracks;
\item Calorimeter objects (see above)
which are NOT associated to charged tracks (ISolated cal. objects).
\end{itemize}
 
\indentitem{Reconstructed Standard V0s from YV0V: KFV0T, KLV0T, KNV0T}

See Section
\ref{sec-AMM} for comments on the daughters of standard V0s.
 
\indentitem{Tracks from standard V0 Vertices: KFDCT, KLDCT, KNDCT}

Charged tracks outgoing from reconstructed DeCay vertices. The momenta
for these tracks are calculated relative to the secondary
vertex position.

 Caution: this section includes only the daughter tracks from
reconstructed standard V0s from YV0V, and not from the ``Long V0s'' 
coming from the new tracking with JULIA version 302 and after (see below).

\indentitem{Reconstructed Long V0s: KFLVT, KLLVT}

 ``Long V0s'' (coming from the new tracking with JULIA version 302 and after). 
One gets the loop limits for them through the following subroutine call:

 CALL QLV0T(KFLVT,KLLVT)                               
                                           
 
\indentitem{Energy Flow objects: KFEFT, KLEFT, KNEFT}

This section includes selected charged tracks
and ECAL and HCAL clusters remaining after subtracting
track energies. These objects
may also be accessed with their particle name `EFLW' using
the functions KPDIR and KFOLLO (described in
\ref{sec-AD}).
This section is not
filled unless the EFLW card is included in the card file
(see Ch. \ref{sec-EF}).

Energy Flow objects made with the ENFLW package (see Chapter 11)
should be used for all analyses which need an optimised calorimeter 
energy resolution.
\indentitem{NEutral Calorimeter Objects: KFNET, KLNET, KNNET}

Neutral objects derived from the PCPA bank.  These objects
may also be accessed with their particle name `NEOB' using
the functions KPDIR and KFOLLO (described in
\ref{sec-AD}). To be used only by experts.
 
\indentitem{Photons from GAmpec: KFGAT, KLGAT, KNGAT}

These objects
may also be accessed with their particle name `GAMP' using
the functions KPDIR and KFOLLO (described in
\ref{sec-AD}).
 
\indentitem{Jets from EJET: KFJET, KLJET, KNJET}

Jets based on EFLW objects using QJMMCL with
YCUT = 0.003.
These objects
may also be accessed with their particle name `EJET' using
the functions KPDIR and KFOLLO (described in
\ref{sec-AD}). They may be used as input for jet finding
with a higher YCUT (see~\ref{sec-QJSIM} and~\ref{sec-EFLWJ}).
 
\indentitem{Monte Carlo particles (``truth'') KFMCT, KLMCT, KNMCT}
\end{indentlist}
 
\subsection{\label{sec-alvert}ALPHA ``VERTICES'' :}
\par

         ALPHA  ``vertices'' are stored in the ALPHA internal bank QVRT.

\begin{indentlist}{ 2.00cm}{ 2.25cm}


\indentitem{Monte Carlo vertices (``truth''): KFMCV, KLMCV, KNMCV}
            These vertices come either from the  KINGAL generator or from the
 interactions in GALEPH.  

\indentitem{REconstructed Vertices: KFREV, KLREV, KNREV}

For datasets coming from the ``old'' tracking (JULIA version 285 and before) 
 this includes:  the Main
vertex (which is the first vertex, KFREV, when it exists),
Standard V0s from the YV0V
package, and Secondary vertices fitted by any of the KVFITx routines, see sec.
\ref{sec-QFIT} on p.~\pageref{sec-QFIT}.

It is possible that there is no Main vertex in an event.

For datasets coming from the ``new'' tracking of 1997
(JULIA version 302 and after):
 in addition to the previous vertices one may get three
 other vertex categories described below.
 To access them  by DO loops, one has
 to call specific subroutines which give the corresponding vertex numbers:

\begin{itemize}
\item ``Long V0s'': 

                     CALL QLV0V(KFLV0,KLLV0)
 
                     KFLV0, KLLV0 are the first and last ``Long V0'' vertex 
                     numbers, if any.

\item Nuclear Interactions:

                     CALL QNUCL(KFNIV,KLNIV)

                     KFNIV, KLNIV are the first and last nuclear interaction
                     vertex numbers, if any


\item Kink Vertices:

                     CALL QKINKV(KFKIV,KLKIV)

                     KFKIV, KLKIV are the first and last kink                  
                     vertex numbers, if any
  


\end{itemize}

    Please take note of the above subroutine calls: they are mandatory to get
the loop limits for these new vertex categories.

\end{indentlist}
 
\section{\label{sec-ALC}Loops over ECAL and HCAL objects}
\par
If ECAL and HCAL objects are topologically associated to each other,
the loops described above give access to composite calorimeter objects
rather than to each contributing ECAL and HCAL object separately.
It is also possible to get access to all ECAL and HCAL objects, regardless
of
whether or not they are associated to other reconstructed objects.
(The
loops described below are equivalent to looping through the PECO and
PHCO
banks.)
 
The following statements perform a loop over all ECAL objects;
see \ref{sec-AD}.
(DO loops cannot be used because the objects are
not stored in consecutive locations.)
\begin{verbatim}
            IOBJ = KPDIR ('ECAL', KRECO)
         10 IF (IOBJ .EQ. 0)  GO TO 999
      C...      Analysis of the ECAL object IOBJ ...
            IOBJ = KFOLLO (IOBJ)
            GO TO 10
\end{verbatim}
(The functions KPDIR and KFOLLO are described in
\ref{sec-AD}.)
The corresponding loop for HCAL objects is:
\begin{verbatim}
            IOBJ = KPDIR ('HCAL', KRECO)
         10 etc ...
\end{verbatim}
\section{\label{sec-AR}Relationships between objects in different
subdetectors}
\par
The JULIA program provides relationships between objects reconstructed
in the various detector components if they are topologically
associated to each other. These relations are available in ALPHA and
can be used for charged tracks, ECAL objects, HCAL objects, and
composite calorimeter
objects. (Below, IOBJ is any ALPHA ``track'' number referring
to a charged track, cal.
object, ECAL object, or HCAL object.)
\begin{indentlist}{ 3.50cm}{ 3.75cm}
\indentitem{KNCHGD (IOBJ)}
Number of charged tracks associated to IOBJ.
\indentitem{KCHGD (IOBJ, N)}
The Nth charged track associated to IOBJ.
\end{indentlist}
%\newpage
For example,
\begin{verbatim}
            IOBJ = ... any calorimeter object ...
            DO 10 N = 1, KNCHGD (IOBJ)
            ICHGD = KCHGD (IOBJ, N)
      C ...   analysis of a charged track ICHGD associated to IOBJ ...
         10 CONTINUE
\end{verbatim}
Note: If IOBJ in the example above is a charged track itself, then
KNCHGD (IOBJ) is 1 and KCHGD (IOBJ,1) gives IOBJ.
Similarly:
\begin{indentlist}{ 3.50cm}{ 3.75cm}
\indentitem{KNECAL (IOBJ)}
Number of ECAL objects associated to IOBJ.
\indentitem{KECAL (IOBJ, N)}
The Nth ECAL object associated to IOBJ.
 
\indentitem{KNHCAL (IOBJ)}
Number of HCAL objects associated to IOBJ.
\indentitem{KHCAL (IOBJ, N)}
The Nth HCAL object associated to IOBJ.
\end{indentlist}
\par
The relation from a composite calorimeter object ICOMP to each
of its contributing ECAL and HCAL object is provided by
the relations described above: KECAL (ICOMP,N) and KHCAL (ICOMP,N).
In addition, the composite object is treated as ``mother'' of the
contributing ECAL and HCAL objects, so the mother$-$daughter or
daughter$-$mother relation described in
\ref{sec-AM} can be used for all calorimeter objects.
\par Note that the composite calorimeter
objects in ALPHA are not identical to those in the PCRL bank.  ALPHA
composite calorimeter objects include at most one HCAL object, while the
PCRL
objects may include many HCAL objects.  ALPHA starts with each HCAL
object and adds the ECAL objects that are associated to it.
If an ECAL object is
associated to more than one HCAL object, its energy is divided equally
among the HCAL objects.
\section{\label{sec-AD}Direct access to particles}
\par
\subsection{\label{sec-ADI}Particle name and class}
\par
In addition to the loops described above, it is possible to access
particles by
their name.
In many cases, this method is faster and the code is easier to read
than the
standard loops
described in \ref{sec-AL}.
Two quantities must be specified:
\begin{itemize}
\item The particle name (example: `E+' or `GAMMA'); see
\ref{sec-PTD}.
\item The object (= track = particle) class which distinguishes between
reconstructed tracks, the
Monte$-$Carlo truth, and any Lorentz frame derived from one of them:
\begin{itemize}
\item Class KRECO: Reconstructed objects read from the event input
file
and everything derived from
them except Lorentz boosted objects.
\item Class KMONTE: Monte$-$Carlo truth.
\item Each Lorentz frame is considered as its own class
(see \ref{sec-QT}).
These classes are denoted by the number of the object which defines
the Lorentz rest frame.
\end{itemize}
\end{itemize}
\par KRECO and KMONTE are
available
everywhere as integer Fortran parameters. Their actual values are
$-$1 and
$-$2, respectively. Positive integers denote Lorentz frames. Integers
less than $-$2 can be used to create your own particle classes
(see KVSAVC in \ref{sec-QVSC}).
\par
The particle name of MC particles is specified in the MC particle
table
(see \ref{sec-PTD}).
Reconstructed objects have the names `CHARGED', `ECAL',
`HCAL', `CALOBJ', `EFLW', 'NEOB', and 'GAMP'
 for charged tracks, ECAL objects, HCAL objects,
unspecified (\eg, composite) calorimeter objects,
energy flow objects, neutral calorimeter objects, and GAMPEC photons,
respectively.
The functions KVSAVE, KVSAVC, and KIDSAV (see~\ref{sec-QVSTID})
can be used to create new tracks with
a name.
A list of standard particle names is given in App. \ref{sec-STP}.
New particle names can be introduced by using them in ALPHA
subroutine calls or by specifying them on data cards (see
\ref{sec-PTPNEW}).
 
\subsection{\label{sec-ADE}Example: Loop over all MC generated positrons}
\par
\begin{verbatim}
              ITK = KPDIR ('E+', KMONTE)
         10   IF (ITK .EQ. 0)  GO TO 90
        C     ... e+ analysis ...
              ITK = KFOLLO (ITK)
              GO TO 10
         90   CONTINUE ...
\end{verbatim}
\begin{indentlist}{ 3.25cm}{ 3.50cm}
\indentitem{KPDIR ('particle$-$name', ICLASS)}
{\it 'particle$-$name':} Character string (1 to 12 characters).
{\it ICLASS:} Track class (see \ref{sec-ADI}): KRECO
or KMONTE or a track number
ITKRST if ITKRST has been used before to define the rest frame for
a
Lorentz boost (see \ref{sec-QT}).
\indentitem{KFOLLO (ITK)}
The following particle with the same particle name in the
same class.
\indentitem{Remarks:}
 
The term ``FOLLOwing'' refers to some arbitrary ordering.
Lower case characters in particle names are translated to
upper case. It is safest, however, to use only
upper case characters with ALPHA.
\end{indentlist}
\subsection{\label{sec-ADT}Particle name versus integer particle
code -- time consumption}
\par
Using character particle names in function calls makes the code easier
to read, but it implies a
lookup in a table. Although the lookup is fast,
in nested loops it may be
desirable to save this time. Consequently, some (not all)
functions are provided in two versions: one which expects the particle
name as an argument and another which expects the corresponding
integer particle code and thus saves the lookup time. The second version
is denoted by a ``C'' (= ``Code'') as the 2nd character of the function
name.
\par Using integer particle codes,
the example given in section \ref{sec-ADE} becomes:
\begin{verbatim}
        C ... somewhere in the job or subroutine initialization:
              IP = KPART ('E+')
        C ...
              ITK = KCDIR (IP, KMONTE)
         10   IF (ITK .EQ. 0)  GO TO 90
        C     ... analysis of the e+ ...
              ITK = KFOLLO (ITK)
              GO TO 10
 
         90   CONTINUE ...
\end{verbatim}
\begin{indentlist}{ 5.00cm}{ 5.25cm}
\indentitem{IP = KPART('name')}
must be called before IP is used. The particle name is the basic
reference to a particle. The integer
code may change from one job to another.
\indentitem{KCDIR (IP, ICLASS)}
First particle with the given particle code in class ICLASS.
\end{indentlist}
 
\subsection{\label{sec-ADA}Loops over a particle and its antiparticle}
\par
The particle table contains the relation between particles and
antiparticles, so loops over particles (or systems of particles)
and their corresponding (systems of) antiparticles can be
performed easily.
\par
{\bf Example:} Loop over MC $-$ generated e+ and e$-$:
\begin{verbatim}
              DO 90 IANTI = 0,1
              ITK = KPDIRA ('E+', KMONTE, IANTI)
         10   IF (ITK .EQ. 0)  GO TO 90
        C     ... analysis of the e+ or e- ...
              ITK = KFOLLO (ITK)
              GO TO 10
 
         90   CONTINUE ...
\end{verbatim}
\begin{indentlist}{ 5.00cm}{ 5.25cm}
\indentitem{KPDIRA ('particle$-$name', ICLASS, IANTI)}
If IANTI=0, KPDIRA returns the first particle with the
given name in the class ICLASS.
If IANTI is not equal to 0, the first corresponding antiparticle
is given.
\end{indentlist}
To use the integer particle code (see \ref{sec-ADT}), replace
\begin{verbatim}
KPDIRA ('E+', KMONTE, IANTI) with KCDIRA (IP, KMONTE, IANTI).
\end{verbatim}
 
\subsection{\label{sec-ADS}Analysis of particle systems: Examples}
\par
Systems of particles can be analyzed by nesting loops with
KPDIR and KPDIRA. The two examples given below illustrate
cases in which care must be taken to avoid multiple counting of the
same
particle combinations.
 
\subsubsection{\underline {Combinations of the same particles: $\pi$$\sp{+}$
$\pi$$\sp{+}$}}
\par
\begin{verbatim}
       C---First select pion candidates
              DO 5 ITK=KFCHT,KLCHT
              IF(condition to select pions) THEN
                ISAVE=KIDSAV(ITK,'PI+')
              ENDIF
            5 CONTINUE
       C---Loop over selected pions.
              IPIONE = KPDIR ('PI+', KRECO)
           10 IF (IPIONE .NE. 0)  THEN
                IPITWO = KFOLLO (IPIONE)                 <--- important
           20   IF (IPITWO .NE. 0)  THEN
                  ... analysis of the pi+ pi+ system ...
                  IPITWO = KFOLLO (IPITWO)
                  GO TO 20
                ENDIF
              IPIONE = KFOLLO (IPIONE)
              GO TO 10
              ENDIF
\end{verbatim}
The 2nd $\pi$$\sp{+}$ (IPITWO) has to be initialized with KFOLLO and
NOT
with KPDIR. See section \ref{sec-QVSTID} for the use of KIDSAV.
\subsubsection{\underline{$\Delta$$\sp{+}$$\sp{+}$
$\rightarrow$ p $\pi$$\sp{+}$}}
\par
Proton and pion candidates must be selected and saved with KVSAVE
or KIDSAV before this code is reached (see \ref{sec-QVST}).
\begin{verbatim}
              IPROT = KPDIR ('P', KRECO)
           10 IF (IPROT .NE. 0)  THEN
                IPIPLU = KPDIR ('PI+', KRECO)
           20   IF (IPIPLU .NE. 0)  THEN
                IF (.NOT.XSAME(IPROT,IPIPLU))  THEN    <--- important
       C         ... analysis of the p pi+ system ...
                ENDIF
                IPIPLU = KFOLLO (IPIPLU)
                GO TO 20
              ENDIF
              IPROT = KFOLLO (IPROT)
              GO TO 10
            ENDIF
\end{verbatim}
 
The logical function XSAME (see \ref{sec-TVATPSO}) tests whether the
two
contributing particles are based on
different reconstructed objects or simply on different mass
hypotheses of the same reconstructed object.
 
\section{\label{sec-AM}Mother $-$ daughter relationships}
\par
\subsection{\label{sec-AMM}Mother to daughters}
\par
The connection
from a mother to its daughters is available for MC particles
and for composite particles established by the QVxxxx
routines described in \ref{sec-QVA}.
\begin{verbatim}
              IMOTH = ... (track number of a mother particle)
              DO 10 I = 1, KNDAU (IMOTH)
                IDAUGH = KDAU (IMOTH,I)
                CALL HFILL (47, QP(IDAUGH))
         10   CONTINUE
\end{verbatim}
\begin{indentlist}{ 3.00cm}{ 3.25cm}
\indentitem{KNDAU (ITK)}
number of daughters for track ITK
= 0 if no daughter exists
\indentitem{KDAU (ITK,I)}
track number of Ith daughter
\end{indentlist}
{\bf Note for standard V0s from YV0V:} The daughters of a standard V0
 (section V0T) are
stored in the DCT section (see \ref{sec-AL}).
These tracks are copies of tracks in the
CHT section, but their momenta are recalculated relative
to the secondary vertex position.
The function KCHT (see \ref{sec-KCHT})
returns the CHT track number corresponding to
a track in the DCT section.
\par

{\bf Example:}
\begin{verbatim}
         DO 10 IV0=KFV0T,KLV0T
c---  First daughter of V0 (in DCT section)
       I1DCT=KDAU(IV0,1)
C --- Corresponding track in CHT section.
       I1CHT=KCHT(I1DCT)
    10 CONTINUE
\end{verbatim}

The above note is {\bf not valid} for the ``Long V0s'' from the new tracking
(vertices in the range KFLV0,KLLV0).
\par
\subsection{\label{sec-AMD}Daughter to mother(s)}
\par
The connection from a daughter to its mother(s) is available for MC
particles and for daughters of ``saved'' composite particles (see
KVSAVE
in \ref{sec-QVST}).
The QVADDx routines (\ref{sec-QVA}) and
the jet / event topology routines \ref{sec-QJ})
do NOT set up this relation.
\begin{verbatim}
              IDAUGH = ... (track number of a daughter particle)
              DO 10 I = 1, KNMOTH (IDAUGH)
                IMOTH = KMOTH (IDAUGH,I)
                CALL HFILL (47, QP(IMOTH))
         10   CONTINUE
\end{verbatim}
\begin{indentlist}{ 3.50cm}{ 3.75cm}
\indentitem{KNMOTH (ITK)}
Number of mothers of track ITK. Note that MC particles
as read in from the event input file have no or one mother.
\indentitem{KMOTH (ITK,I)}
Track number of the Ith mother.
\end{indentlist}
\section{\label{sec-AS}Access to the ``same'' object}
\par
The ``same'' object means:
\begin{itemize}
\item any copy of an object;
\item for reconstructed tracks, the ``same'' object with different
mass or
vertex hypothesis;
\item The ``same'' object boosted into any Lorentz frame.
\end{itemize}
\subsection{Loops over copies of the ``same'' object using KSAME}
Example:
\begin{verbatim}
          ITKSAM = KSAME (ITK)
       10 IF (ITKSAM .EQ. ITK)  GO TO 90
    C       ... analysis of the same object, e.g.: search for the object
    C           in a specific Lorentz frame ITKRST (see >):
          IF (KCLASS (ITKSAM) .EQ. ITKRST)  THEN
    C       ...
          ENDIF
          ITKSAM = KSAME (ITKSAM)
          GO TO 10
       90 CONTINUE ...
\end{verbatim}
\begin{indentlist}{ 2.50cm}{ 2.75cm}
\indentitem{Remarks:}
This loop is terminated if it arrives at the original track.
KSAME never returns 0.
The same particle can be boosted several times into the same
Lorentz frame provided that the boosts are performed with
different mass or other hypotheses (see
\ref{sec-QTL}); if you start with the original track ITK,
the most recently boosted hypothesis is reached first.
\end{indentlist}
\par
\subsection{\label{sec-KCHT}Find original copy of a charged track}
For copies of charged tracks, the function KCHT returns the original
track number in the CHT section.
\begin{indentlist}{ 3.50cm}{ 3.75cm}
\indentitem{KCHT (ITK)}
If $
KFCHT\leq ITK\leq KLCHT
$,
{\bf KCHT (ITK)} is equal to ITK. Otherwise (i.e., ITK is a
copy of a track in the CHT section),
{\bf KCHT (ITK)}
equals the corresponding track number in the CHT section.
This function can be used only for charged tracks; for other objects,
use KSAME.
\end{indentlist}
\section{\label{sec-AX}Match reconstructed tracks and MC truth}
\par
The relation between reconstructed and MC particles is not necessarily
one$-$to$-$one. Therefore, a loop has to be constructed:
\begin{verbatim}
        ITK1 = ... (any given MC or reconstructed track number)
        DO 10 I = 1, KNMTCH(ITK1)
          IF (KSMTCH (ITK1,I) .LE. (min. required shared hits))  GO TO 10
          ITK2 = KMTCH (ITK1,I)
  C       ...
     10 CONTINUE
\end{verbatim}
If ITK1 is a reconstructed track then ITK2 is a matching MC track.
If ITK1 is a MC track then ITK2 is a matching reconstructed track.
\begin{indentlist}{ 3.25cm}{ 3.50cm}
\indentitem{KNMTCH (ITK)}Number of matching candidates for track ITK.
\indentitem{KMTCH (ITK,I)}Track number of Ith matching particle.
\indentitem{KSMTCH (ITK,I)}Number of shared hits between MC and reconstructed
track.
\indentitem{KBESTM (ITK)}Best  match to track ITK.

\end{indentlist}
\newpage
{\bf Remarks:}
\begin{itemize}
\item The match is performed on the basis of shared hits in the
TPC and ITC.
\item A much better matching is provided for charged tracks
since ALPHA 122 by a set of new 
routines:
 MCMATCH, JULMATCH, VDHMATCH which are described in Section \ref{sec-OAMCMAT}
on p.~\pageref{sec-OAMCMAT}.
\item A specific matching is available for Energy Flow objects (charged and
neutral);  
it is described in Section \ref{sec-EFLWMA} on p.~\pageref{sec-EFLWMA}.

\end{itemize}
\section{\label{sec-AV}Track $-$ vertex relationships}
\par
\begin{indentlist}{ 4.75cm}{ 5.00cm}
\indentitem{IVX = KORIV (ITK)}origin vertex of a track
\indentitem{IVX = KENDV (ITK)}end vertex of a track
\indentitem{ITK = KVINCP (IVX)}particle incoming to vertex IVX
 
\end{indentlist}
To find the tracks outgoing from a vertex, the following
loop must be performed:
\begin{verbatim}
              IVX = ... (vertex number; defined before)
              DO 10 I = 1, KVNDAU (IVX)
                ITK = KVDAU (IVX,I)
                CALL HFILL (47, QP(ITK))
         10   CONTINUE
\end{verbatim}
\begin{indentlist}{ 3.25cm}{ 3.50cm}
\indentitem{KVNDAU (IVX)}number of outgoing tracks
\indentitem{KVDAU (IVX,I)}track number of Ith outgoing track
\end{indentlist}
