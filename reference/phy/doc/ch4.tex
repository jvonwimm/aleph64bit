\chapter{\label{sec-DC}Data Cards}
\par
\par In this chapter, the ALPHA data cards are described.
The cards file is used to control input and output for ALPHA, and
is
used to control many ALPHA features.
For completeness, all
ALPHA cards are listed in this chapter; some cards are described in
more detail
in other chapters.
 
 
The following rules should be followed for all entries in the card
file.
\begin{enumerate}
\item Start the text of your cards in column 1.
\item Use only upper case characters unless the lower case characters
are significant.
\item Except for FILI cards (\ref{sec-DCFILI}),
data cards can be given in any order.
\item The ENDQ card must be the last entry in the card file.
\end{enumerate}
 
Data cards may also be used to enter your
own data into the program. If your cards are given in standard BOS
format, their
contents will be available as standard BOS
banks. For example, if the card {\bf CUTS 4 3.7}
appears in the ALPHA card file, the following Fortran may be used to
get access to the values:
\begin{verbatim}
    ICUTS=IW(NAMIND('CUTS'))
    IF(ICUTS.NE.0)THEN
       ICUT1=IW(ICUTS+1)
       RCUT1=RW(ICUTS+2)
    ENDIF
\end{verbatim}
 
\section{\label{sec-DCIO}Input/Output}
\par
\subsection{\label{sec-DCFT}ALEPH file types}
\par
\par There are several ALEPH file types:
\begin{indentlist}{ 2.50cm}{ 2.75cm}
\indentitem{NATIVE}machine$-$dependent input/output (RAW data are written in VAX NATIVE format)
\indentitem{EPIO}machine$-$independent input/output (all official ALEPH datasets other than RAW data)
\indentitem{EDIR}event directories (for fast access to data using the CLAS data card)
\indentitem{DAF}direct access files (e.g., ADBSCONS database)
 
\indentitem{CARDS}card image files (e.g., ALPHA data cards)
\indentitem{HIS}histogram files (machine$-$dependent HBOOK format)
\indentitem{EXCH}histogram files (machine$-$independent HBOOK format)
\end{indentlist}
\par The ALEPH file type cannot be recognized automatically. The file type
should be given as the extension part of the data set name.
 
{\bf Examples:}
 
\noindent On DEC/VMS computers:\begin{verbatim} MYFILE.EPIO\end{verbatim}
\noindent On UNIX:\begin{verbatim} myfile.epio\end{verbatim}
 
\par ALPHA uses the data set name to determine the format.
For file names which do not follow this convention,
see the following section.
\par
\subsection{\label{sec-DCFILI}FILI: Input datasets}
\par
\par
\begin{description}\item[\bf{Format}]{\it FILI `data$-$set$-$name
$\mid$ parameters'}\end{description}
 
Any number of FILI cards may be given $-$ the data sets are read in
the
order the cards are given. Different file formats (e.g., NATIVE, EPIO)
and data from POT, DST,
and MINI can be processed in the same job.

The program SCANBOOK should be
used to create FILI cards with the correct syntax.
\subsubsection{\underline{How to specify file or EDIR names on FILI cards: Examples}}
\par
{\bf Disk files:}
\begin{verbatim}
            DEC / VMS                          UNIX
 
1  FILI 'HADR.NATIVE'                 native files not recommended
2  FILI 'HADR.EPIO | EPIO'            FILI '{cwd}/hadr.epio'
3  FILI 'HADR.DATA | EPIO'            FILI '{cwd}/hadr.data | EPIO'
4  FILI 'AL$EDIR:AB1234.EDIR'         FILI '/aleph/edir/ab1234.edir'
 
 
\end{verbatim}
\newpage
Explanation:
\begin{enumerate}
\item Complete specification. `` .NATIVE'' defines the file
format.
\item ``$\mid$ EPIO'' can be omitted here because the format is
already
specified in the data set name. The vertical bar separates the file
name from the parameters.
\item ``DATA'' is non$-$standard and not recommended. In such a case,
the format must be given as a parameter: `` $\mid$ EPIO''. Please not that this EPIO parameter must be UPPER CASE,
even on UNIX machines.
\end{enumerate}
\par
\subsubsection{\underline{How to specify tapes (cartridges, DLTs, Redwoods,..)}}
\par
The same format can be used for DEC/VMS and UNIX:
\begin{verbatim}
FILI 'ALDATA | EPIO | CART AS2010.25.AL -s nnn'
\end{verbatim}
Here, ALDATA is the data set name, AS2010 is the cartridge VID,
25 is the file sequence number of the dataset on the cartridge, and AL denotes an ANSI labelled tape.

At CERN and in some Homelab centers, the corresponding tape will not be read directly by ALPHA: the system will
first stage the specified file (or several files in one go) on disk; then ALPHA will read this disk file.

\par
At CERN only the additional parameter  -s nnn  must be
added at the end of a FILI CARD to ask for the staging of a file with size nnn greater
than 590 Mbytes. This is needed for large files which otherwise will be truncated by the stager, resulting in a BOS reading
error.
 
\par
 
Don't forget that the best solution to have access to cartridges or DLTs, either real data or
Monte Carlo data  with or without EDIRs, is to use the FILI CARDS
generated by the SCANBOOK interactive facility.
SCANBOOK generates the FILI cards with the correct syntax for the
datasets you want to analyse. Moreover if needed they will be generated
with the exact -s nnn parameter corresponding to the real size of each dataset.
\par
\subsubsection{\underline{stagelist: query on staged datasets on CERN UNIX computers}}

An interactive facilty called {\bf stagelist} is available on CERN UNIX computers to know which datasets are
presently staged. This may be very useful to make  tests, in order to avoid the staging time which may be very long.

At most 3 arguments may be provided to {\bf stagelist} to select datasets, e.g.:

\begin{indentlist}{ 4.50cm}{ 4.75cm}
\indentitem{stagelist MINI 1994 tau}
 will give the list of staged Monte Carlo MINIs with 1994 geometry containing tau lepton simulations
\indentitem{stagelist POT  1996 DA}
 will give the list of staged POTs of 1996 real data.
\end{indentlist}


\par
\subsection{\label{sec-DCRS}{Run / event selection}}
\par
\par The following cards may be used to select particular
runs or events for analysis. These data cards are standard from BOS, none of them being specific to ALPHA.
\begin{indentlist}{ 4.50cm}{ 4.75cm}
\indentitem{SEVT 15 2 4 6 8 $-$11}
Select EVenTs 2,4,6,8,9,10,11 of run 15
The 1st number is a run number, the following ones are event numbers.
Negative numbers define a range of events. It is possible to include
several SEVT cards in a card file, but
only one SEVT card can be
given for each run. The SEVT card, as well as the SRUN card described
below, will work if the input files are ordered to have increasing
run/event numbers. If the input files are not in sequential order,
the
selection cards
will work correctly only if the NSEQ card (see below) is
included in the card file.
 
\indentitem{SRUN 2 $-$4 6 8 $-$10}
Select RUNs 2,3,4,6,8,9,10.
See note under SEVT on sequential order of runs.
\indentitem{IRUN 1 5 7 11 $-$9999999}
Ignore RUNs 1,5,7,11,12,13,14,...,9999999
 
\indentitem{NEVT 5 $-$7}
Select the 5th, 6th, and 7th events (in the order they are stored
on
the input file regardless of their run / event numbers).
 
\indentitem{NEVT 3}
Select the 1st, 2nd, and 3rd events.
More than two numbers are not allowed on this card.
Be careful that this card selects a total number of events READ by BOS on the input stream
if there is no CLAS card ( see below ) , and the number of events SELECTED if there is a CLAS nn card .
\indentitem{NSEQ}
This card must be included to use the selection functions described
above with files that do not have run/event numbers in increasing
order.
 
\indentitem{NWRT 1000}
This card allows to write only the first 1000 selected events on output
tape . More that 1 number is not allowed on this card .
\end{indentlist}

Remark: There is no standard card to skip a single event (no ``IEVT" data card). 
Skipping an event must be done either through a SEVT card or through 
a modification of the QMRDSB subroutine (reading routine of ALPHA).

\subsection{\label{sec-DCFILO}FILO : Output files}
\par
\par Event output is controlled by the FILO card and by the subroutine
QWRITE (see
\ref{sec-QWR}). The data set name and options are given on the FILO
card. Calling
QWRITE writes the current event to the output file. The COPY card
(see~\ref{sec-DCCOPY}) may also be used to write events to a file.
If a FILO card is given,
all run records will be written out by default (see ALLR, NORU, and SELR
below).
\begin{description}\item[\bf{Format:
}]{\it FILO `data$-$set$-$name $\mid$ parameters'}\end{description}
\begin{indentlist}{ 3.75cm}{ 4.00cm}
\indentitem{data set name}same as on FILI cards; see examples in
\ref{sec-DCFILI}.
\begin{indentlist}{ 2.50cm}{ 2.75cm}
\indentitem{File format}NATIVE, EPIO, or EDIR\end{indentlist}
\indentitem{parameters:} (optional)

\indentitem{label}
if the output is on cartridge or DLT, will be the magnetic label for this tape. Usually set as ALDATA for POTs,DSTs,MINIs
and to KINDATA for KINGAl outputs.
\indentitem{ALLR}
write all run records to the output file (Default when writing output POTs/DSTs).
Must be used
if you intend to read the output file/EDIR with the QFND option
for real data of 1993 and after (QFNDIP needs special banks from the
run header).
\indentitem{NORU}
write no run records to the output file.
\indentitem{SELR}
write run records as soon as the first event record
corresponding to it is written (default when writing a MINI).
 It can be used if few events
are selected from a large data sample; without this option,
the output file may consist mainly of run records.
With SELR, only run records which
are followed by event records are written.
DO NOT use this option if you write an output EDIR !
\indentitem{SREC}
write all ``special'' records to the output file. Without this
card, all records which are neither event nor run records will
not be written.
\indentitem{NOOV}
simple$-$minded protection against involuntarily overwriting
data sets. If this parameter is given AND the output data set
already exists, the program will stop. Note that problems with
overwriting do not arise on the VAX.
 
\indentitem{DISP}
The DISPose option ensures that your EPIO output file will be sent back to the UNIX platform
from which you have sent your ALPHA batch job.
 
\indentitem{Examples:}
{\it FILO `abc.native $\mid$ SELR $\mid$ NATIVE $\mid$ NOOV'}
The 2nd ``NATIVE'' is redundant; see \ref{sec-DCFILI}.
 
{\it FILO `mydata.epio $\mid$ EPIO $\mid$ DISP'}
The output EPIO file will be sent back to the computer where you run alpharun.

{\it FILO `ALDATA $\mid$ EPIO $\mid$ CART AT1234.45.AL'}
The output EPIO file will be written on file 45 of the DLT tape AT1234 with ANSI label ALDATA.
This is valid at CERN only. Please see with your system administrator if this syntax for tapes/DLTs works on your computer, if
you are working in a Homelab.
\end{indentlist}
\par More than one FILO card is not accepted. If you want to write
on
several output units simultaneously, use the standard BOS routines.
\par The output event type (POT, DST, MINI $-$ see
\ref{sec-DCFT})
is the same as the input event type unless the MINI card
is given (see below). Event directories can be created
from any input event type (see \ref{sec-DCEVD}).
\par
%\subsubsection{\underline{I/O with Filedefs or Assign statements}}
%For special applications, input and output files can be defined
%using filedefs or assign statements. As stated in Section
%\ref{sec-MCU},
%unit 20 is used for input and unit 50 for event output.
%The FILI and FILO cards must be used to define the file
%format.  For example, to read and write EPIO files using
%assign statements
%on the VAX, the card file must include the following lines:
%\begin{verbatim}
%     FILI ' | EPIO'
%     FILO ' | EPIO'.
%\end{verbatim}
%The corresponding assign statements are
%\begin{verbatim}
%     ASS FILEIN.EPIO  FOR020
%     ASS FILEOUT.EPIO FOR050.
%\end{verbatim}
\par
\subsubsection{\underline{MINI:  Select Mini-DST for output file}}
\par
If the MINI card is given,
the output file specified with the FILO card will be written
in Mini-DST format; see Appendix \ref{sec-miniapp} and the
Mini-DST User's Guide for details. The user must put a CALL QWRITE statement  (\ref{sec-QWR} on p.~\pageref{sec-QWR})
in his code to write out the events. To be used only for private productions.

Beware: the banks written by default on output are the LEP2-type ones (see Appendix \ref{sec-miniapp}).

\par
\subsubsection{\underline{MINP:  Write Mini-DST with official content}}
\par
If the MINP card is given,
the output file specified with the FILO card will be written
in the same Mini-DST format as official ALEPH MINI productions. All important ALPHA packages 
used to build the MINI are automatically initialised with the right constants (especially ENFLW, QIPBTAG and QSELEP).
The event writing is done for ALL events, the user must NOT have a CALL QWRITE in his code,
otherwise the events will be written twice.


Official ALEPH MINIs are done using this MINP data card and reading POTs, not DSTs, in order to have the 
informations needed to build the pad dE/dx information with the good calibration.
For that purpose, the MINP card sets automatically the unpacking (see \ref{sec-DCUNPK} on p.~\pageref{sec-DCUNPK})
 to UNPK 'AL SO' .

With the MINP card, the programs recognises which kind of input is used and writes the
right kind of MINI for LEP1 or LEP2 data/ Monte Carlo (see Appendix \ref{sec-miniapp}).

\par
\subsubsection{\underline{COMP: Data compression}}
\par
Integer numbers are written in
compressed format by default. The data card
\par
\begin{indentlist}{ 3.75cm}{ 4.00cm}
\indentitem{COMP `NONE'}
suppresses the compression.
\end{indentlist}
\par
\subsubsection{\underline{NWRT: Number of events to write out}}
\par
\par
\begin{indentlist}{ 3.75cm}{ 4.00cm}
\indentitem{NWRT 15}
Set maximum number of events to be written on the output file to 15.
\end{indentlist}
 
\subsection{\label{sec-DCEVD}Event Directories}
\par
Event directories make it possible to read ALEPH
data files in direct access mode.
\par
\subsubsection{\underline{Creating Event Directories}}
There are two ways to create an event directory with ALPHA.
\begin{itemize}
\item One can specify EDIR as a file type in the FILO card:
\begin{verbatim}
       FILO 'TEST.EDIR'
\end{verbatim}
The event directory can be created by using the COPY data card, or
by
calling QWRITE from the user program.
\item It is also possible to create the event directory at the same
time as an EPIO output file. The required FILO card is
\begin{verbatim}
       FILO 'TEST.EPIO | WITH TEST.EDIR '.
\end{verbatim}
\end{itemize}
With either of the above options, it is also possible to set the
30 bit classification word stored for each event in the event directory.
For each bit which is to be set, the user must call the routine
QWCLAS (see \ref{sec-QWCLAS}):
\begin{verbatim}
       CALL QWCLAS(IBIT)    IBIT = 1, 30
\end{verbatim}
If three bits are to be set, QWCLAS has to be
called three times.
Note that a call to QWCLAS simply turns on a single
bit while leaving other bits unchanged.  The intial classification
word is the one read from the input file;  therefore,
the classification word must be zeroed by calling QWCLAS with
IBIT=0 before storing your own values.
If QWCLAS is not called, the classification
word will be set equal to that on the input file.
\par
\subsubsection{\underline{Reading data with event directories}}
The event directory must be
specified in the FILI card:
\begin{verbatim}
       FILI 'TEST.EDIR'
\end{verbatim}
All of the run / event selection cards (Sec. \ref{sec-DCRS})
can be used with event directories.
If the CLAS card (described below)
is given in the card file, only events with certain
classification words will be read from the input file.
\par
\subsubsection{\underline{CLAS: Select events with certain classification word}}
\par
\begin{indentlist}{ 2.50cm}{ 2.75cm}
\indentitem{Format}{\it CLAS ibit1, ibit2, ... , ibitn}
read events with bit ibit1 and/or ibit2 etc. $= 1$\end{indentlist}
\par
It is also possible to make more complicated selections
based on the event classification word by
supplying a new version of the routine BSELEC.  This routine
should be extracted from the BOS77 library and modified. The
default version of BSELEC, shown below, checks to see if
a MASK has been supplied with the CLAS card.  If so, it checks
to see if the event classification word IWORD and MASK have any
bits in common.  Events are read in only if BSELEC is .TRUE.
The line KCLASW=IWORD should not be changed;  this line allows access
to the event directory classification word inside of ALPHA
(\eg, inside QUEVNT).
\begin{verbatim}
      LOGICAL FUNCTION BSELEC (IWORD,MASK)
      INCLUDE 'PHYINC:QCDE.INC'
      BSELEC = .TRUE.
      IF (MASK.NE.0 .AND. IAND(MASK,IWORD).EQ.0) BSELEC = .FALSE.
      KCLASW=IWORD
      END
\end{verbatim}
\par
\subsection{\label{sec-DCFDBA}FDBA: Select an ADBSCONS database}
\par
In the absence of any FDBA card, the officially released ALEPH database (with name adbscons.daf)
 is used by default in ALPHA.
\par 
The FDBA card directs ALPHA to use a database which is not the default one.
This may be useful to use the database under test and not yet officially released, or to use
an old version.
\begin{description}
\item[\bf{On VAX/VMS:}]{\it FDBA 'dbase:name.daf'}
\item[\bf{On UNIX:}]{\it FDBA '\$ALEPH/dbase/name.daf'}
\end{description}
\par If name = adbstest, the current test database under developement will be selected.
\par To get an old version: give explicitely its name , e.g.  adbs217.
\par {\bf Important remark:}
 
In order to analyze 1989/1990 data, it is {\bf mandatory} to
 ask for a special database with name = adbs8990.
\par
\subsection{\label{sec-DCCOPY}COPY: Copying events}
\par
\par The COPY card directs ALPHA to copy events using the data cards
described above (i.e., FILI, FILO, SEVT, SRUN, IRUN, NEVT, NWRT).
\begin{description}\item[\bf{Format}]{\it COPY}  (no parameters)
\end{description}
\par All ALPHA features except data card handling and event input
/ output
are switched off. User routines are never called. Most data cards
not referring to event input / output are ignored.
Therefore, if the COPY card is used,
any ALPHA program (Fortran
code or load module) can serve as a simple copy job which
digests the standard ALPHA data cards.
\par
\section{\label{sec-DCPC}ALPHA Process cards}
\par
\par To reduce processing time, certain categories of objects can
be
excluded from ALPHA analysis (i.e. the ALPHA variables will not be filled).
\begin{indentlist}{ 2.50cm}{ 2.75cm}
\indentitem{NOMC}no Monte Carlo ``truth''
\indentitem{NOCH}no CHarged tracks (also excludes V0s)
\indentitem{NOEM}no Error Matrix for charged tracks . Be careful : if you use this card ,
 you cannot use any fitting routine ( KVFITx ) , and you cannot use QIPBTAG .
\indentitem{NOV0}no V0s . Do no use it if you call QIPBTAG !
\indentitem{NOCO}no CalOrimeters
\indentitem{NOPC}no NEutral OBjects (from PCPA)
\indentitem{NOGA}no GAmpec objects (from EGPC or PGPC or PGAC ) . Do not use it if you need QPI0DO !
\indentitem{NONE}no ALPHA banks will be filled. This option is useful
if you
don't want to use any of ALPHA's ``track''
and vertex sections, but you want
to use ALPHA to do all of the I/O and bank unpacking.
\end{indentlist}
\par
\section{\label{sec-DCUNPK}UNPK: POT / DST / MINI unpacking}
\par
\par Unpacking of POT / DST / MINI banks is performed automatically. To save
time, coordinates and some other banks are normally NOT unpacked.
The
default unpack options can be modified with the UNPK card.
\begin{description}\item[\bf{Format
}]{\it UNPK `ab cd ef ... '}\end{description}
The two$-$character options have the following meanings:
\begin{indentlist}{ 2.50cm}{ 2.75cm}
\indentitem{AL }all banks are unpacked but no
coordinate sorting is done
\indentitem{VD } VDET coordinates
\indentitem{IT } ITC coordinates
\indentitem{TP } TPC coordinates
\indentitem{TE } dE/dx
\indentitem{EC } ECAL (electron id. )
\indentitem{HC } HCAL
\indentitem{MU } Muons
\indentitem{FI } track fits
\indentitem{SO } to sort coordinates in phi to redo pattern recognition
\indentitem{CR } cal. object relationship banks
\indentitem{' ` } NO unpacking
\end{indentlist}
The default options correspond to the card:
{\it UNPK 'TE EC HC MU FI '}; TPC and ITC coordinates are not
unpacked by default.
 
\par
\section{\label{sec-DCREAD}READ: Input from different card files}
\par
\par The READ card allows input cards to be read from different
card
files.
\begin{description}\item[\bf{Format
}]{\it READ `card$-$file$-$name'}\end{description}
The default file format is CARDS.
\par
Card files may contain any number of READ cards. Files specified on
a
READ card may contain other READ cards. Recursive READ cards (file
Z
contains a READ `Y' card, and file Y a READ `Z' card) are not allowed.
\par
Note that each card file specified with a READ card must end with
an ENDQ card.
\par
\section{\label{sec-DCDEBU}DEBU: Debug output}
\par
There are two debug levels:
\begin{indentlist}{ 2.50cm}{ 2.75cm}
\indentitem{DEBU 0}minimum debug output (no BOS summary and no particle
table printed).
\indentitem{DEBU 1}(default) Print BOS statistics and particle table
summary
at the end of the job.
Print a message for each step in the ALPHA initialization and
termination.
\end{indentlist}
The debug level is
available as the variable KDEBUG.
\section{\label{sec-DCTIME}TIME: Job time control}
\par
\par
\begin{indentlist}{ 2.50cm}{ 2.75cm}
\indentitem{TIME 5}
causes program termination (CALL QMTERM) if less than 5 seconds
are available.
 
\indentitem{Remarks}
If no TIME card is given, 65 seconds is assumed by default on UNIX systems , 15 seconds on
other computers .
The number on the TIME card must be given WITHOUT a decimal
point. In ALPHA, it is converted to a floating point number
and is available as the variable QTIME (see
\ref{sec-MCT}).
On all CERN computers, time is counted in
IBM 370/168 seconds.
\end{indentlist}
 
\section{\label{sec-DCHIST}Histograms}
\par
The cards used in connection with the histogram package are described
in detail
in Chapter \ref{sec-HIST}.
For completeness, the cards are listed here also.
 
\subsection{\label{sec-DCHW}HIST: Write histogram file}
\par
The HIST card must be supplied to write histograms and Ntuples to
a
histogram file which can be edited / modified / analyzed in a
subsequent interactive session (PAW).
\begin{description}\item[\bf{Format
}]{\it HIST `data$-$set$-$name $\mid$ parameters'}\end{description}
\begin{indentlist}{ 4.50cm}{ 4.75cm}
\indentitem{data set name}see \ref{sec-DCFT}.
\indentitem{Default file format}HIS
 
\indentitem{parameters:}(optional $-$ described in \ref{sec-HISTW})
\indentitem{UPDA}
\indentitem{NOOV}
\indentitem{DISP}
\indentitem{NREC}
\indentitem{RECL}
\end{indentlist}
 
\subsection{\label{sec-DCHT}HTIT: General histogram title}
\par
\par The HTIT card corresponds to the HBOOK routine HTITLE; it assigns
a general
title to all histograms.
\begin{indentlist}{ 2.50cm}{ 2.75cm}
\indentitem{Format:}
{\it HTIT `This is the general title'}\end{indentlist}
\subsection{\label{sec-DCHP}NOPH: Histogram Printing}
\par
\par Including the NOPH card suppresses the printing
of HBOOK histograms to the terminal or log file; histograms will still
be
written to a direct access file if the HIST card was used.
\begin{indentlist}{ 2.50cm}{ 2.75cm}
\indentitem{Format:}
{\it NOPH}\end{indentlist}
 
\par
\section{\label{sec-DCFIEL}FIEL: Magnetic field}
\par
\par Magnetic field can be set to a given value (only for Monte Carlo studies):
\begin{indentlist}{ 2.50cm}{ 2.75cm}
\indentitem{FIEL 15.}Set magnetic field to 15 KGauss.\end{indentlist}
\par
\section{\label{sec-FRF0}FRF0: Use track fit without vertex detector (POT/DST only)}
\par
If the FRF0 card is included,
the FRFT bank with NR=0 (which has track parameters found without
hits
from the vertex detector)
will be used to fill the charged track
variables rather than FRFT NR=2. Only FRFT NR=2 is available on the MiniDST.

\par
\section{\label{sec-FR12} FR10 or FR12: Use unsmeared track hits (Monte Carlo POTS only)}
\par

To be used by experts only.

Monte-Carlo only: for datasets produced with JULIA 306 and after, the TPC hits are smeared to
better reproduce the data. The track banks FRFT/0 and  FRFT/2  contain the parameters of the tracks fitted from the
smeared hits and are transmitted to ALPHA variables.

If the data card FR12 (resp. FR10) is given, the ALPHA variables are filled with  parameters of tracks fitted from
unsmeared hits  with (resp. without) VDET.  This allows to make some comparisons w.r.t the smeared hits.

FR10 or FR12 is not compatible with FRF0. Cannot work on MINIs.

\par
\section{\label{sec-DCWF}Weight factors for calorimeters}
\par
\par Weight factors
for the 3 ECAL stacks can be given by the data card
\begin{indentlist}{ 2.50cm}{ 2.75cm}
\indentitem{CEEW 1. 1. 1. }Set weight factors to 1. for each stack
(default).\end{indentlist}
\par A weight factor for
the HCAL stack can be given by the data card
\begin{indentlist}{ 2.50cm}{ 2.75cm}
\indentitem{CHEW 1.}set weight factor to 1. for HCAL (default).
\end{indentlist}
\section{\label{sec-DCEFLW}EFLW and EFLJ: Energy Flow}
\par
The EFLW card enables the filling of energy flow objects in ALPHA
(see Ch.~\ref{sec-EF}).
By default, the EFLW card selects the ENFLW (Janot) energy flow package.
 
Replacing the EFLW card with the EFLJ card
causes ALPHA to store jets based on energy flow objects in
addition to the energy flow objects themselves. See Ch.~\ref{sec-EF} for details on energy flow packages.
\par
\section{\label{sec-DCPT}Particle table}
\par
\par The cards used in connection with the ALPHA particle table are
described in
detail in Chapter \ref{sec-PT}. For completeness,
the cards are listed here also.
\subsection{\label{sec-DCPMOD}PMOD: Modify particle attributes}
\par
\par
\begin{description}\item[\bf{Format}]
{\it PMOD `part$-$name antipart$-$name ' mass charge life$-$time width}
\end{description}
\begin{indentlist}{ 3.75cm}{ 4.00cm}
\indentitem{Parameters:}
\indentitem{'part$-$name antipart$-$name'}see \ref{sec-PTH}.
The attributes of a particle and its antiparticle are
modified at the same time. If a particle is its own anti$-$
particle, the same name has to be given twice.
\indentitem{mass charge life$-$time width:}
Real numbers (with decimal point).
The charge of the antiparticle is set to $-$charge. If less than
four numbers are given, the remaining particle attributes
are not changed.
\end{indentlist}
\subsection{\label{sec-DCPNEW}PNEW: New particles}
\par
\par Modify attributes of an existing particle.
\begin{description}\item[\bf{Format
}]
{\it PNEW `part$-$name antipart$-$name ' mass charge life$-$time width}
\end{description}
Same parameters and format as PMOD; used to create new particles.
\par
\subsection{\label{sec-DCPTRA}PTRA: Modify particle names in the
MC particle table}
\par
The PTRA card can be used to assign an arbitrary particle name to
a
specific MC integer code.
\begin{description}\item[\bf{Format
}]{\it PTRA `part$-$name antipart$-$name' iMCcode iMCanticode}\end{description}
\begin{indentlist}{ 3.75cm}{ 4.00cm}
\indentitem{Parameters:}
\indentitem{'part$-$name antipart$-$name'}see \ref{sec-PTPTRA}.
denote the names for the particle and its antiparticle
which have to be used inside the ALPHA program.
\indentitem{iMCcode:}
integer particle code used in the MC generator
(WITHOUT decimal point and NOT included in apostrophes.)
\indentitem{iMCanticode:}
integer particle code used by the MC
generator for the corresponding antiparticle.
\end{indentlist}
\par
\section{\label{sec-DCSYNT}SYNT: Syntax Check}
\par
\par The general structure of the BOS card reading routines does not
allow
for a thorough syntax check of data cards.
To prevent long jobs from dying as a result of syntax errors, ALPHA
provides a
facility to check the data cards.
If the data card
\begin{indentlist}{ 2.50cm}{ 2.75cm}
\indentitem{SYNT}
is given, then
 
\begin{itemize}
\item all data cards are read in;
\item the existence (or, if required, the non$-$existence) of all
input/output files is checked;
\item NO files (except the log file) are created or modified even
if the
log file indicates otherwise;
\item NO events are processed.
\end{itemize}
\end{indentlist}

\section{\label{sec-DCHUNK}BPER,BPWT,BSIZ,SIBE : chunk-by-chunk beam position for MCarlo}
\par
 
The ALPHA variables XGETBP, QVTXBP(I), QVTEBP(I), and QVTSBP(I) (see chapter \ref{sec-MR})
provide chunk-by-chunk beam position information for real data.  They also
provide effective beam positions for correctly simulating the size of the
luminous region in Monte Carlo.  For Monte Carlo, the default behavior is
to use the luminous region size parameters (database bank ALRP) for the
geometry year of the Monte Carlo file.
 
The BPER card allows the user to
override the default.  The parameters from the year(s) specified on this
card are then used; each event is assigned at random to a particular year.
The years (in the LEP~1 era) are weighted according to the numbers of qqbar
events in the MAYB/PERF runs in the VD run selection.
 
Different weights may be specified by means of the BPWT card .  This card is mandatory for
specifying the relative weights of LEP~2 periods, if two or more periods
are selected with BPER .
 
Arbitrary beam sizes may be simulated by
means of the BSIZ card.
 
The x,y beam position uncertainties for Monte Carlo datasets may be smeared using the SIBE data card.

\par
The BPER, BPWT, BSIZ and SIBE cards have no effect when real data is analyzed.
 
\begin{indentlist}{ 4.50cm}{ 4.75cm}
\indentitem{BPER [per1 [per2...]]}
 
\par
        per1,per2... are run periods as specified in the
  first column of ADBR.  Years may be given as ``1992'' or ``92''
  or ``9200''.  Other run periods have the form ``yymm'', such as
  ``9510''.  A list of valid run periods is printed (and
  execution terminates) if the BPER card is given with invalid
  or no run periods.
 
\indentitem{BPWT [iwt1 [iwt2...]]}
 
        iwt1,iwt2,... are *integer* weights for the periods
  listed on the BPER card.  An error results if there is no
  accompanying BPER or if it has a different number of tokens.
  The weights may have an arbitrary normalization.
 
\indentitem{BSIZ $\sigma_x$  $\sigma_y$ [$\sigma_z$]}
 
        $\sigma_x$ and $\sigma_y$ are floating point numbers giving the
  desired sizes of the luminous region in cm, transmitted to QVTXBP(1) and (2).  The BSIZ card
  overrides all values in ALRP.  The value of $\sigma_z$ , if given,
  is returned in QVTXBP(3) but has no other effect.  If $\sigma_z$
  is not given, the value 1 is returned in QVTXBP(3).
 
  Notes: the beam position uncertainty is set to zero if BSIZ is
  given.  BPER/BPWT cards are ignored if BSIZ is given.

\indentitem{SIBE meanx $\sigma_x$  meany $\sigma_y$}

 Allows to perform a smearing of the x and y beam position uncertainties for Monte Carlo datasets.
 meanx (resp. meany) is the central value of the error, $\sigma_x$  (resp $\sigma_y$) is the gaussian dispersion   
 around this mean value.
 Units are in $10^{-3}$ cm. The results are returned in QVTEBP(i).  



\end{indentlist}
{\bf Examples :}
 
\par
1. To simulate 1993 and 1994 beam size conditions with natural relative
   weights:\
 
{\bf BPER 1993 1994}
\par
 
2. To simulate 1993 and 1994 with equal weights:\
 
{\bf BPER 1993 1994}
 
{\bf BPWT   50   50}
\par
3. To simulate a luminous region size of 120 x 7 microns:
 
{\bf BSIZ 0.0120 0.0007 }
\par 
4. To smear the beam position uncertainties as follows: 

 error in x 27.4 microns with 2 microns dispersion,  

 error in y 23.3 microns, 4.7 microns dispersion:

{\bf SIBE 2.74 0.2 2.33 0.47}
         


\section{\label{sec-DCBOBS}BOBS: Beam spot information using LEP BOMs}
\par

If this data card is given, the  ALPHA variables XGETBP, QVTXBP(I), QVTEBP(I), and QVTSBP(I) (see chapter \ref{sec-MR})
are filled with the values obtained from the LEP BOMs for data in 1996 and after. The data card has no effect for LEP1 data
or for Monte Carlo data.

This data card cannot work on datasets produced with a JULIA version before 284, which build the bank BLQP containing the
necessary information. 

\par
\begin{description}\item[\bf{Format}]
{\it BOBS sigmax  sigmay  ibomonly idebug}
\end{description}

\begin{indentlist}{ 3.75cm}{ 4.00cm}
\indentitem{Parameters:}
\indentitem{sigmax:}
Real number equal to the LEP BOM+QSO x intrinsic resolution in cm (default : 0.0040, i.e. 40 microns)
\indentitem{sigmay:}
Real number equal to the LEP BOM+QSO y intrinsic resolution in cm (default : 0.0010, i.e. 10 microns)
\indentitem{ibomonly:}
integer flag  = 1 if one wants to use only the LEP BOM + QSO information\\
              = 0 to use the weighted average between the LEP BOM + QSO information and the VDET chunk-by-chunk information
\indentitem{idebug:}
integer flag = 1 to get a debug printout of the BOM informations, = 0 otherwise.
\end{indentlist}

To know the status of the ALPHA beam spot position for a given event when using this card, one has to use
the subroutine QFILBP$\_$STATUS (see \ref{sec-BOMSTAT} on p.~\pageref{sec-BOMSTAT}).


\section{\label{sec-DCQFND}QFND: Calling the QFNDIP package}
\par
     The QFNDIP package of D. Brown et al. is described in the ALEPH Note 92$-$047,  PHYSIC 92$-$042.
 It  performs a precise determination of
the interaction point on an event$-$by$-$event basis. It needs the beam constraint
from the event$-$chunk beam position computed in the routine QFGET$\_$BP (see sec. \ref{sec-CHUNKINF}).
\par
All necessary input banks are read by ALPHA at initialisation time.
The results of QFNDIP are available as the
main vertex variables  QVX, QVCHIF and QVEM described in sec. \ref{sec-TVAVA} on p. ~\pageref{sec-TVAVA}.
\par
You should NOT call yourself neither QFGET$\_$BP nor QFNDIP. QFGET$\_$BP is called
automatically for each event; QFNDIP is called when you put the QFND card.

\subsection{QFND card:}

If the data card:
\begin{indentlist}{ 2.50cm}{ 2.75cm}
\indentitem{QFND}
is given, QFNDIP is called automatically and the results are stored in the
main vertex variables    (sec \ref{sec-TVAVA} on p. ~\pageref{sec-TVAVA}).
\end{indentlist}
\par
  {\bf Warnings:} 
  The event$-$chunk beam position
and  therefore the interaction point determined by QFNDIP is NOT available for real data
taken in 1989 and 1990. It is available for all Monte-Carlo data.
 
              QFNDIP must be called before calling the b$-$tagging package QIPBTAG (sec \ref{sec-OARQIPB}).

\subsection{DWIN card:}
\par

This data cards allows to use in QFNDIP a selection of charged tracks fulfilling specified cuts on
their longitudinal components.

\begin{description}\item[\bf{Format}]
{\it DWIN ilong  window}
\end{description}
\begin{indentlist}{ 3.75cm}{ 4.00cm}
\indentitem{Parameters:}
\indentitem{ilong}
if = 1 use tracks with : Distance(track-jet) $<$ window\\
if = 2 use tracks with : Abs(Distance(track-jet)) $<$ window\\
\indentitem{window}
cut value in cm
\end{indentlist}

If DWIN is absent, or if ilong = 1 and window=0., the results are the standard QFNDIP ones.

Recommended values for LEP 2 data:

{\bf DWIN}  1  0.035


\section{\label{sec-DCSPCA}Special Cards:}
\par
\par Hereafter are described special data cards, to be used only by specialists.
\begin{indentlist}{ 2.50cm}{ 2.75cm}
\indentitem{MEXT}Forces the muon track extrapolation through HCAL and muon chambers.
This works only on POTs/DSTs.                                            
\indentitem{EFOU}Forces the writing of the Energy Flow bank EFOL on the output tape.
See sec \ref{sec-EFLWJ} for details.
\indentitem{REV0}Forces the V0s to be redone with VDET coordinates, the result being put in the
bank YV0V, NR=3, even if old V0s are present (YV0V NR = 0 or 1). This works only on POTs/DSTs.
Please note that all MINIs produced since June 1993 have been done with this option.
See sec \ref{sec-TVAYV0V} for details.
\indentitem{REPG}Calls automatically the JULIA routine GAPGAC to rebuild from scratch the photon bank PGAC.
This works only on POTs/DSTs, and if JULIA has been linked.
Please note that all MINIs produced  since December 1994 have been done with this option.
See sec \ref{sec-TVAPGAC} for details.
\indentitem{DRGA}Forces the dropping of the PGAC bank read from the data. Mandatory if one wants to rebuild
properly this bank using the REPG card.
\indentitem{MINA 'list'}  add non-standard banks to a private MINI, see Appendix \ref{sec-miniapp} for details.
\indentitem{NOSC}Removes the SICAL clusters from the energy flow objects created by ENFLW.
See sec \ref{sec-EFLWJ} for details.
\indentitem{NOPX}Do not rebuild the pad dE/dx bank PTPX when reading a POT.
\indentitem{NSID}Suppresses the call to the SICAL cleaning routine SICLID in the Energy Flow package.
Valid only for data processed after May 12th, 1999.
See sec \ref{sec-EFLWJ} for details.

\indentitem{NQVJ}Suppresses the internal jet finding of QVSRCH, to use user-defined jets.
See sec \ref{sec-OARVSRC} for details.
\indentitem{PCOR 0}Get the ``sagitta correction" made for all charged tracks - see QPCORR (Sect. \ref{sec-QPCOR})
for details.
\indentitem{PCOR 1}Get the ``sagitta correction" made only for charged tracks having VDET coordinates -
 see QPCORR (Sect. \ref{sec-QPCOR})
for details.
 
\end{indentlist}
