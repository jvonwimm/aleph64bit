\chapter{\label{sec-TVA}ALPHA Track and Vertex Attributes}
\par
Not all of the attributes listed in this chapter are available
when using the Mini-DST.  See Appendix \ref{sec-miniapp}, as well as
the
Mini-DST User's Guide, for a list of variables which are filled
from the MINI.
\par
The units used throughout
ALPHA are cm, sec, GeV, GeV/c, GeV/c$^2$, kG.
\par
\section{\label{sec-TVATA}``Track'' attributes.}
\par
These quantities are defined for all ALPHA ``tracks'' (e.g., charged
tracks, cal.
objects, MC truth, etc.) ``I'' always refers to the ALPHA ``track''
number.
 
\subsection{\label{sec-TVABA}Basic attributes}
\par
\begin{indentlist}{ 2.00cm}{ 2.25cm}
\indentitem{QP (I)}P = momentum of vector I.
\indentitem{QX (I)}$x$ momentum component
\indentitem{QY (I)}$y$ momentum component
\indentitem{QZ (I)}$z$ momentum component
\indentitem{QE (I)}Energy
\indentitem{QM (I)}Mass (use QMASV0 for V0 mass; see below)
\indentitem{QCH (I)}CHarge
\indentitem{KCH (I)}NINT (QCH(I)) (be careful with quarks)
\end{indentlist}
For charged tracks, the pion mass is assumed; the mass can be changed
with QVSETM (see \ref{sec-QVM}).
For angles and more kinematics quantities, see \ref{sec-MK}.
 
\subsection{\label{sec-TVAV0M}V0 Mass}
\par
\begin{indentlist}{ 3.00cm}{ 3.25cm}
\indentitem{QMASV0 (I,'name')}Mass of V0 with hypothesis `name'
\end{indentlist}
The function QMASV0(I,'name') provides the mass for a given V0
hypothesis, where `name' is the name from the ALPHA particle table
or the abbreviation listed here:
\begin{itemize}
\item 'K0S' or  `K0'
\item 'Lam0' or `LA'
\item 'Lam\#0' or `AL'
\item 'GAMMA' or `GA'.
\end{itemize}
This function can be used only for KFV0T $
\leq
$ I $
\leq
$ KLV0T.
See also QIDV0, Sec. \ref{sec-QIDV0}.
\par
 The argument 'name' in QMASV0 can be given in lower or upper case .
\subsection{\label{sec-TVATM}Track error covariance matrix}
\par
\begin{indentlist}{ 3.00cm}{ 3.25cm}
\indentitem{XSIG (I)}{\bf .TRUE.} if covariance matrix available
\indentitem{QSIG (I,N,M)}element (N,M) of the covariance matrix
N,M = 1,2,3,4 in the order QX,QY,QZ,QE
\indentitem{QSIGEE (I)}Error$^2$ on energy
\indentitem{QSIGE (I)}Error on energy
\indentitem{QSIGPP (I)}Error$^2$ on momentum
\indentitem{QSIGP (I)}Error on momentum
\indentitem{QSIGMM (I)}Error$^2$ on mass
\indentitem{QSIGM (I)}Error on mass
The mass error is not defined for particles with mass = 0.
\end{indentlist}
QSIG (I,1,1) is set to  $-1$
if the matrix is not available.
 
\subsection{\label{sec-TVAD}Distance to the beam position:}
\par
Available for charged reconstructed tracks.
\begin{indentlist}{ 3.00cm}{ 3.25cm}
\indentitem{QDB (I)}distance of closest approach to beam axis
\indentitem{QDBS2 (I)}error$^2$ on QDB
\indentitem{QZB (I)}z coordinate of track point where QDB is measured
\indentitem{QZBS2 (I)}error$^2$ on QZB
\indentitem{QBC2 (I)}$\chi^2$ due to QDB and QZB.
\end{indentlist}
The coordinates of the beam position used to compute these
values are QVTXBP(I)  (see~\ref{sec-CHUNKINF} on p. ~\pageref{sec-CHUNKINF}).

\par
For more geometrical track attributes, see sections \ref{sec-TVAFRFT}
and \ref{sec-MK}.
 
\subsection{\label{sec-TVASC}Stability code}
\par
\begin{indentlist}{ 3.00cm}{ 3.25cm}
\indentitem{KSTABC (I)}Stability code
\end{indentlist}
The stability code is designed to avoid double counting when making
loops over Monte Carlo particles. The possible values of KSTABC are:
\begin{indentlist}{ 1.25cm}{ 1.50cm}
\indentitem{ 1}Particle does not decay.
\indentitem{ 2}Neutral particle that decays in the calorimeter volume.
Charged particle that decays in the TPC or calorimeter volume. Here,
TPC and calorimeter volumes are full cylinders (including the beam
pipe region).
\indentitem{ 3}One of the ancestors of this stable particle has
interacted with matter. Energy and momentum are NOT conserved.
\indentitem{ 0}Decay products of ``stable'' particles including all
garbage in the calorimeter.
\indentitem{$-$1}Particle decays immediately (resonance etc.).
\indentitem{$-$2}Particle decays with finite decay length but before
reaching
the detector volume (see above).
\indentitem{$-$3}Particle interacts with matter before reaching
the detector volume.
The decay products do not conserve energy and momentum.
\end{indentlist}
 
A loop over all MC particles with  KSTABC $>$ 0
selects the generation of decay
particles which
will probably be visible in the detector -- energy is never counted
twice.
The energy sum of these particles gives the total generated
energy only if
no particle interacted with
matter inside the detector volume.
A loop over MC particles with KSTABC = 1, 2, and $-$3
is similar, but it always gives the generated total energy.
\par
 
\subsection{\label{sec-TVATPN}Test a particle's name}
\par
\begin{indentlist}{ 5.75cm}{ 6.00cm}
\indentitem{XPEQU (I,'part$-$name')}
{\bf = .TRUE.} if track I is a particle with the name
`part$-$name'.
\indentitem{XPEQOR (I,'part$-$name')}
 
{\bf = .TRUE.} if track I is a particle with the name
`part$-$name' or if it is the corresponding antiparticle.
\indentitem{XPEQAN (I,'part$-$name',IANTI)}
 
{\bf = .TRUE.} if track I is a particle with the name
`part$-$name' and if IANTI = 0.
{\bf = .TRUE.} if track I is the antiparticle corresponding to
`part$-$name' and if IANTI is not equal to 0.
\end{indentlist}
 {\bf Important remark :}  These functions MUST  be called
 with 'part$-$name' being a member of the official list of ALPHA particle names given in Appendix E.
  They are CASE-SENSITIVE and do not work if you supply the wrong
 particle name .
\par
The same functions exist for
integer particle codes IPC = KPART ('part$-$name') instead of
the particle names (see \ref{sec-ADT}):
\begin{verbatim}
        XCEQU (I, IPC)
        XCEQOR (I, IPC)
        XCEQAN (I, IPC, IANTI)
\end{verbatim}
 
\subsection{\label{sec-TVATPSO}Test if particles are based on the
same object}
\par
\begin{indentlist}{ 3.25cm}{ 3.50cm}
\indentitem{XSAME (I,J)}{\bf = .TRUE.}
if tracks I and J or one of their daughters,
granddaughters, etc. are based on the same object (see
\ref{sec-AS}) or, in other words, belong to the same family (see
\ref{sec-QLO}). I and J must both be reconstructed tracks or
MC particles; they may, however, belong to different
Lorentz frames. XSAME uses the same bit masks as the
lock algorithm.
XSAME(IJET,ITK)
can be used for testing whether a track ITK belongs to
a given jet (see \ref{sec-QJA}).
An example how to use XSAME in reconstructing decay chains
is given in \ref{sec-ADS}, example 2.
\end{indentlist}
\subsection{\label{sec-TVAFP}Flags, pointers, etc.}
\par
Pointers to other tracks and to vertices: see ch. \ref{sec-A}.
\begin{indentlist}{ 3.00cm}{ 3.25cm}
\indentitem{KTN (I)}GALEPH/ JULIA/ ENFLW  track number.
                    This means:
\begin{itemize}
\item for MCarlo ``truth'' particles (I=KFREV to KLREV) KTN(I) is the row number of ALPHA track I in the
                    FKIN bank;
\item  for reconstructed tracks (I=KFCHT to KLCHT) KTN(I) is the row number in the FRFT bank;
\item                     for CAL Objects it is the row number in the PECO or PHCO banks; 
\item    for ENFLW objects it is the row number in the
                    EFOL bank.
\end{itemize}

\indentitem{KCLASS(I)}
Track class:
\begin{itemize}
\item  $-$1 (= KRECO) for reconstructed tracks
\item  $-$2 (= KMONTE) for MC truth
\item = 0: track attributes = 0
\item $>$ 0: Lorentz frame. See \ref{sec-AD}.
\end{itemize}
\indentitem{KTPCOD (I)}track's Particle Code
\indentitem{CQTPN (I)}track's particle name (12 char.).
= ` ' if particle code = 0
\indentitem{KLUNDS (I)}LUND status code (MC particles only!); defined only if the event generator is JETSET or is interfaced
 with JETSET.
\indentitem{XMC (I)}{\bf .TRUE.} if MC particle
\indentitem{KRDFL (I,IFLAG) }Integer value of user flag IFLAG (IFLAG=1$-$18).
Flag is set to IVAL with CALL QSTFLI(I,IFLAG,IVAL); see
\ref{sec-USFL}.
\indentitem{QRDFL (I,IFLAG) }Floating$-$point value of user flag
IFLAG (IFLAG=1$-$18).
Flag is set to VAL with CALL QSTFLR(I,IFLAG,VAL); see
\ref{sec-USFL}.
\end{indentlist}
 
\section{\label{sec-TVATRD}``Track'' related detector data}
\par
These mnemonic symbols give access to information in BOS banks
corresponding to an ALPHA ``track''.
These symbols return the integer or floating
point value 0 if detector data are not available for a track.
The names of these mnemonic
symbols follow the same convention as the HAC parameters.
 
\subsection{\label{sec-TVAFRFT}Global geometrical track fit: Bank
FRFT}
\par
\par If the FRF0 card is present in the ALPHA cards file, the NR=0
version of the FRFT bank (track parameters determined without vertex
detector coordinates) will be used.  Otherwise, the NR=2 version
of FRFT (TPC + ITC + VDET tracks) will be used.
\par
\begin{indentlist}{ 3.00cm}{ 3.25cm}
\indentitem{XFRF (I)}{\bf .TRUE.} if track fit data are available
for track I
\indentitem{QFRFIR (I)}Inverse radius of curvature in x$-$y projection
Signed positive if track bends counterclockwise, negative
if track bends clockwise
\indentitem{QFRFTL (I)}Tangent of dip angle
\indentitem{QFRFP0 (I)}Phi at closest approach to the z axis
\indentitem{QFRFD0 (I)}Distance of closest approach to z axis
\indentitem{QFRFZ0 (I)}z coordinate of track point where QFRFD0 is
measured
Note: QDB and QZB (see \ref{sec-TVAD})
correspond to the closest approach to the beam axis.
\indentitem{QFRFAL (I)}Multiple scattering angle between TPC and ITC
\indentitem{QFRFEM (I,N,M)}
 
Element N,M of the error covariance matrix
N,M = 1 to 6 in the order IR,TL,PH,D0,Z0,AL.
Note that for data processed before April 1993 the error matrix is valid at the innermost
point used in the track fit, and therefore does not include
multiple scattering in material before the tracking chambers.
For data processed - or reprocessed - after April 1993 the error matrix is valid at the interaction point
and therefore includes the effect of multiple scattering .
\indentitem{QFRFC2 (I)}$\chi^2$ of helix fit
\indentitem{KFRFDF (I)}Number of degrees of freedom
\indentitem{KFRFNO (I)}Option flag for track fit
\end{indentlist}
 
\subsection{\label{sec-TVAFRTL}Number of coordinates used for the
global fit: Bank FRTL}
\par
\begin{indentlist}{ 3.00cm}{ 3.25cm}
\indentitem{KFRTNV (I)}Number of coordinates in Vdet
\indentitem{KFRTNI (I)}Number of coordinates in ITC
\indentitem{KFRTNE (I)}Number of coordinates in ITC in following spirals
\indentitem{KFRTNT (I)}Number of coordinates in TPC
\indentitem{KFRTNR (I)}Number of coordinates in TPC in following spirals
\end{indentlist}
\subsection{\label{sec-TVAFRID}Charged$-$particle identification:
Bank FRID}
\par
\begin{indentlist}{ 3.00cm}{ 3.25cm}
\indentitem{KFRIBP (I)}Bit pattern for tracking devices
\indentitem{KFRIDZ (I)}Dead zone pattern for tracking devices. To be used with care: the TPC bits were
not filled properly until JULIA 281 (July 1996) and the VDET bits are always 0 for all datasets since Summer 1995.
\indentitem{KFRIBC (I)}Bit pattern for calorimeters
\indentitem{KFRIDC (I)}Dead zone pattern for calorimeters
\indentitem{QFRIPE (I)}Electron probability
\indentitem{QFRIPM (I)}Muon probability
\indentitem{QFRIPI (I)}Pion probability
\indentitem{QFRIPK (I)}Kaon probability
\indentitem{QFRIPP (I)}Proton probability
\indentitem{QFRINK (I)}No Kink probability
\indentitem{KFRIQF (I)}Track Quality Flag from UFITQL
\indentitem{XFRIQF (I)}{\bf .TRUE.} if KFRIQF(I) = 1 or 3
\end{indentlist}
 
\subsection{\label{sec-TVATEXS}dE/dx data: Bank TEXS}
\par
Note: These functions return uncalibrated numbers. In general,
dE/dx information should be accessed with subroutines QDEDX and
QDEDXM (see \ref{sec-OARDEDX}).
\begin{indentlist}{ 3.00cm}{ 3.25cm}
\indentitem{XTEX (I)}{\bf .TRUE.} if dE/dx is available for track
I
\indentitem{KNTEX (I)}Number of TPC sectors on track I
\end{indentlist}
In the following, N is
the loop index of: {\it DO 10 N = 1, KNTEX(I)}
\begin{indentlist}{ 3.00cm}{ 3.25cm}
\indentitem{KTEXSI (I,N)}Sector slot number
\indentitem{QTEXTM (I,N)}Truncated Mean of dE/dx measurements
\indentitem{QTEXTL (I,N)}Useful Track Length for dE/dx
\indentitem{KTEXNS (I,N)}Number of Samples used for dE/dx
\indentitem{QTEXAD (I,N)}Average Drift length of samples
\end{indentlist}
 
\subsection{\label{sec-TVAEIDT}Electron identification: Bank EIDT}
\par
\begin{indentlist}{ 3.00cm}{ 3.25cm}
\indentitem{XEID (I)}{\bf .TRUE.}
if electron identification is available for track I
\indentitem{KEIDIF (I)}Quality flag
\indentitem{QEIDRI (I,N)}R(N) estimator, N = 1 ... 7.
N = 1: Energy balance; N = 2: compactness;
N = 3,4: long. profile; N = 5: dE/dx; N = 6: Dtheta barycenter;
N = 7: Dphi barycenter.
\indentitem{QEIDEC (I)}Corrected energy with electron hypothesis
\indentitem{KEIDIP (I)}Particle hypothesis (= 1 if electron)
\indentitem{QEIDEI (I,N)}Energy in centered storeys stack N
\end{indentlist}
 
\subsection{\label{sec-TVAHMAD}Muon $-$ HCAL association: Bank HMAD}
\par
\begin{indentlist}{ 3.00cm}{ 3.25cm}
\indentitem{XHMA (I)}{\bf .TRUE.} if HCAL data are available for track
I
\indentitem{KHMANF (I)}Number of Fired planes
\indentitem{KHMANE (I)}Number of Expected fired planes
\indentitem{KHMANL (I)}Number of Fired planes within Last ten planes
\indentitem{KHMAMH (I)}Mult Hits: number of clusters in last ten planes
\indentitem{KHMAIG (I)}IGeomflag: flag of possible dead zone
\indentitem{QHMAED (I)}Energy Deposit in corresponding HCAL storey
\indentitem{QHMACS (I)}$\chi^2$
\indentitem{KHMAND (I)}Number of Degrees of freedom
\indentitem{KHMAIE (I)}Expected bit map
\indentitem{KHMAIT (I)}True bit map
\indentitem{KHMAIF (I)}Preliminary identification flag
\end{indentlist}
\subsection{\label{sec-TVAMCAD}Muon chamber data: Bank MCAD}
\par
\begin{indentlist}{ 3.00cm}{ 3.25cm}
\indentitem{XMCA (I)}{\bf .TRUE.}
if muon chamber data are available for track I
 
N = 1,2: Int/Ext chambers
\indentitem{KMCANH (I,N)}Number of associated hits
\indentitem{QMCADH (I,N)}Minimum distance hit$-$track
\indentitem{QMCADC (I,N)}Cutoff on hit$-$track distance
\indentitem{QMCAAM (I)}Min. angle between extrapolated
and measured (in
muon ch.) track
\indentitem{QMCAAC (I)}cutoff on minimum angle
\end{indentlist}
\subsection{\label{sec-TVAMUID}QMUIDO Muon Identification: Bank MUID}
\par
\begin{indentlist}{ 3.00cm}{ 3.25cm}
\indentitem{XMUI (I)}{\bf .TRUE.}
if QMUIDO information is available for track I
\indentitem{KMUIIF (I)}Identification Flag
\indentitem{QMUISR (I)}Sum of HCAL residuals
\indentitem{QMUIDM (I)}Distance between track and closest muon chamber hit.
\indentitem{KMUIST (I)}FRFT track number of shadowing track
\end{indentlist}
\subsection{\label{sec-TVAPECO}ECAL objects: Bank PECO}
\par
\begin{indentlist}{ 3.00cm}{ 3.25cm}
\indentitem{XPEC (I)}{\bf .TRUE.} if ECAL data (PECO) are available
for calorimeter object ``track'' I
\indentitem{QPECER (I)}Raw energy.
\indentitem{QPECE1 (I)}Fraction of energy in stack 1
\indentitem{QPECE2 (I)}Fraction of energy in stack 2
\indentitem{QPECTH (I)}Theta
\indentitem{QPECPH (I)}Phi
\indentitem{QPECEC (I)}Energy corrected for geometrical effects
\indentitem{KPECKD (I)}Region code $-$ see ALEPH 88$-$134
\indentitem{KPECCC (I)}Correction code $-$ see bank description
\indentitem{KPECRB (I)}Relation bits $-$ see bank description
\indentitem{KPECPC (I)}PCOB number of associated cal. object
\end{indentlist}
\subsection{\label{sec-TVAPEPT}ECAL objects: Bank PEPT}
\par
\begin{indentlist}{ 3.00cm}{ 3.25cm}
\indentitem{XPEP (I)}{\bf .TRUE.} if ECAL data (PEPT) are available
for calorimeter object ``track'' I
\indentitem{QPEPT1 (I)}Theta in stacks 1 and 2
\indentitem{QPEPP1 (I)}Phi in stacks 1 and 2
\indentitem{QPEPT3 (I)}Theta in stack 3
\indentitem{QPEPP3 (I)}Phi in stack 3
\end{indentlist}
\subsection{\label{sec-TVAPHCO}HCAL objects: Bank PHCO}
\par
\begin{indentlist}{ 3.00cm}{ 3.25cm}
\indentitem{XPHC (I)}{\bf .TRUE.} if HCAL data (PHCO) are available
for calorimeter object ``track'' I
\indentitem{QPHCER (I)}Raw energy
\indentitem{QPHCTH (I)}Theta
\indentitem{QPHCPH (I)}Phi
\indentitem{QPHCEC (I)}Energy corrected for geometrical effects
\indentitem{KPHCKD (I)}Region code $-$ see ALEPH 88$-$134
\indentitem{KPHCCC (I)}Correction code $-$ see bank description
\indentitem{KPHCRB (I)}Relation bits $-$ see bank description
\indentitem{KPHCPC (I)}PCOB number of associated cal. object
\end{indentlist}
\subsection{\label{sec-TVAYV0V}Reconstructed V0s: Bank YV0V}
\par
\begin{indentlist}{ 3.00cm}{ 3.25cm}
\indentitem{XYV0 (I)}{\bf .TRUE.} if V0 data are available for track I
\indentitem{KYV0K1 (I)}JULIA/FRFT track number of positive track from V0 ( NOT the ALPHA track number ! )
\indentitem{KYV0K2 (I)}JULIA/FRFT track number of negative track from V0 ( NOT the ALPHA track number ! )
\indentitem{QYV0VX (I)}V0 x coordinate
\indentitem{QYV0VY (I)}V0 y coordinate
\indentitem{QYV0VZ (I)}V0 z coordinate
\indentitem{QYV0X1 (I)}First constraint on V0 mass (r in ALEPH 88$-$46)
\indentitem{QYV0X2 (I)}Second constraint on V0 mass (b in ALEPH 88$-$46)
\indentitem{QYV0C2 (I)}$\chi^2$ of V0 vertex fit
\indentitem{KYV0IC (I)}Fit hypothesis (see YV0V bank description)
\indentitem{QYV0DM (I)}Minimum distance between helices
\indentitem{QYV0S1 (I)}Psi angle for $+$ track from V0
\indentitem{QYV0S2 (I)}Psi angle for $-$ track from V0
\end{indentlist}
 
 
\subsection{\label{sec-TVAEFOL}Energy Flow: Bank EFOL}
\par
\begin{indentlist}{ 3.00cm}{ 3.25cm}
\indentitem{XEFO (I)}{\bf .TRUE.} if energy flow (EFOL)
data are available for ``track'' I ( of the EFT section )
\indentitem{KEFOTY (I)}Type of energy flow object (see Sec.~\ref{sec-EFLWM})
\indentitem{KEFOLE (I)}PECO number of associated ECAL object
\indentitem{KEFOLT (I)}FRFT number of associated charged track
\indentitem{KEFOLH (I)}PHCO number of associated HCAL object
\indentitem{KEFOLC (I)}PCOB number of associated calorimeter object
\indentitem{KEFOLJ (I)}EJET number of associated jet
\end{indentlist}
\subsection{\label{sec-TVAPCQA}Neutral objects from PCPA: Bank PCQA}
\par
\begin{indentlist}{ 3.00cm}{ 3.25cm}
\indentitem{XPCQ (I)}{\bf .TRUE.} if PCQA data are available for track I
\indentitem{KPCQNA (I)}NAture of neutral object (see Sec.~\ref{sec-EFLWP})
\end{indentlist}
 
\section{\label{sec-GAMPHO} Photon attributes}
\par
    The photon-finding package has been introduced in 1992 in the ALEPHLIB and has been deeply
  modified many times, getting each time a new name and giving
  different output banks.
 
\par
   The GAMPEK photon algorithm is described in the reports ALEPH 93$-107$  and ALEPH 94$-057$.
\par
       Users who want to study only $\pi^0$ production may use directly the QPI0DO algorithm described in
     \ref{sec-OARQPI0} , page ~\pageref{sec-OARQPI0}.
 
\par
  In the descriptions below , the index I refers to the ``tracks''  of the GAT section and
  runs from KFGAT to KLGAT.
 
 
\subsection{\label{sec-TVAEGPC}Photons from GAMPEC: Output bank EGPC (Obsolete since July 1993)}
\par
 This bank was written on POTs, DSTs and MINIs before July 1993 
 (JULIA versions 264 and before, MINI version 8 and before). It should not be used any more.
  Please refer to previous ALPHA manuals for the description of the related variables.
 
\subsection{\label{sec-TVAPGPC}Photons from GAMPEX: Output bank PGPC (Obsolete since December 1994)}
\par
 This bank was written on POTs , DSTs and MINIs between July 1993 and  December 1994 by
 JULIA versions 265 to 271 or MINI 9.
  Please refer to previous ALPHA manuals for the description of the related variables.
     
\subsection{\label{sec-TVAPGAC}Photons from GAMPECK: Output bank PGAC}
\par
 This bank is written on POTs , DSTs and MINIs since December 1994
 by JULIA version 272 and after, or MINI version 10 and after.
 
 On these datasets, the photons from the old GAMPEC package (bank EGPC, see above) or GAMPEX (bank PGPC, see above),
 are no more available.
\par
\begin{indentlist}{ 3.00cm}{ 3.25cm}
\indentitem{XPGAC (I)}{\bf .TRUE.} if GAMPECK data are available for ``track'' I of the GAT section
\indentitem{QX (I)}Momentum of photon , x component
\indentitem{QY (I)}Momentum of photon , y component
\indentitem{QZ (I)}Momentum of photon , z component
\indentitem{QE (I)}Energy  of photon
\indentitem{QPGPR1 (I)}Energy fraction in stack 1
\indentitem{QPGPR2 (I)}Energy fraction in stack 2
\indentitem{QPGPF4 (I)}Energy fraction in 4 central towers
\indentitem{QPGPDM (I)}Distance to the closest track (cm)
\indentitem{KPGAST (I)} NST1$+100*$NST2+$10000*$NST3, NSTi=number of storeys in stack i
\indentitem{KPGPQU(I)} QUality flag
\begin{itemize}
\item CRCK + 10*DST1 + 100*DST2 + 1000*DST3
\item DSTi = 1 if dead storey(s) in stack i
\item CRCK = 1 if photon in crack region
\end{itemize}
\indentitem{QPGPQ1 (I)}Quality estimator 1 for photon
\indentitem{QPGPQ2 (I)}Quality estimator 2 for photon
\indentitem{QPGPM1 (I)}1st moment from CLMOMS analysis
\indentitem{QPGPM2 (I)}2nd moment from CLMOMS analysis
\indentitem{QPGPMA (I)}Pi0 mass estimated from CLMOMS analysis
\indentitem{QPGPER (I)}Raw energy of photon
\indentitem{QPGPTR (I)}Raw Theta  of photon
\indentitem{QPGPPR (I)}Raw Phi    of photon
\indentitem{QPGAEF (I)}Expected fraction in 4 towers
\indentitem{QPGAEF (I)}Geometrical correction
\indentitem{QPGAZS (I)}Zero suppression
\indentitem{QPGAPL (I)}Probability to be a fake photon of electromagnetic origin
\indentitem{QPGAPH (I)}Probability to be a fake photon of hadronic origin
\indentitem{KPGAPN (I)}Row number , in PGAC , of parent giving a fake photon
\indentitem{KPGAFA (I)}Flag for fake discrimination
\indentitem{KPGAPE (I)}Row number of corresponding PECO cluster
\end{indentlist}
 
\section{\label{sec-TVAVA}Vertex attributes}
\par
The following attributes are
all vertices.
The argument (IVX) always refers to vertex IVX.
\par
 
{\bf Warning : JULIA or QFNDIP Main vertex ?}
 
For the Main vertex (IVX = KFREV and KVTYPE(IVX) = 1, see below) the quantities described below may have been
obtained either from the JULIA vertexing or from the QFNDIP package if you asked
for it through the data card QFND ( sec \ref{sec-DCQFND}). See the entry QVCHIF below to know which
package was used to give the Main vertex positions .

\begin{indentlist}{ 3.00cm}{ 3.25cm}
\indentitem{QVX (IVX)}x position
\indentitem{QVY (IVX)}y position
\indentitem{QVZ (IVX)}z position
\indentitem{KVN (IVX)}JULIA/GALEPH vertex number
\indentitem{KVTYPE(IVX)}vertex type 

\begin{itemize}

\item  = 1 for the Main vertex; 
 
\item  = 0 for standard reconstructed V0s from YV0V

 (IVX = KFREV,KLREV) 

     or for MC ``truth'' vertices (IVX = KFMCV,KLMCV)

\item  = 2 for a secondary vertex fitted by any of the KVFITx routines, see sec \ref{sec-QFIT} on p.~\pageref{sec-QFIT}.

\item  = 3 for a ``Long V0'' vertex from the new tracking

\item  = 4 for a Nuclear Interaction vertex from the new tracking

\item  = 5 for a Kink vertex from the new tracking

\end{itemize}

 Vertex types 3,4,5 are defined only for datasets produced with JULIA version 302 and after.


\indentitem{QVCHIF(IVX)}Chisquare of the vertex fit.

 Special meaning for the Main vertex:

 Always 0 if the Main vertex was found by JULIA.

 If the Main vertex was found by QFNDIP, this chisquare is positive.

 If QFNDIP was called and was unable to find a Main vertex, this Chisquare is set
 to a huge negative value.
\indentitem{QVEM (IVX,N,M)}element (N,M) of the covariance matrix
N,M = 1,2,3 in the following order: QVX,QVY,QVZ. 

QVEM (IVX,1,1) is set to $-1.$
if the error matrix is not available.
\end{indentlist}
 
See Section \ref{sec-AV} for pointers between ALPHA tracks and
vertices.


\section{\label{sec-TVKINK}Special attributes for a Kink Vertex}
\par

\par
\fbox{CALL QVKINK(IVX,PKINK,PTOUT,PLOUT,IHYPK,IER)}
\par
This routine allows to get special attributes defined only for Kink vertices
produced by JULIA version 302 and after. These quantities are available from POTs, DSTs and MINIs.
\par
{\bf Input arguments:}
\begin{indentlist}{ 3.50cm}{ 3.75cm}
\indentitem{IVX}ALPHA vertex number of a kink vertex (IVX must be between KFKIV and KLKIV, see sec \ref{sec-alvert}).
\end{indentlist}
\par
{\bf Output arguments:}
\begin{indentlist}{ 3.50cm}{ 3.75cm}
\indentitem{IER} = 0 if IVX is really a kink vertex, = 1 if not.           
\indentitem{PKINK} Momentum of parent track at the kink position
\indentitem{PTOUT} Pt of outgoing charged particle relative to parent
\indentitem{PLOUT} Pl of outgoing charged particle relative to parent
\indentitem{IHYPK} Hypothesis of mass cuts satisfied by this decay:  
\begin{itemize}
\item = 1  $\pi        \rightarrow  \mu \nu$
\item = 2  $K          \rightarrow  \mu \nu$
\item = 3  $K          \rightarrow  \pi \pi^{0}$
\item = 4  $\Sigma^{+} \rightarrow  p \pi^{0}$
\item = 5  $\Sigma^{+} \rightarrow  n \pi^{+} $
\item = 6  $\Sigma^{-} \rightarrow  n \pi^{-} $
\item = 7  $\Xi^{-}    \rightarrow  \Lambda \pi^{-}$
\item = 8  $\Omega^{-} \rightarrow  \Lambda K^{-}$
\item = 9  $\Omega^{-} \rightarrow  \Xi^{0} \pi^{-}$
\item = 10 $\Omega^{-} \rightarrow  \Xi^{-} \pi^{0}$
\item = 0  if none of the above hypotheses are satisfied
\end{itemize}

\end{indentlist}


\par
\fbox{CALL QKINKT(NKINKS,LISTKINKS)}
\par
This routine allows to get the number and list of charged tracks ending in Kink vertices
produced by JULIA version 302 and after. 

\par
{\bf Output arguments:}
\begin{indentlist}{ 3.50cm}{ 3.75cm}
\indentitem{NKINKS} = number of kink vertices in the event.
\indentitem{LISTKINKS(I)} = list of charged tracks ending in the I=1,NKINKS kink vertices; must be dimensioned
                            to 50 in the calling routine.
\end{indentlist}


